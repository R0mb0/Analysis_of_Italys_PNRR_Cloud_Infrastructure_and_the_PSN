\documentclass[hidelinks,aspectratio=169]{beamer}
\usepackage[italian]{babel} 
\usepackage[utf8]{inputenc} 
\usepackage{fourier} 

%Slide colors
\usetheme{Madrid}
%\usecolortheme{beaver}

% Images
\usepackage{graphicx}
\usepackage{caption}
\usepackage{subcaption}
\usepackage{float}
\graphicspath{{Images}}

% Stop hyphenation
\usepackage[none]{hyphenat}

% Minipages in the same line
\usepackage{tabularx}

% Coloring links
\usepackage{xcolor}

% Enumerate abc
\usepackage{enumerate}

% Redefines caption setup in way to remove "Figure:"
\usepackage{caption}
\captionsetup[figure]{labelformat=empty}

% License
\usepackage[
type={CC},
modifier={by-nc-sa},
version={4.0},
]{doclicense}

%------------------- Commands zone --------------------

%Command to zoom in
\usepackage{mwe}
\makeatletter
\newsavebox\zb@x
\newcounter{z@@m}
\usepackage{calc}
\newdimen\B@r\newdimen\P@r
\newdimen\@zw\newdimen\@zh\newdimen\@zd

\newcommand{\zoombox}[2][0]{%
	\leavevmode%
	\sbox\zb@x{#2}%
	\setlength\B@r{1pt*\ratio{\wd\zb@x}{\ht\zb@x+\dp\zb@x}}%
	\setlength\P@r{1pt*\ratio{\paperwidth}{\paperheight}}%
	\ifdim\B@r>\P@r\relax%
	\setlength\@zw{\wd\zb@x}\setlength\@zh{\@zw*\ratio{\paperheight}{\paperwidth}}%
	\setlength\@zd{(\@zh-\ht\zb@x-\dp\zb@x)*\real{0.5}+\dp\zb@x}%
	\setlength\@zh{\@zh-\@zd}%
	\else%
	\setlength\@zh{\ht\zb@x+\dp\zb@x}%
	\setlength\@zw{\@zh*\ratio{\paperwidth}{\paperheight}}%
	\setlength\@zh{\ht\zb@x}\setlength\@zd{\dp\zb@x}%
	\fi%
	\makebox[0pt][l]{\makebox[\wd\zb@x][c]{\makebox[\@zw][l]{%
				\pdfdest name {zbfs\thez@@m} fitr
				width  \@zw\space
				height \@zh\space
				depth  \@zd\space
	}}}%
	\pdfdest name {zb\thez@@m} fitr
	width  \wd\zb@x\space
	height \ht\zb@x\space
	depth  \dp\zb@x\space
	\immediate\pdfannot 
	width  \wd\zb@x\space
	height \ht\zb@x\space
	depth  \dp\zb@x\space
	{%
		/Subtype/Link/H/N
		/Border [0 0 #1 [1 2]]
		/A <<
		/S/JavaScript
		/JS (
		if(typeof(zoomed)=='undefined'||!zoomed){
			var lastView=this.viewState;
			if(app.fs.isFullScreen) this.gotoNamedDest('zbfs\thez@@m');
			else this.gotoNamedDest('zb\thez@@m');
			zoomed=true;
		}else{
			this.viewState=lastView;
			zoomed=false;
		}
		)
		>>
	}%
	\usebox{\zb@x}%
	\stepcounter{z@@m}%
} 
\makeatother

%------------------- Header --------------------
\title[	PNRR, PSN e PDND]{\small \textbf{PNRR, PSN e PDND}}
\author[Francesco Rombaldoni]{}
\date{Anno Accademico 2025/2026}

\begin{document}
	
	%------------------------------------------------
	% Title
	%------------------------------------------------
	\begin{frame}
		\vspace*{-5mm}
		\begin{center}
			\hspace*{30mm}\zoombox{\includegraphics[scale=0.2]{logo-uniurb-2016.jpg}}
			\vspace*{2mm}
			\newline
			{\Large UNIVERSITÀ DEGLI STUDI DI URBINO CARLO BO}\\
			\vspace*{0.5mm}
			Dipartimento di Scienze Pure e Applicate\\
			\vspace*{0.5mm}
			Corso di Laurea in Informatica e Innovazione Digitale\\
			\hspace*{10mm}\noindent\rule{110mm}{0.4pt}\newline
			\vspace*{0.5mm}
			Presentazione progetto per \textbf{Sistemi Distribuiti}\\
			\vspace*{5mm}
			\textbf{\large {Trasformazione digitale della PA italiana: \\
					PNRR, PSN e Piattaforma Digitale Nazionale Dati}}
		\end{center}
	\end{frame}
	
	%------------------------------------------------
	% Note PDF
	%------------------------------------------------
	\begin{frame}
		\centering
		\fboxrule=2pt
		\fbox
		{
			\begin{minipage}{0.9\linewidth}
				\small{Documento ottimizzato per la visualizzazione digitale con
					\href{https://get.adobe.com/it/reader/}{\textcolor{blue}{Adobe~Acrobat~Reader}}.}
			\end{minipage}
		}
	\end{frame}
	
	%------------------------------------------------
	% TOC
	%------------------------------------------------
	\begin{frame}
		\tableofcontents
	\end{frame}
	
	%================================================
	\section{Introduzione}
	%================================================
	\begin{frame}{Introduzione}
		\begin{itemize}
			\item Obiettivo: capire la transizione digitale della PA (PNRR)
			\item Tesi: \textbf{governance più centralizzata}, esecuzione \textbf{distribuita (cloud)}
			\item Focus: \textbf{PDND} (interoperabilità) e \textbf{PSN} (infrastruttura)
		\end{itemize}
		\vspace{2mm}
		
		
		\begin{center}
			\zoombox{\includegraphics[scale=0.45]{ecosistema.png}}
		\end{center}
	\end{frame}
	
	%================================================
	\section{Prima della transizione}
	%================================================
	\begin{frame}{Prima della transizione}
		\begin{itemize}
			\item Sistemi a \textbf{silos} (ogni ente ``fa da sé'')
			\item Integrazioni spesso \textbf{bilaterali} (caso per caso)
			\item Sicurezza e continuità: \textbf{molto disomogenee}
		\end{itemize}
		\vspace{2mm}
		% Immagine consigliata: schema a isole / integrazioni punto-punto
		% \includegraphics[width=0.75\linewidth]{prima_vs_dopo.png}
	\end{frame}
	
	\begin{frame}{Transizione: prima vs dopo}
		\centering
		\begin{tabularx}{0.95\linewidth}{X X}
			\textbf{Prima} & \textbf{Dopo (obiettivo)}\\
			Silos e CED frammentati & Cloud + consolidamento (PSN/qualificati)\\
			Accordi bilaterali & Regole comuni + interoperabilità (PDND)\\
			Standard variabili & Standardizzazione + controlli\\
		\end{tabularx}
		\vspace{2mm}
		
		% Immagine consigliata: diagramma a 2 colonne o freccia evolutiva
		% \includegraphics[width=0.8\linewidth]{before_after.png}
	\end{frame}
	
	%================================================
	\section{PNRR e governance}
	%================================================
	\begin{frame}{PNRR e governance}
		\begin{itemize}
			\item PNRR = \textbf{finanziamento + indirizzo} (milestone/target)
			\item Spinta a: \textbf{cloud-first}, piattaforme nazionali, interoperabilità
			\item Effetto: riduzione della frammentazione (ma nuovi trade-off)
		\end{itemize}
	\end{frame}
	
	%================================================
	\section{Piano Triennale 2024--2026}
	%================================================
	\begin{frame}{Piano Triennale 2024--2026}
		\begin{itemize}
			\item \textbf{API-first} + principio \textbf{once-only}
			\item Architetture a \textbf{microservizi} e riuso
			\item Piattaforme nazionali come \textbf{abilitatori}
		\end{itemize}
		
		\vspace{2mm}
		% Immagine consigliata: schema "API-first/once-only" oppure flusso dati cittadino->PA
		% \includegraphics[width=0.8\linewidth]{once_only.png}
	\end{frame}
	
	\begin{frame}{PDND}
		\begin{itemize}
			\item PDND = \textbf{catalogo e-service} + accesso via API
			\item Gestisce: \textbf{authN/authZ}, tracciamento, regole di accordo
			\item Obiettivo: interoperabilità \textbf{scalabile} (non ``uno a uno'')
		\end{itemize}
		\vspace{2mm}
		% Immagine consigliata: schema provider -> PDND -> consumer (catalogo al centro)
		% \includegraphics[width=0.85\linewidth]{pdnd_schema.png}
	\end{frame}
	
	%================================================
	\section{Infrastrutture e cloud}
	%================================================
	\begin{frame}{Cloud Italia: perché (in breve)}
		\begin{itemize}
			\item \textbf{Autonomia} tecnologica e controllo dei dati
			\item \textbf{Resilienza} e continuità dei servizi
			\item Migrazione = occasione per ridurre \textbf{debito tecnico} e rischi
		\end{itemize}
		
		\vspace{2mm}
		% Immagine consigliata: cloud-first pipeline / migrazione (rehost->replatform->rearchitect)
		% \includegraphics[width=0.8\linewidth]{cloud_migration.png}
	\end{frame}
	
	\begin{frame}{PSN: ruolo e idea}
		\begin{itemize}
			\item PSN = infrastruttura nazionale per \textbf{consolidamento} CED
			\item Non ``tutto su PSN'': anche \textbf{cloud qualificati} e modelli ibridi
			\item Governance + sicurezza come valore aggiunto
		\end{itemize}
		
		\vspace{2mm}
		% Immagine consigliata: schema PSN + cloud qualificati + enti
		% \includegraphics[width=0.85\linewidth]{psn_overview.png}
	\end{frame}
	
	%================================================
	\section{PSN 2025: modelli di erogazione}
	%================================================
	\begin{frame}{PSN 2025: modelli (1 messaggio)}
		\begin{itemize}
			\item PSN Managed (es. Oracle Alloy)
			\item Secure Public Cloud (Azure/GCP con governance PSN)
			\item Industry Standard (servizi ``di mercato'' integrati)
		\end{itemize}
		\vspace{2mm}
		% Immagine consigliata: 3 blocchi (Managed / SPC / Industry)
		% \includegraphics[width=0.9\linewidth]{psn_models.png}
	\end{frame}
	
	\begin{frame}{PSN Managed: resilienza}
		\begin{itemize}
			\item Ridondanza su \textbf{fault domain}
			\item Storage e DB con \textbf{replica}
			\item Maintenance senza downtime (obiettivo)
			\item DR inter-region: \textbf{opzionale/configurabile}
		\end{itemize}
		\vspace{2mm}
		% Immagine consigliata: schema region con 3 fault domain + freccia verso region DR
		% \includegraphics[width=0.85\linewidth]{fault_domain_dr.png}
	\end{frame}
	
	\begin{frame}{Secure Public Cloud}
		\begin{itemize}
			\item \textbf{Public cloud} in Italia (hyperscaler)
			\item \textbf{Security \& governance} nel perimetro PSN
		\end{itemize}
		
		\vspace{1mm}
		\begin{itemize}
			\item Chiavi: controllo esterno al CSP
			\item Rete: modello \textbf{hub \& spoke}
			\item Backup nel perimetro PSN
		\end{itemize}
		
		\vspace{2mm}
		% Immagine consigliata: diagramma 2 blocchi (Public cloud + PSN security)
		% \includegraphics[width=0.85\linewidth]{spc_architecture.png}
	\end{frame}
	
	%================================================
	\section{Sicurezza e gestione del rischio}
	%================================================
	\begin{frame}{Sicurezza: da tecnologia a governance}
		\begin{itemize}
			\item Cybersecurity come \textbf{requisito di sistema}
			\item Centralità di: logging, monitoraggio, audit
			\item ACN: indirizzo e rafforzamento capacità nazionali
		\end{itemize}
	\end{frame}
	
	\begin{frame}{Caso operativo}
		\begin{itemize}
			\item Processi e certificazioni (ISO, ALM, DevOps)
			\item \textbf{DR} con test periodici (semestrali/annuali)
			\item \textbf{SOC} e SIEM: centralizzazione log e incident response
			\item \textbf{IAM}: gestione identità e autorizzazioni
		\end{itemize}
		\vspace{2mm}
		% Immagine consigliata: schema SOC (log -> SIEM -> alert -> IR)
		% \includegraphics[width=0.8\linewidth]{soc_siem.png}
	\end{frame}
	
	%================================================
	\section{Centralizzazione vs Ibridazione}
	%================================================
	\begin{frame}{In che senso è ``più centralizzato''?}
		\begin{itemize}
			\item Regole comuni (governance)
			\item Piattaforme nazionali (abilitatori)
			\item Cataloghi, identità, auditing (punti di controllo)
		\end{itemize}
	\end{frame}
	
	\begin{frame}{Perché resta ``distribuito''?}
		\begin{itemize}
			\item Cloud: multi-region, replica, DR
			\item Coesistenza di modelli (PSN + hyperscaler)
			\item Obiettivo: resilienza e scalabilità
		\end{itemize}
	\end{frame}
	
	%================================================
	\section{Rischi e trade-off}
	%================================================
	\begin{frame}{Concentrazione del rischio}
		\begin{itemize}
			\item Nodi critici = \textbf{high-value target}
			\item Maggiore interconnessione = blast radius più alto
			\item Serve segmentazione + controlli continui
		\end{itemize}
	\end{frame}
	
	\begin{frame}{Il fattore umano}
		\begin{itemize}
			\item Credenziali e privilegi (IAM/PAM)
			\item Errori operativi e procedure
			\item Social engineering e supply chain
		\end{itemize}
	\end{frame}
	
	\begin{frame}{Dipendenza operativa}
		\begin{itemize}
			\item Strumenti \textbf{non bastano}: serve qualità reale
			\item Log completi + correlazione + risposta rapida
			\item DR efficace solo con \textbf{prove periodiche}
		\end{itemize}
	\end{frame}
	
	%------------------------------------------------
	% Closing
	%------------------------------------------------
	\begin{frame}{Conclusione}
		\begin{center}
			\Large
			\textbf{La PA evolve verso governance centralizzata e infrastrutture distribuite:}
			\\[2mm]
			standardizzazione e resilienza aumentano, ma crescono anche
			\textbf{dipendenze} e \textbf{rischi di concentrazione}.
		\end{center}
	\end{frame}
	
\end{document}