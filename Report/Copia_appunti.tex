\documentclass[12pt]{article}
\begin{document}
	\section{Analisi del documento\underline{Piano Triennale per l'informatca nella pubblica amministraione}}
	
	Per prima cosa si parla di come gli enti dovrebbero avere un sistema di gestione informatico strutturato, diciamo ad alto livello, quindi vuol dire che in questo momento, gli enti fanno un uso sregolato della tecnologia, ma nel senso che non ci sono regole comuni di interoperabilità. 
	
	L'Approccio emotivo è quello volto alla centralizzazione delle informazioni, siccome si pensa che mettere tutte le cose importanti in un unico luogo ben gestito possa essere meglio rispetto ad avere un sistema decentralizzato... In particolare, Esempi di quello che si vuole ottenere è: 
	
	•
	il Fascicolo Sanitario Elettronico 2.0;
	•
	la Piattaforma Digitale Nazionale Dati – PDND;
	•
	il Sistema degli Sportelli Unici (SSU) riferito a SUAP/SUE;
	•
	l’implementazione nazionale dello Sportello Digitale Unico europeo - SDG;
	•
	l’ecosistema nazionale di approvvigionamento digitale (e-procurement)
	•
	l’Hub del Turismo Digitale (TDH).
	
	Parola chiave dei servizi - Once only e api first
	
	Il governo dietro a questo progetto di centralizzazione vuole anche fornire ai cittadini l'istruzione necessaria per accedere a questo nuovo paradigma informatico. 
	
	Syllabus per le competenze digitali. 
	
	Usare i fondi del PNRR per il raggiungimento di tale scopo
	
	Quindi fino a questo punto io ho appena appreso che, prima di tutto lo stato vorrebbere creare una rete di comunicazione centralizzata tra i vari enti. questo per definire delle strategia do cuminicazione comuni a tutto gli enti. poi, vorrebbe anche facendo ciò rinforzare la sicurezza dei servizi con il sistema once-only e il API first. Dato questo cambiamento, desiderano creare anche dei corsi formativi per la popolazione e per i dipendenti della pa. tutta questa cosa fatta con i soldi del PNRR 
	
	Cloud first -> dentro i pubblici uffici. Necessità di appaltare servizi esterni, in modo da non avere dentro casa elementi sensibili. -> si vede a questo punto la prima contraddizione, si parla di cloud, quindi di sistema distribuito, quando in realtà la volontà è quella di centralizzare tutto. 
	
	Quindi le informazioni sensibili vanno centralizzate (quindi esternalizzate) mentre gli apparati informatici devono essere esternalizzati a favore di soluzioni cloud -> Questo serve per dimminuire la superfice d'attacco. 
	
	Lo stato italiano è codardo -> questo alla fine è un gioco della deresponsabilizzazione. -> la morale è che la PA non la si vede cone un ente informatico valido, per cui, il lavoro va dato ai professionisti. 
	
	La PA si figura nuovamente come collocamento. 
	
	Dice il documento: 
	
Capitolo 3 - Servizi
Negli ultimi anni, la digitalizzazione è diventata una forza trainante per l'innovazione nei servizi pubblici, con gli enti locali al centro di questo cambiamento.
L'adozione di tecnologie digitali è essenziale per migliorare l'efficienza, aumentare la trasparenza e garantire la qualità dei servizi offerti ai cittadini. In questo processo di trasformazione è indispensabile anche definire un framework di riferimento per guidare ed uniformare le scelte tecnologiche. In particolare, l'architettura a microservizi può esser considerata come una soluzione agile e scalabile, che permette di standardizzare i processi digitali e di facilitare anche il processo di change management nelle organizzazioni governative locali.
Per garantire la possibilità a tutti gli Enti di poter cogliere questa enorme opportunità, anche a coloro che si trovano in condizioni di carenze di know-how e risorse, il presente Piano propone e promuove un’evoluzione del modello di interoperabilità passando dalla sola condivisione dei dati a quella della condivisione dei servizi.
I vantaggi dell’utilizzo di un’architettura basata su micro-servizi sono:
•
Flessibilità e scalabilità
•
Agilità nello sviluppo
•
Integrazione semplificata
•
Resilienza e affidabilità
La transizione verso un'architettura a microservizi richiede la consapevolezza che non sia necessario solo un intervento tecnologico ma che richiede soprattutto un controllo per la gestione del cambiamento che, come abbiamo visto nel cap. 1 coinvolge diverse fasi chiave, quali la formazione continua, il coinvolgimento attivo degli stakeholder, il monitoraggio dell'impatto del cambiamento e naturalmente anche una comunicazione efficace.
Per gli enti locali che potrebbero non avere un know-how interno sufficiente, l'architettura a microservizi offre l'opportunità di sfruttare le soluzioni e i servizi già sviluppati da altri enti. Questo approccio non solo consente di colmare il gap informativo interno, ma fornisce anche un vantaggio significativo in termini di risparmio di tempo e ottimizzazione delle risorse.
L'architettura a microservizi, attraverso la condivisione di processi e lo sviluppo once only riduce la duplicazione degli sforzi e dei costi. La condivisione di e-service vede nella Piattaforma Digitale Nazionale Dati Interoperabilità (PDND) il layer focale per la condivisione di dati e processi.
La sostenibilità e la crescita collaborativa nell'ambito dell'architettura a microservizi non si limita al singolo ente locale. In molte situazioni, possono entrare in gioco altre istituzioni a supporto, come Regioni, Unioni o Enti capofila (HUB tecnologici), che possono agire svolgendo un ruolo fondamentale nello sviluppo fornendo soluzioni tecnologiche e/o amministrative, per facilitare l’integrazione e l’implementazione del processo di innovazione. Questo approccio consente agli enti più piccoli di beneficiare delle risorse condivise e delle soluzioni già implementate, accelerando così il processo di digitalizzazione.
Il coinvolgimento attivo delle istituzioni aggregate come facilitatori tecnologici è essenziale per garantire una transizione armoniosa verso l'architettura a microservizi. Guardando al futuro, la sinergia tra enti locali, Regioni e altre istituzioni aggregate pone le basi per un ecosistema digitale coeso, capace di affrontare sfide complesse e di offrire servizi pubblici sempre più efficienti. La
43
collaborazione istituzionale diventa così un elemento fondamentale per plasmare un futuro digitale condiviso e orientato all'innovazione.
E-Service in interoperabilità tramite PDND
Scenario
L’interoperabilità facilita l'interazione digitale tra Pubbliche Amministrazioni, cittadini e imprese, recependo le indicazioni dell'European Interoperability Framework e, favorendo l’attuazione del principio once only secondo il quale la PA non deve chiedere a cittadini e imprese dati che già possiede.
A fine di raggiungere la completa interoperabilità dei dataset e dei servizi chiave tra le PA centrali e locali e di valorizzare il capitale informativo delle pubbliche amministrazioni, nell’ambito del Sub-Investimento M1C1_1.3.1 “Piattaforma nazionale digitale dei dati” del Piano Nazionale di Ripresa e Resilienza, è stata realizzata la Piattaforma Digitale Nazionale Dati (PDND).
La PDND è lo strumento per gestire l’autenticazione, l’autorizzazione e la raccolta e conservazione delle informazioni relative agli accessi e alle transazioni effettuate suo tramite. La Piattaforma fornisce un insieme di regole condivise per semplificare gli accordi di interoperabilità snellendo i processi di istruttoria, riducendo oneri e procedure amministrative. Un ente può aderire alla Infrastruttura interoperabilità PDND siglando un accordo di adesione, attraverso le funzionalità messe a disposizione dell’infrastruttura.
La PDND permette alle amministrazioni di pubblicare e-service, ovvero servizi digitali conformi alle Linee Guida realizzati ed erogati attraverso l’implementazione di API (Application Programming Interface) REST o SOAP (per retrocompatibilità) cui vengono associati degli attributi minimi necessari alla fruizione. Le API esposte vengono registrate e popolano il Catalogo pubblico degli e-service.
La Piattaforma dovrà evolvere recependo le indicazioni pervenute dalle varie amministrazioni e nel triennio a venire dovrà anche:
1.
consentire la condivisione di dati di grandi dimensioni (bulk) prodotti dalle amministrazioni e l’elaborazione di politiche data-driven;
2.
offrire alle amministrazioni la possibilità di accedere ai dati di enti o imprese di natura privata non amministrativa e di integrarsi con i processi di questi ultimi;
3.
permettere alle amministrazioni di essere informate, in maniera asincrona, su eventuali variazioni a dati precedentemente fruiti, abilitando anche una gestione intelligente dei meccanismi di caching locale delle informazioni;
4.
attivare modelli di erogazione inversa, con i quali un ente, potrà erogare e-service, abilitati a ricevere dati da altri soggetti;
5.
abilitare lo scambio dato sia in modalità sincrona che asincrona, permettendo anche il trasferimento di grosse moli di dati, o di pacchetti dati che necessitano di elevati tempi di elaborazione per il confezionamento;
6.
consentire ad una amministrazione di delegare un altro aderente alla piattaforma ad utilizzare per suo conto le funzionalità dell’infrastruttura medesima per la registrazione, la modifica degli e-service sul Catalogo API e la gestione delle richieste di fruizione degli e-service, ivi compresa la compilazione dell’analisi dei rischi;
7.
pubblicare i propri dati aperti attraverso API che siano catalogate secondo le norme pertinenti (DCAT_AP-IT, INSPIRE, …) e che possano essere raccolte nei portali nazionali ed europei.
44
Al fine di sviluppare servizi integrati sempre più efficienti ed efficaci e di fornire a cittadini e imprese servizi rispondenti alle rispettive esigenze, il Dipartimento per la Trasformazione Digitale supporta le PA nell’adozione del Modello di interoperabilità, pianificando e coordinando iniziative di condivisione, anche attraverso protocolli d'intesa e accordi finalizzati a:
•
costituzione di tavoli e gruppi di lavoro;
•
avvio di progettualità congiunte;
•
capitalizzazione di soluzioni realizzate dalla PA in open source o su siti o forum per condividere la conoscenza (Developers Italia e Forum Italia)


Progettazione dei servizi: accessibilità e design
Scenario
Il miglioramento della qualità e dell’inclusività dei servizi pubblici digitali costituisce la premessa indispensabile per l’incremento del loro utilizzo da parte degli utenti, siano questi cittadini, imprese o altre pubbliche amministrazioni.
Nell’attuale processo di trasformazione digitale è essenziale che i servizi abbiano un chiaro valore per l’utente. Questo obiettivo richiede un approccio multidisciplinare nell'adozione di metodologie e tecniche interoperabili per la progettazione di un servizio. La qualità finale, così come il costo complessivo del servizio, non può infatti prescindere da un’attenta analisi dei molteplici layer, tecnologici e organizzativi interni, che strutturano l’intero processo della prestazione erogata, celandone la complessità sottostante.
Ciò implica anche la necessità di un’adeguata semplificazione dei procedimenti e un approccio sistematico alla gestione dei processi interni, sotto il coordinamento del Responsabile per la transizione al digitale, dotato di un ufficio opportunamente strutturato e con il fondamentale coinvolgimento delle altre strutture responsabili dell’organizzazione e del controllo strategico.
È cruciale, inoltre, il rispetto degli obblighi del CAD in materia di progettazione, accessibilità, privacy, gestione dei dati e riuso, al fine di massimizzare l'efficienza dell'investimento di denaro pubblico e garantire la sovranità digitale con soluzioni software strategiche sotto il completo controllo della Pubblica Amministrazione.
Occorre quindi agire su più livelli e migliorare la capacità delle pubbliche amministrazioni di generare ed erogare servizi di qualità attraverso:
•
l'adozione di modelli e strumenti validati e a disposizione di tutti;
•
il costante monitoraggio da parte delle PA dei propri servizi online;
•
l'incremento del livello di accessibilità dei servizi erogati tramite siti web e app mobile;
•
lo scambio di buone pratiche tra le diverse amministrazioni, da attuarsi attraverso la definizione, la modellazione e l’organizzazione di comunità di pratica;
•
Il riuso e la condivisione di software e competenze tra le diverse amministrazioni.
Per incoraggiare tutti gli utenti a privilegiare il canale online rispetto a quello esclusivamente fisico, rimane necessaria una decisa accelerazione nella semplificazione dell'esperienza d’uso complessiva e un miglioramento dell’inclusività dei servizi, nel pieno rispetto delle norme riguardanti l’accessibilità e il Regolamento generale sulla protezione dei dati.
Per il monitoraggio dei propri servizi, le PA possono utilizzare Web Analytics Italia, una piattaforma
nazionale open source che offre rilevazioni statistiche su indicatori utili al miglioramento continuo
dell’esperienza utente.
Per la realizzazione dei propri servizi digitali, le PA possono utilizzare il Design System del Paese, che consente la realizzazione di interfacce coerenti e accessibili by default, concentrando i budget di progettazione e sviluppo sulle parti e i processi caratterizzanti dello specifico servizio digitale.
Contesto normativo e strategico


Formazione, gestione e conservazione dei documenti informatici
Scenario
Le nuove Linee guida sulla formazione, gestione e conservazione dei documenti informatici dell'Agenzia per l'Italia Digitale, adottate ai sensi dell’art. 71 del CAD e in vigore dal 1° gennaio 2022, rappresentano un importante contributo nel rafforzamento e nell’armonizzazione del quadro normativo di riferimento in tema di produzione, gestione e conservazione dei documenti informatici, mirando a semplificare e rendere più accessibile la materia, integrandola ove necessario, per ricondurla in un unico documento sistematico di pratico utilizzo.
Al loro interno sono delineati i necessari adeguamenti organizzativi e funzionali richiesti alle pubbliche amministrazioni, chiamate a consolidare e rendere concreti i principi di trasformazione digitale enunciati nel CAD e nel Testo Unico sulla Documentazione Amministrativa - TUDA.
Le Linee guida costituiscono la premessa fondamentale dell’agire amministrativo in ambiente digitale, in attuazione degli obiettivi di semplificazione, trasparenza, partecipazione e di economicità, efficacia ed efficienza, già prescritti dalla Legge n.241/1990, assicurando la corretta impostazione metodologica per la loro realizzazione nel complesso percorso di transizione digitale.
La Pubblica Amministrazione è tenuta ad assicurare la rispondenza alle Linee guida, adeguando i propri sistemi di gestione informatica dei documenti, al fine di garantire effetti giuridici conformi alle stesse nei processi documentali, nonché ad ottemperare alle seguenti misure:
•
gestione appropriata dei documenti sin dalla loro fase di formazione per il corretto adempimento degli obblighi di natura amministrativa, giuridica e archivistica tipici della gestione degli archivi pubblici, come delineato nel paragrafo 1.11 delle Linee guida;
•
gestione dei flussi documentali mediante aggregazioni documentali informatiche, come specificato nel paragrafo 3.3;
•
nomina dei ruoli e delle responsabilità previsti, come specificato ai paragrafi 3.1.2 e 4.4;
•
adozione del Manuale di gestione documentale e del Manuale di conservazione, come specificato ai paragrafi 3.5 e 4.7;
•
pubblicazione dei provvedimenti formali di nomina e dei manuali in una parte chiaramente identificabile dell’area “Amministrazione trasparente”, prevista dall’art. 9 del d.lgs. 33/2013;
•
rispetto delle misure minime di sicurezza ICT, emanate da AGID con circolare del 18 aprile 2017, n. 2/2017;
•
rispetto delle in materia di protezione dei dati personali, ai sensi dell’art. 32 del Regolamento UE 679/2016 (GDPR);
•
trasferimento dei documenti al sistema di conservazione, ai sensi del paragrafo 4 e dell’art. 44, comma 1-bis, del CAD.
Il corretto assolvimento di tali obblighi incide significativamente non solo sull'efficacia e l'efficienza della Pubblica Amministrazione, migliorando i processi interni e facilitando gli scambi informativi tra le amministrazioni e il settore privato, ma rappresenta anche un elemento fondamentale nella prestazione di servizi di alta qualità ai cittadini e alle imprese, assicurando trasparenza, accessibilità e protezione di dati e documenti.
Nell'ambito delle sue funzioni di vigilanza, verifica, controllo e monitoraggio, e conformemente a quanto stabilito dall'articolo 18-bis del Codice dell'Amministrazione Digitale - CAD, l'Agenzia per l'Italia Digitale ha pianificato di avviare un'attività di monitoraggio riguardante l'adempimento degli obblighi specificati dalle Linee guida.
52
A questo scopo, entro il 2024 verrà sviluppato un modello basato su indicatori chiari e dettagliati, supportato da un accurato percorso metodologico. Questo permetterà di procedere con un monitoraggio efficace e sistematico, da realizzarsi entro il 2025 per le disposizioni sulla Gestione documentale, e entro il 2026 per quelle relative alla Conservazione digitale.
Contesto normativo


Single Digital Gateway
Scenario
Nel triennio precedente è stata attuata la parte core del Regolamento Europeo EU 2018/1724 sul Single Digital Gateway (SDG) che, con l’obiettivo di costruire uno sportello unico digitale a livello europeo per consentire a cittadini e imprese di esercitare più facilmente i propri diritti e fare impresa all’interno dell’Unione europea, ha di fatto messo online le 21 procedure richieste (19 applicabili in Italia) delle pubbliche amministrazioni direttamente coinvolte in quanto titolari dei servizi.
Il Regolamento, entrato in vigore il 2 ottobre 2018, infatti, ha stabilito le norme per:
1.
l’istituzione e la gestione di uno sportello digitale unico per offrire ai cittadini e alle imprese europee un facile accesso a:
a.
informazioni di alta qualità;
b.
procedure efficienti e interamente online;
c.
servizi di assistenza e di risoluzione dei problemi;
2.
l'uso di procedure da parte di utenti transfrontalieri e l’applicazione del principio once only in accordo con le specifiche normative dei differenti Stati Membri.
A dicembre 2023 AGID ha completato le attività di integrazione e collaudo delle componenti architetturali nazionali SDG, sia per l'interoperabilità tra PA italiane, sia per quella tra PA italiane e quelle degli Stati Membri. Le pubbliche amministrazioni competenti per i procedimenti amministrativi relativi alle procedure (di cui all’Allegato II del Regolamento UE 2018/1724) hanno adeguato i propri procedimenti amministrativi alle specifiche tecniche di implementazione del Single Digital Gateway.
Dopo aver reso disponibile online i servizi relativi delle procedure previste, le attività per il Single Digital Gateway del triennio 2024-2026 riguarderanno prevalentemente azioni di mantenimento,
54
monitoraggio e miglioramento della qualità e dell’accesso ai servizi digitali offerti dallo Sportello per l’Italia, in particolare:
1.
monitoraggio delle componenti nazionali e dei servizi delle PA competenti per garantire l’operatività di tutta la filiera coinvolta nell'attuazione dei processi nazionali e trans-frontalieri del Single Digital Gateway (SDG) attraverso la progettazione e sviluppo di un Operation Center, capace di mettere a sistema tutti gli stakeholder coinvolti che dovranno lavorare in maniera sinergica e standardizzata nella gestione dei processi di operation. Il sistema prevede la predisposizione di un servizio di supporto continuativo di gestione del portafoglio delle applicazioni realizzate e rilasciate, che comprende la presa in carico e la risoluzione delle richieste utente pervenute ad AGID da cittadini e pubbliche amministrazioni (help desk);
2.
supporto alla diffusione dello sportello e del suo utilizzo presso i cittadini e le imprese: rientrano in questa azione attività di supporto alla diffusione dei servizi e attività statistiche di monitoraggio e analisi riguardanti le visite degli utenti alle pagine web impattate dalle singole procedure, nonché al catalogo dei servizi.


Capitolo 4 - Piattaforme
Come per i precedenti Piani, il Piano triennale per l’informatica nella Pubblica Amministrazione 2024-26 prende in esame l’evoluzione delle piattaforme della Pubblica Amministrazione, che offrono funzionalità fondamentali nella digitalizzazione dei processi e dei servizi della PA.
La raggiunta maturità di alcune piattaforme, già presentate nelle precedenti edizioni del Piano, permette qui di focalizzarsi sui servizi che esse offrono a cittadini, a imprese e ad altre amministrazioni, in continuità con quanto descritto nel capitolo precedente “Servizi”.
Nella prima parte di questo capitolo, quindi, si illustrano le piattaforme nazionali che erogano servizi a cittadini e imprese: PagoPA, AppIo, Send, Spid e Cie, NoiPA, Fascicolo sanitario elettronico e SUAP/SUE.
L’obiettivo riferibile a tutte queste piattaforme è comune, si tratta di migliorare i servizi già erogati nei termini che verranno dettagliati nei risultati attesi e nelle linee di azione. In questa sezione, la descrizione di ciascuna piattaforma riporterà lo stesso obiettivo mentre gli altri elementi descrittivi saranno specifici della piattaforma presa in esame. Nella seconda parte di questo capitolo verranno descritte le piattaforme che attestano attributi ed infine si parlerà di basi di dati di interesse nazionale.
Piattaforme nazionali che erogano servizi a cittadini/imprese o ad altre PA
Scenario
pagoPA
pagoPA è la piattaforma che consente ai cittadini di effettuare pagamenti digitali verso la Pubblica Amministrazione in modo veloce e intuitivo. pagoPA offre la possibilità ai cittadini di scegliere tra i diversi metodi di pagamento elettronici in base alle proprie esigenze e abitudini, grazie all’opportunità per i singoli enti pubblici di interfacciarsi con diversi attori del mercato e integrare i propri servizi di incasso con soluzioni innovative. L’obiettivo di pagoPA, infatti, è portare a una maggiore efficienza e semplificazione nella gestione dei pagamenti dei servizi pubblici, sia per i cittadini sia per le amministrazioni, favorendo una costante diminuzione dell’uso del contante.
AppIO
L’app IO è l’esito di un progetto open source nato con l’obiettivo di mettere a disposizione di enti e cittadini un unico canale da cui fruire di tutti i servizi pubblici digitali, quale pilastro della strategia del Governo italiano per la cittadinanza digitale. La visione alla base di IO è mettere al centro il cittadino nell’interazione con la Pubblica Amministrazione, attraverso un’applicazione semplice e intuitiva disponibile direttamente sul proprio smartphone. In particolare, l’app IO rende concreto l’articolo 64 bis del Codice dell’Amministrazione Digitale, che istituisce un unico punto di accesso per tutti i servizi digitali, erogato dalla Presidenza del Consiglio dei Ministri.
SEND
La piattaforma SEND - Servizio Notifiche Digitali (anche noto come Piattaforma Notifiche Digitali di cui all'art. 26 del decreto-legge 76/2020 s.m.i.) rende più veloce, economico e sicuro l’invio e la ricezione delle notifiche a valore legale: permette infatti di riceverle, scaricare i documenti notificati e pagare eventuali spese direttamente online su SEND o nell'app IO.
59
SEND solleva gli enti da tutti gli adempimenti legati alla gestione delle comunicazioni a valore legale e riduce l’incertezza della reperibilità del destinatario.
SPID
L’identità digitale SPID è la soluzione che permette di accedere a tutti i servizi online della Pubblica Amministrazione con un'unica identità digitale. Attraverso credenziali classificate su tre livelli di sicurezza, abilita ad accedere ai servizi, ai quali fornisce dati identificativi certificati.
SPID è costituito come insieme aperto di soggetti pubblici e privati che, previo accreditamento da parte dell'Agenzia per l'Italia Digitale, gestiscono i servizi di registrazione e di messa a disposizione delle credenziali e degli strumenti di accesso in rete nei riguardi di cittadini e imprese.
A dicembre 2023 sono state rilasciate ai cittadini oltre 36 milioni e mezzo di identità digitali SPID, che hanno permesso nel 2023 di totalizzare oltre 1.000.000.000 di autenticazioni a servizi online di pubbliche amministrazioni e privati. Attualmente la federazione SPID è composta da più di 15.000 fornitori di servizi pubblici e 177 fornitori di servizi privati.
Nell’ambito del PNRR il sub-investimento M1C1 1.4.4 “Rafforzamento dell'adozione delle piattaforme nazionali di identità digitale (SPID, CIE) e dell'Anagrafe nazionale della popolazione residente (ANPR)”, di cui è soggetto titolare il Dipartimento per la Trasformazione Digitale della Presidenza del Consiglio dei Ministri, include fra le sue finalità che i gestori delle identità SPID assicurino l'innalzamento del livello dei servizi, della qualità, sicurezza e di interoperabilità degli stessi stabiliti dalle Linee guida AGID, come previsto dall’art. 18 bis del D.L. 24/02/2023 n. 13, convertito dalla L. 21/04/2023 n. 41.
A tal fine, è necessario che il Sistema SPID evolva in base alle seguenti indicazioni:
•
attuazione delle “Linee guida OpenID Connect in SPID” (Determinazione del Direttore Generale di AGID n. 616/2021) comprensive dell’Avviso SPID n. 41 del 23/3/2023 versione 2.0 e il “Regolamento - SPID OpenID Connect Federation 1.0” (Determinazione del Direttore Generale di AGID n. 249/2022);
•
attuazione delle “Linee guida operative per la fruizione dei servizi SPID da parte dei minori” (Determinazione del Direttore Generale di AGID n. 133/2022);
•
attuazione delle “Linee guida recanti le regole tecniche dei Gestori di attributi qualificati” (Determinazione del Direttore Generale di AGID n. 215/2022);
•
promozione dell’utilizzo dello SPID dedicato all’uso professionale per l’accesso ai servizi online rivolti a professionisti e imprese.
CIE
L’identità digitale CIE (CIEId), sviluppata e gestita dall’Istituto Poligrafico e Zecca dello Stato, consente la rappresentazione informatica della corrispondenza tra un utente e i suoi attributi identificativi, ai sensi del CAD, verificata attraverso l’insieme dei dati raccolti e registrati in forma digitale al momento del rilascio della CIE. La CIEId è comprovata dal cittadino attraverso l’uso della CIE o delle credenziali rilasciate dal Ministero.
Alla data di metà dicembre 2023 sono state rilasciate ai cittadini oltre 40 milioni di Carte di Identità Elettroniche, che hanno permesso nel 2023 di totalizzare circa 32.000.000 di autenticazioni a servizi online di pubbliche amministrazioni e privati. Attualmente la federazione CIE è composta da più di 10.000 fornitori di servizi pubblici e circa 100 fornitori di servizi privati.
Come sancito dal Decreto 8 settembre 2022 “Modalità di impiego della carta di identità elettronica”, sono previste le seguenti evolutive sul servizio CIEId:
60
1.
Ampliamento del set di attributi forniti tramite autenticazione con CIEId, come previsto dall’art. 6;
2.
ampliamento delle funzionalità del portale del cittadino, come previsto dall’art. 14, tra cui la possibilità di visualizzare, esprimere o revocare la volontà in merito alla donazione di organi e tessuti;
3.
implementazione dei servizi correlati al NIS (Numero Identificativo Servizi), come previsto dall’art. 17;
4.
implementazione di una piattaforma di firma elettronica qualificata remota attraverso l’utilizzo della CIE;
5.
implementazione dell’integrazione con il sistema ANPR, al fine di ricevere giornalmente i dati afferenti ai soggetti deceduti e procedere al blocco tempestivo della CIEId;
6.
sviluppo di un meccanismo di controllo genitoriale per consentire un accesso controllato ai servizi online offerti ai minori.
NoiPA
NoiPA è la piattaforma dedicata a tutto il personale della Pubblica Amministrazione, che offre servizi evoluti per la gestione, integrata e flessibile, di tutti i processi in ambito HR, inclusi i relativi adempimenti previsti dalla normativa vigente. Inoltre, attraverso il portale Open Data NoiPA, è possibile la piena fruizione dell’ampio patrimonio informativo gestito, permettendo la consultazione, in forma aggregata, dei dati derivanti dalla gestione del personale delle pubbliche amministrazioni servite.
Fascicolo Sanitario Elettronico
Il Fascicolo Sanitario Elettronico (FSE 2.0) ha l‘obiettivo di garantire la diffusione e l’accessibilità dei servizi di sanità digitale in modo omogeneo e capillare su tutto il territorio nazionale a favore dei cittadini e degli operatori sanitari delle strutture pubbliche, private accreditate e private.
La verifica formale e semantica della corretta implementazione e strutturazione dei documenti secondo gli standard ha lo scopo di assicurare omogeneità a livello nazionale per i servizi del FSE 2.0 disponibili ai cittadini e ai professionisti della Sanità.
Attraverso interventi sistematici di formazione, si intende superare le criticità legate alle competenze digitali dei professionisti del sistema sanitario, innalzandone significativamente il livello per un utilizzo pieno ed efficace del FSE 2.0.
SUAP e SUE
Nel panorama della Pubblica Amministrazione, gli Sportelli Unici per le Attività Produttive (SUAP) e per l'Edilizia (SUE) assumono un ruolo centrale come punto di convergenza per imprese, professionisti e cittadini nell'interazione con le istituzioni, nell’ambito degli adempimenti previsti per le attività produttive (quali, ad esempio, la produzione di beni e servizi, le attività agricole, commerciali e artigianali, le attività turistiche alberghiere ed extra-alberghiere, i servizi resi dalle banche e dagli intermediari finanziari e i servizi di telecomunicazione, ecc.) e gli interventi edilizi. Si tratta di due pilastri fondamentali in un contesto in continua evoluzione, dove la digitalizzazione si configura non solo come una necessità imprescindibile, ma anche come una leva strategica fondamentale per favorire la competitività delle imprese, stimolare la crescita economica del Paese e ottimizzare la tempestività nell'evasione delle richieste. In questo scenario, la semplificazione e l'accelerazione dei procedimenti amministrativi diventano così il mezzo con cui costruire un futuro in cui le opportunità digitali diventino accessibili a tutti.
61
Nell’ambito delle iniziative previste dal Piano Nazionale di Ripresa e Resilienza (PNRR), è stato avviato il percorso di trasformazione incentrato sulla digitalizzazione e la semplificazione dei sistemi informatici, partendo dalla redazione delle Specifiche tecniche, elaborate attraverso il lavoro congiunto del Gruppo tecnico (istituito dal Ministero delle Imprese e del Made in Italy e dal Dipartimento della Funzione Pubblica e coordinato dall’Agenzia per l’Italia Digitale), le quali delineano l’insieme delle regole e delle modalità tecnologiche che i Sistemi Informatici degli Sportelli Unici (SSU) devono adottare, per la gestione ottimale dei procedimenti amministrativi riguardanti le attività produttive, conformemente alle disposizioni del DPR 160/2010 e ss.mm.ii.
La fase operativa di questo percorso è stata condotta partendo da un’attenta analisi della situazione esistente, rafforzata, successivamente, dalla somministrazione di un questionario di valutazione, volto ad identificare la maturità tecnologica iniziale degli sportelli unici, grazie alla diretta collaborazione delle amministrazioni coinvolte. Attualmente, è terminata la raccolta delle informazioni, perfezionata con altre attività di indagine, come la consulta dei fornitori dei servizi IT relativi alle piattaforme, i tavoli tematici regionali e il coinvolgimento di altri stakeholder e si sta procedendo con la definizione dei piani di intervento, da realizzarsi attraverso risorse finanziarie messe a disposizione dal Dipartimento della Funzione Pubblica, tramite la pubblicazione di bandi/stipula di accordi per l’adeguamento delle piattaforme.
In tale percorso di trasformazione, che vedrà impegnate le pubbliche amministrazioni nel prossimo triennio, per garantire il raggiungimento delle milestone definite nell’ambito del PNRR, deve essere assicurato il supporto tecnico necessario all’adeguamento delle soluzioni informatiche alle Specifiche tecniche, attraverso la condivisione delle conoscenze e dell’esperienza maturata nel campo, utili a fornire una corretta interpretazione delle stesse durante la fase di realizzazione degli interventi.
Contesto normativo e strategico
In materia di Piattaforme esistono una serie di riferimenti, normativi o di indirizzo, cui le Amministrazioni devono attenersi. Di seguito si riporta un elenco delle principali fonti, generali o specifiche, della singola piattaforma citata nel capitolo:


Piattaforme che attestano attributi
Scenario
Negli ultimi anni le iniziative intraprese dai vari attori coinvolti nell’ambito del Piano, hanno favorito una importante accelerazione nella diffusione di alcune delle principali piattaforme abilitanti, in termini di adozione da parte delle PA e di fruizione da parte degli utenti. Il Piano descrive lo sviluppo di nuove piattaforme e il consolidamento di quelle già in essere attraverso l’aggiunta di nuove funzionalità. Tali piattaforme rendono disponibili i dati di settore ai cittadini e PA, consentono di razionalizzare i servizi per le amministrazioni e di semplificare tramite l’utilizzo delle tecnologie digitali l’interazione tra cittadini e PA (per la Piattaforma Digitale Nazionale Dati – PDND).
Ad esempio, nel luglio 2023 la Piattaforma INAD è andata in esercizio, in consultazione, sia tramite il sito web sia tramite le API esposte su PDND, attualmente in esercizio. La piattaforma è quindi a disposizione per entrambe le modalità di fruizione, da parte delle pubbliche amministrazioni. Si invitano pertanto le PA a fruire dei relativi servizi, compatibilmente con il loro dimensionamento.
In questo ambito vengono attuate le seguenti Piattaforme che hanno la caratteristica di attestare attributi anagrafici e di settore.
ANPR: è l’Anagrafe Nazionale che raccoglie tutti i dati anagrafici dei cittadini residenti in Italia e dei cittadini italiani residenti all’estero, aggiornata con continuità dagli oltre 7900 comuni italiani, consentendo di avere un set di dati anagrafici dei cittadini certo, accessibile, affidabile e sicuro su cui sviluppare servizi integrati ed evoluti per semplificare e velocizzare le procedure tra Pubbliche amministrazioni e con il cittadino.
Sul portale ANPR, nell’area riservata del cittadino, sono attualmente disponibili i servizi che consentono al cittadino di:
•
visualizzare i propri dati anagrafici;
•
effettuare una richiesta di rettifica per errori materiali;
•
richiedere autocertificazioni precompilate con i dati anagrafici presenti in ANPR;
•
richiedere un certificato anagrafico in bollo o in esenzione (sono disponibili 15 tipologie differenti di certificati);
•
comunicare un cambio di residenza;
•
visualizzare il proprio domicilio digitale, costantemente allineato con l’Indice Nazionale dei Domicili Digitali (INAD);
•
comunicare un punto di contatto (mail o telefono).
A dicembre 2022 sono stati resi disponibili i servizi per consentire, da parte dei Comuni, l'invio dei dati elettorali dei cittadini in ANPR. Attualmente oltre il 97% dei comuni italiani hanno aderito ai servizi, inviando i dati elettorali dei cittadini.
70
La presenza dei dati elettorali in ANPR consentirà ai cittadini di visualizzare nell’area riservata i dati relativi alla propria posizione elettorale e richiedere certificati di godimento dei diritti politici e di iscrizione nelle liste elettorali.
Inoltre, consentirà di verificare in tempo reale la posizione elettorale di un cittadino da parte di altre Amministrazioni che ne abbiano necessità per fini istituzionali. Una prima applicazione si avrà con l’integrazione dei servizi ANPR con la Piattaforma Referendum, piattaforma online che consentirà la sottoscrizione di proposte referendarie e di iniziativa popolare, verificando in tempo reale la posizione elettorale del cittadino sottoscrittore.
Al fine di agevolare lo sviluppo di sistemi integrati ed evoluti, che semplifichino e velocizzino le procedure tra le Pubbliche Amministrazioni, ANPR ha reso disponibili 28 e-service sulla Piattaforma Nazionale Digitale Dati (PDND) - Interoperabilità, consentendo la consultazione dei dati ANPR da parte di altri Enti aventi diritto, nel rispetto dei principi del Regolamento Privacy.
In aggiunta, l’integrazione dell’ANPR con i servizi dello Stato civile digitale ha un rilievo centrale e strategico nel processo di digitalizzazione della Pubblica Amministrazione e costituisce un significativo strumento di semplificazione per i Comuni e per i cittadini. Si prevede, infatti, la completa digitalizzazione dei registri dello Stato civile tenuti dai Comuni (nascita, matrimonio, unione civile, cittadinanza e morte), con conseguente eliminazione dei registri cartacei, e la conservazione dei relativi atti digitali in un unico archivio nazionale del Ministero dell’Interno, permettendone la consultazione a livello nazionale e offrendo la possibilità di produrre estratti o certificati tramite il sistema centrale, senza doverli richiedere necessariamente al Comune che li ha generati. Alcuni Comuni pilota ad ottobre 2023 hanno iniziato ad utilizzare i servizi resi disponibili da ANPR, formando atti digitali di stato civile con effetti giuridici.
ANPR si sta integrando con le anagrafi settoriali del lavoro, della pensione e del welfare e ogni nuova anagrafe che abbia come riferimento la popolazione residente sarà logicamente integrata con ANPR.
In questo contesto, per rafforzare gli interventi nei settori di istruzione, università e ricerca, accelerare il processo di automazione amministrativa e migliorare i servizi per i cittadini e le pubbliche amministrazioni, sono istituite due Anagrafi:
•
ANIST: l'Anagrafe nazionale dell'istruzione, a cura del Ministero dell’Istruzione e del Merito
•
ANIS: l'Anagrafe nazionale dell'istruzione superiore, a cura del Ministero dell’Università e della Ricerca.
Le due Anagrafi mirano ad assicurare:
•
La centralizzazione dei dati attualmente distribuiti su tutto il territorio italiano in oltre 10.000 scuole (ANIST) e 500 istituti di formazione superiore (ANIS);
•
la disponibilità e l’accesso ai dati per:
o
scuole e istituti di formazione superiore (IFS), al fine di facilitare il reperimento delle informazioni relative al percorso scolastico e/o accademico dei propri studenti, efficientando le procedure di iscrizione;
o
cittadini, al fine rendere possibile, attraverso il Portale dedicato, la consultazione online dei dati relativi al proprio percorso scolastico e/o accademico, anche a fini certificativi;
o
PA per fini istituzionali;
o
soggetti privati autorizzati, per gli scopi previsti dalla legge.
•
l'interoperabilità con altre banche dati (es. con ANPR per la gestione dei dati anagrafici degli studenti, eliminando duplicazioni e rischi di disallineamento);
•
il riconoscimento nell’UE e extra-EU dei titoli di studio.
Per l’avvio progettuale di ANIST si attende la conclusione del relativo iter normativo.

Intelligenza artificiale per la Pubblica Amministrazione
Scenario
Per sistema di Intelligenza Artificiale (IA) si intende un sistema automatico che, per obiettivi espliciti o impliciti, deduce dagli input ricevuti come generare output come previsioni, contenuti, raccomandazioni o decisioni che possono influenzare ambienti fisici o virtuali. I sistemi di IA variano nei loro livelli di autonomia e adattabilità dopo l'implementazione (Fonte: OECD AI principles overview).
Figura 3 - Sistema di intelligenza artificiale (Fonte OECD)
L'intelligenza artificiale ha il potenziale per essere una tecnologia estremamente utile, o addirittura dirompente, per la modernizzazione del settore pubblico. L'IA sembra essere la risposta alla crescente necessità di migliorare l'efficienza e l'efficacia nella gestione e nell'erogazione dei servizi pubblici. Tra le potenzialità delle tecnologie di intelligenza artificiale si possono citare le capacità di:
•
automatizzare attività di ricerca e analisi delle informazioni semplici e ripetitive, liberando tempo di lavoro per attività a maggior valore;
•
aumentare le capacità predittive, migliorando il processo decisionale basato sui dati;
•
supportare la personalizzazione dei servizi incentrata sull'utente, aumentando l'efficacia dell'erogazione dei servizi pubblici anche attraverso meccanismi di proattività.
L'Unione Europea mira a diventare leader strategico nell'impiego dell'intelligenza artificiale nel settore pubblico. Questa intenzione è chiaramente espressa nella Comunicazione “Piano Coordinato sull'Intelligenza Artificiale” COM (2021) 205 del 21 aprile 2021 in cui la Commissione europea propone specificamente di "rendere il settore pubblico un pioniere nell'uso dell'IA".
La revisione del Piano sull’intelligenza artificiale è stata accompagnata dalla “Proposta di Regolamento del Parlamento Europeo e del Consiglio che stabilisce regole armonizzate sull’intelligenza artificiale” (AI Act) COM (2021) 206 del 21 aprile 2021. La proposta di regolamento mira ad affrontare i rischi legati all’utilizzo dell'IA, classificandoli in quattro diversi livelli: rischio inaccettabile (divieto), rischio elevato, rischio limitato e rischio minimo. Inoltre, il regolamento intende porre le basi per costruire un ecosistema di eccellenza nell'IA e rafforzare la capacità dell'Unione Europea di competere a livello globale.
L’AI Act ha introdotto una importante sfida in materia di normazione tecnica. La Commissione Europea ha adottato il 25 maggio 2023 la Decisione C(2023)3215 - Standardisation request M/5932 con la quale ha affidato agli Enti di normazione europei CEN e CENELEC la redazione di norme tecniche europee a vantaggio dei sistemi di intelligenza artificiale in conformità con i principi dell’AI Act.
87
Il “Dispositivo per la ripresa e la resilienza” ha tra gli obiettivi quello di favorire la creazione di una industria dell’intelligenza artificiale nell’Unione Europea al fine di assumere un ruolo guida a livello globale nello sviluppo e nell'adozione di tecnologie di IA antropocentriche, affidabili, sicure e sostenibili. In Italia il PNRR prevede importanti misure di finanziamento sia per la ricerca in ambito di intelligenza artificiale sia per lo sviluppo di piattaforme di IA per i servizi della Pubblica Amministrazione.
Il DTD di concerto con ACN e AGID promuoverà l’obiettivo di innalzare i livelli di cybersecurity dell’Intelligenza Artificiale per assicurare che sia progettata, sviluppata e impiegata in maniera sicura, anche in coerenza con le linee guida internazionali sulla sicurezza dell’Intelligenza Artificiale. La cybersecurity è un requisito essenziale dell’IA e serve per garantire resilienza, privacy, correttezza ed affidabilità, ovvero un cyberspazio più sicuro.
La Pubblica Amministrazione italiana conta esperienze rilevanti nello sviluppo e utilizzo di soluzioni di intelligenza artificiale. A titolo esemplificativo si citano le esperienze di:
•
Agenzia delle entrate, utilizzo di algoritmi di machine learning per analizzare schemi e comportamenti sospetti, aiutando nella prevenzione e rilevazione di frodi;
•
INPS, adozione di chatbot per semplificare e personalizzare l'interazione con l’utente, migliorando l'accessibilità e l'usabilità dei servizi;
•
ISTAT, utilizzo di foundation models per generare ontologie a partire dalla descrizione in linguaggio naturale del contesto semantico al fine di migliorare la qualità della modellazione dei dati.
In questo contesto, l’affermarsi dei foundation models costituisce un importante fattore di accelerazione per lo sviluppo e l’adozione di soluzioni di intelligenza artificiale. Per foundation models si intendono sistemi di grandi dimensioni in grado di svolgere un'ampia gamma di compiti specifici, come la generazione di video, testi, immagini, la conversazione in linguaggio naturale, l'elaborazione o la generazione di codice informatico. L’AI Act definisce inoltre come foundation models “ad alto impatto” i modelli addestrati con una grande quantità di dati e con complessità, capacità e prestazioni elevate.
Principi generali per l’utilizzo dell’intelligenza artificiale nella Pubblica Amministrazione
Le amministrazioni pubbliche devono affrontare molte sfide nel perseguire l'utilizzo dell'intelligenza artificiale. Di seguito si riportano alcuni principi generali che dovranno essere adottati dalle pubbliche amministrazioni e declinati in fase di applicazione tenendo in considerazione lo scenario in veloce evoluzione.
1.
Miglioramento dei servizi e riduzione dei costi. Le pubbliche amministrazioni concentrano l’investimento in tecnologie di intelligenza artificiale nell’automazione dei compiti ripetitivi connessi ai servizi istituzionali obbligatori e al funzionamento dell'apparato amministrativo. Il conseguente recupero di risorse è destinato al miglioramento della qualità dei servizi anche mediante meccanismi di proattività.
2.
Analisi del rischio. Le amministrazioni pubbliche analizzano i rischi associati all'impiego di sistemi di intelligenza artificiale per assicurare che tali sistemi non provochino violazioni dei diritti fondamentali della persona o altri danni rilevanti. Le pubbliche amministrazioni adottano la classificazione dei sistemi di IA secondo le categorie di rischio definite dall’AI Act.
3.
Trasparenza, responsabilità e informazione. Le pubbliche amministrazioni pongono particolare attenzione alla trasparenza e alla interpretabilità dei modelli di intelligenza artificiale al fine di garantire la responsabilità e rendere conto delle decisioni adottate con il
88
supporto di tecnologie di intelligenza artificiale. Le amministrazioni pubbliche forniscono
informazioni adeguate agli utenti al fine di consentire loro di prendere decisioni informate riguardo all'utilizzo dei servizi che sfruttano l'intelligenza artificiale.
4.
Inclusività e accessibilità. Le pubbliche amministrazioni sono consapevoli delle responsabilità e delle implicazioni etiche associate all'uso delle tecnologie di intelligenza artificiale. Le pubbliche amministrazioni assicurano che le tecnologie utilizzate rispettino i principi di equità, trasparenza e non discriminazione.
5.
Privacy e sicurezza. Le pubbliche amministrazioni adottano elevati standard di sicurezza e protezione della privacy per garantire che i dati dei cittadini siano gestiti in modo sicuro e responsabile. In particolare, le amministrazioni garantiscono la conformità dei propri sistemi di IA con la normativa vigente in materia di protezione dei dati personali e di sicurezza cibernetica.
6.
Formazione e sviluppo delle competenze. Le pubbliche amministrazioni investono nella formazione e nello sviluppo delle competenze necessarie per gestire e applicare l’intelligenza artificiale in modo efficace nell’ambito dei servizi pubblici. A tale proposito si faccia riferimento agli obiettivi individuati nel Capitolo 1.
7.
Standardizzazione. Le pubbliche amministrazioni tengono in considerazione, durante le fasi di sviluppo o acquisizione di soluzioni basate sull'intelligenza artificiale, le attività di normazione tecnica in corso a livello internazionale e a livello europeo da CEN e CENELEC con particolare riferimento ai requisiti definiti dall’AI Act.
8.
Sostenibilità: Le pubbliche amministrazioni valutano attentamente gli impatti ambientali ed energetici legati all’adozione di tecnologie di intelligenza artificiale e adottando soluzioni sostenibili dal punto di vista ambientale.
9.
Foundation Models (Sistemi IA “ad alto impatto”). Le pubbliche amministrazioni, prima di adottare foundation models “ad alto impatto”, si assicurano che essi adottino adeguate misure di trasparenza che chiariscono l’attribuzione delle responsabilità e dei ruoli, in particolare dei fornitori e degli utenti del sistema di IA.
10.
Dati. Le pubbliche amministrazioni, che acquistano servizi di intelligenza artificiale tramite API, valutano con attenzione le modalità e le condizioni con le quali il fornitore del servizio gestisce di dati forniti dall’amministrazione con particolare riferimento alla proprietà dei dati e alla conformità con la normativa vigente in materia di protezione dei dati e privacy.
Dati per l’intelligenza artificiale
La disponibilità di dati di alta qualità e il rispetto dei valori e dei diritti europei, quali la protezione dei dati personali, la tutela dei consumatori e la normativa in materia di concorrenza sono i prerequisiti fondamentali nonché un presupposto per lo sviluppo e la diffusione dei sistemi di IA. La disponibilità di dati rappresenta peraltro un requisito chiave per l’adozione di un approccio all’intelligenza artificiale attento alle specificità nazionali.
La Strategia Europea per i dati è implementata dal punto normativo dagli atti sopra citati che costituiscono il quadro regolatorio entro il quale deve muoversi una Pubblica Amministrazione che intende operare con sistemi di IA sui dati aperti.
Riguardo l’utilizzo dei dati da parte di sistemi di intelligenza artificiale, l’AI Act richiede ai fornitori di sistemi di IA di adottare una governance dei dati e appropriate procedure di gestione dei dati (con particolare attenzione alla generazione e alla raccolta dei dati, alle operazioni di preparazione dei dati, alle scelte di progettazione e alle procedure per individuare e affrontare le distorsioni e le potenziali distorsioni per correlazione o qualsiasi altra carenza pertinente nei dati). L’AI Act pone particolare attenzione agli aspetti qualitativi dei set di dati utilizzati per addestrare, convalidare e testare i sistemi
89
di IA (tra cui rappresentatività, pertinenza, completezza e correttezza). La Commissione Europea ha avviato una specifica attività presso il CEN e il CENELEC per definire norme tecniche europee per rispondere a tali esigenze.
Nel contesto nazionale, tenuto conto di una architettura istituzionale che organizza i territori in regioni e comuni, che devono avere livelli di servizio omogenei, diventa cruciale progettare e implementare soluzioni nazionali basate sull'IA. Queste soluzioni devono essere in grado, da un lato, di superare eventuali disparità che caratterizzano le diverse amministrazioni territoriali e, dall'altro, di assicurare un pieno coordinamento tra territori differenti riguardo a servizi chiave per la società.
Riguardo l’affermarsi dei foundation models nel settore pubblico, una sfida fondamentale consiste nella creazione di dataset di elevata qualità, rappresentativi della realtà della Pubblica Amministrazione, con particolare riguardo al corpus normativo nazionale e comunitario, ai procedimenti amministrativi e alla struttura organizzativa della Pubblica Amministrazione italiana stessa.
Contesto normativo e strategico
Riferimenti normativi europei:
•
Comunicazione della Commissione al Parlamento Europeo e al Consiglio, “Piano Coordinato sull'Intelligenza Artificiale”, COM (2021) 205 del 21 aprile 2021
•
“Proposta di Regolamento del Parlamento Europeo e del Consiglio che stabilisce regole armonizzate sull’intelligenza artificiale” (AI Act), COM (2021) 206, del 21 aprile 2021
•
Decisione della Commissione “on a standardisation request to the European Committee for Standardisation and the European Committee for Electrotechnical Standardisation in support of Union policy on artificial intelligence” C (2023) 3215 del 22 maggio 2023

\underline{L'argomento estratto che parla della intelligenza artificiale è solo una curiosità ai fini del report che si vuol sviluppare}
	
Capitolo 6 - Infrastrutture
Infrastrutture digitali e Cloud
Scenario
La strategia “Cloud Italia”, pubblicata a settembre 2021 dal Dipartimento per la Trasformazione Digitale e dall’Agenzia per la Cybersicurezza Nazionale nell’ambito del percorso attuativo definito dall’art.33-septies del Decreto-Legge n.179 del 2012 e gli investimenti del PNRR legati all’abilitazione cloud rappresentano una grande occasione per supportare la riorganizzazione strutturale e gestionale delle pubbliche amministrazioni.
Non si tratta di una operazione unicamente tecnologica, le cui opportunità vanno esplorate a fondo da ogni ente.
La Strategia Cloud risponde a tre sfide principali: assicurare l’autonomia tecnologica del Paese, garantire il controllo sui dati e aumentare la resilienza dei servizi digitali. In coerenza con gli obiettivi del PNRR, la strategia traccia un percorso per accompagnare le PA italiane nella migrazione dei dati e degli applicativi informatici verso un ambiente cloud sicuro.
Con il principio cloud first, si vuole guidare e favorire l’adozione sicura, controllata e completa delle tecnologie cloud da parte del settore pubblico, in linea con i principi di tutela della privacy e con le raccomandazioni delle istituzioni europee e nazionali. In particolare, le pubbliche amministrazioni, in fase di definizione di un nuovo progetto, e/o di sviluppo di nuovi servizi, in via prioritaria devono valutare l’adozione del paradigma cloud prima di qualsiasi altra tecnologia.
Secondo tale principio, quindi, tutte le Amministrazioni sono obbligate ad effettuare una valutazione in merito all’adozione del cloud che rappresenta l’evoluzione tecnologica più dirompente degli ultimi anni e che sta trasformando radicalmente tutti i sistemi informativi della società a livello mondiale. Nel caso di eventuale esito negativo, tale valutazione dovrà essere motivata.
L’adozione del paradigma cloud rappresenta, infatti, la chiave della trasformazione digitale abilitando una vera e propria rivoluzione del modo di pensare i processi di erogazione dei servizi della PA verso cittadini, professionisti ed imprese.
L’attuazione dell’art.33-septies del Decreto-legge n. 179 del 2012, non rappresenta solo un adempimento legislativo, ma è soprattutto una occasione perché ogni ente attivi gli opportuni processi di gestione interna con il fine di modernizzare i propri applicativi e al contempo migliorare la fruizione dei procedimenti, delle procedure e dei servizi erogati.
È anche quindi una grande occasione per:
•
ridurre il debito tecnologico accumulato negli anni dalle amministrazioni;
•
mitigare il rischio di lock-in verso i fornitori di sviluppo e manutenzione applicativa;
•
ridurre significativamente i costi di manutenzione di centri elaborazione dati (data center) obsoleti e delle applicazioni legacy, valorizzando al contempo le infrastrutture digitali del Paese più all’avanguardia che stanno attuando il percorso di adeguamento rispetto ai requisiti del Regolamento AGID e relativi atti successivi dell’Agenzia per la Cybersicurezza Nazionale;
•
Incrementare la postura di sicurezza delle infrastrutture pubbliche per proteggerci dai rischi cyber.
94
In tal modo, le infrastrutture digitali saranno più affidabili e sicure e la Pubblica Amministrazione potrà rispondere in maniera organizzata agli attacchi informatici, garantendo continuità e qualità nella fruizione di dati e servizi.
Nell’ambito dell’attuazione normativa della Strategia Cloud Italia e dell’articolo 33-septies del Decreto-Legge n.179/2021 è stata realizzato il Polo Strategico Nazionale (PSN), l’infrastruttura promossa dal Dipartimento per la Trasformazione Digitale che, insieme alle altre infrastrutture digitali qualificate e sicure, consente di fornire alle amministrazioni tutte le soluzioni tecnologiche adeguate e gli strumenti per realizzare il percorso di migrazione.
Il Regolamento attuativo dell’articolo 33-septies del Decreto-Legge n.179/2021 ha fissato al 28 febbraio 2023 il termine per la trasmissione dei piani di migrazione da parte delle amministrazioni.
Dopo la presentazione dei Piani di migrazione, le amministrazioni devono gestire al meglio il trasferimento in cloud di dati, servizi e applicativi. Una fase da condurre e concludere entro il 30 giugno 2026, avendo cura dei riferimenti tecnici e normativi necessari per completare una migrazione di successo.
Per realizzare al meglio il proprio piano di migrazione, le amministrazioni possono far riferimento al sito cloud.italia.it dove sono disponibili diversi strumenti a supporto, tra cui:
•
il manuale di abilitazione al cloud, che da un punto di vista tecnico accompagna le PA nel percorso che parte dall’identificazione degli applicativi da migrare in cloud fino ad arrivare alla valutazione degli indicatori di risultato a migrazione avvenuta;
•
un framework di lavoro che descrive il modello organizzativo delle unità operative (unità di controllo, unità di esecuzione e centri di competenza) che eseguiranno il programma di abilitazione;
•
articoli tecnici di approfondimento relativi ai principali aspetti da tenere in considerazione durante una migrazione al cloud.
In particolare, mediante l'accesso agli strumenti sopra citati le amministrazioni possono trovare suggerimenti utili riguardo ai seguenti temi:
•
come riconoscere e gestire possibili situazioni di lock-in;
•
raccomandazioni sugli aspetti legati al back up dei dati e al disaster recovery;
•
consigli sulla scelta della migliore strategia di migrazione dal re-host al re-architect in base alle caratteristiche degli applicativi da migrare;
•
come migliorare la migrazione in cloud grazie a un approccio DevOps;
•
come definire e separare correttamente i ruoli tra Unità di Controllo (chi governa il progetto di migrazione) e Unità di esecuzione (chi realizza la migrazione);
•
come misurare costi/benefici derivanti dalla migrazione;
•
come stabilire un perimetro di responsabilità condivise tra amministrazione utente e fornitore di servizi cloud;
•
come sfruttare al massimo le opportunità del cloud grazie alle applicazioni cloud native, al re-architect e al re-purchase.
In caso di disponibilità all’interno del Catalogo dei servizi cloud per la PA qualificati da ACN di una soluzione SaaS che risponda alle esigenze delle amministrazioni, è opportuno valutare la migrazione verso il SaaS come soluzione prioritaria (principio SaaS-first) rispetto alle altre tipologie IaaS e PaaS.
95
Quindi, anche al fine di riqualificare la spesa della PA in sviluppo e manutenzione applicativa, le amministrazioni possono promuovere anche iniziative per la realizzazione di applicativi cloud native da erogare come SaaS mediante accordi verso altre amministrazioni anche attraverso il riuso di codice disponibile sul catalogo Developers Italia, nel rispetto della normativa applicabile.
Altro aspetto da curare è quello dei costi operativi correnti. Con la migrazione al cloud, ci sono grandi opportunità di risparmio economico, ma occorre strutturarsi per una corretta gestione dei costi cloud, sia da un punto di vista contrattuale che tecnologico.
Inoltre, con il crescere di servizi digitali forniti ad uno stesso ente da una molteplicità di fornitori diversi, anche via cloud, cresce notevolmente la complessità della gestione del parco applicativo, rendendo difficile la concreta integrazione tra i software dell’ente, l’effettiva possibilità di interoperabilità verso altri enti, la corretta gestione dei dati, ecc. Questo richiede all’Ufficio RTD, in forma singola o associata, l'evoluzione verso nuove architetture a “micro-servizi”.
Lo stesso concetto di “Sistema Pubblico di Connettività” (SPC), ancora presente nel CAD all’art.73, dovrà trovare una sua evoluzione basato sulla nuova logica cloud. Oggi è proprio il cloud computing, con la sua natura decentrata, policentrica e federata, a rendere possibile il disegno originario del SPC e salvaguardare pienamente l'autonomia degli enti, la neutralità tecnologica e la concorrenza sulle soluzioni ICT destinate alle PA.
Accanto agli aspetti di natura organizzativa è necessario porre attenzione anche ad una serie di elementi di natura più tecnologica.
Lo sviluppo delle infrastrutture digitali, infatti, è parte integrante della strategia di modernizzazione del settore pubblico: esse devono essere affidabili, sicure, energeticamente efficienti ed economicamente sostenibili e garantire l’erogazione di servizi essenziali per il Paese.
L’evoluzione tecnologica espone, tuttavia, i sistemi a nuovi e diversi rischi, anche con riguardo alla tutela dei dati personali. L’obiettivo di garantire una maggiore efficienza dei sistemi non può essere disgiunto dall’obiettivo di garantire contestualmente un elevato livello di sicurezza delle reti e dei sistemi informativi utilizzati dalla Pubblica Amministrazione.
Tuttavia, come già rilevato a suo tempo da AGID attraverso il Censimento del Patrimonio ICT della PA, molte infrastrutture della PA risultano prive dei requisiti di sicurezza e di affidabilità necessari e, inoltre, sono carenti sotto il profilo strutturale e organizzativo. Ciò espone il Paese a numerosi rischi, tra cui quello di interruzione o indisponibilità dei servizi e quello di attacchi cyber, con conseguente accesso illegittimo da parte di terzi a dati (o flussi di dati) particolarmente sensibili o perdita e alterazione degli stessi dati.
Lo scenario delineato pone l’esigenza immediata di attuare un percorso di razionalizzazione delle infrastrutture per garantire la sicurezza dei servizi oggi erogati tramite infrastrutture classificate come gruppo B, mediante la migrazione degli stessi verso infrastrutture conformi a standard di qualità, sicurezza, performance e scalabilità, portabilità e interoperabilità.
Con il presente documento, in coerenza con quanto stabilito dall’articolo 33-septies del decreto-legge 18 ottobre 2012, n. 179, si ribadisce che:
96
●
con riferimento alla classificazione dei data center di cui alla Circolare AGID 1/2019 e ai fini della strategia di razionalizzazione dei data center, le categorie “infrastrutture candidabili ad essere utilizzate da parte dei PSN” e “Gruppo A” sono rinominate “A”;
●
al fine di tutelare l'autonomia tecnologica del Paese, consolidare e mettere in sicurezza le infrastrutture digitali delle pubbliche amministrazioni di cui all'articolo 2, comma 2, lettere a) e c) del decreto legislativo 7 marzo 2005, n. 82, garantendo, al contempo, la qualità, la sicurezza, la scalabilità, l’efficienza energetica, la sostenibilità economica e la continuità operativa dei sistemi e dei servizi digitali, il Dipartimento per la Trasformazione Digitale della Presidenza del Consiglio dei Ministri promuove lo sviluppo di un’infrastruttura ad alta affidabilità localizzata sul territorio nazionale, anche detta Polo Strategico Nazionale (PSN), per la razionalizzazione e il consolidamento dei Centri per l'elaborazione delle informazioni (CED) destinata a tutte le pubbliche amministrazioni;
●
le amministrazioni centrali individuate ai sensi dell'articolo 1, comma 3, della legge 31 dicembre 2009, n. 196, nel rispetto dei principi di efficienza, efficacia ed economicità dell'azione amministrativa, migrano i loro Centri per l'elaborazione delle informazioni (CED) e i relativi sistemi informatici, privi dei requisiti fissati dalla Circolare AGID 1/2019 e, successivamente, dal regolamento di cui all’articolo 33-septies, comma 4, del decreto-legge 18 ottobre 2012, n. 179 (di seguito Regolamento cloud e infrastrutture), verso l’infrastruttura del PSN o verso altra infrastruttura propria già esistente e in possesso dei requisiti fissati dalla Circolare AGID 1/2019 e, successivamente, dal Regolamento cloud e infrastrutture. Le amministrazioni centrali, in alternativa, possono migrare i propri servizi verso soluzioni cloud qualificate, nel rispetto di quanto previsto dalle Circolari AGID n. 2 e n. 3 del 2018 e, successivamente, dal Regolamento cloud e infrastrutture;
●
le amministrazioni locali individuate ai sensi dell'articolo 1, comma 3, della legge 31 dicembre 2009, n. 196, nel rispetto dei principi di efficienza, efficacia ed economicità dell’azione amministrativa, migrano i loro Centri per l'elaborazione delle informazioni (CED) e i relativi sistemi informatici, privi dei requisiti fissati dalla Circolare AGID 1/2019 e, successivamente, dal regolamento cloud e infrastrutture, verso l'infrastruttura PSN o verso altra infrastruttura della PA già esistente in possesso dei requisiti fissati dallo stesso Regolamento cloud e infrastrutture. Le amministrazioni locali, in alternativa, possono migrare i propri servizi verso soluzioni cloud qualificate nel rispetto di quanto previsto dalle Circolari AGID n. 2 e n. 3 del 2018 e, successivamente, dal Regolamento cloud e infrastrutture;
●
le amministrazioni non possono investire nella costruzione di nuovi data center per ridurre la frammentazione delle risorse e la proliferazione incontrollata di infrastrutture con conseguente moltiplicazione dei costi. È ammesso il consolidamento dei data center nel rispetto di quanto previsto dall'articolo 33-septies del Decreto-legge 179/2012 e dal Regolamento di cui al comma 4 del citato articolo 33-septies.
Nel delineare il processo di razionalizzazione delle infrastrutture è necessario far riferimento anche a quanto previsto dalla “Strategia Cloud Italia”. In tal senso il documento prevede:
i) la creazione del PSN, la cui gestione e controllo di indirizzo siano autonomi da fornitori extra UE, destinato ad ospitare sul territorio nazionale principalmente dati e servizi strategici la cui compromissione può avere un impatto sulla sicurezza nazionale, in linea con quanto previsto
97
in materia di perimetro di sicurezza nazionale cibernetica dal Decreto-legge 21 settembre 2019, n. 105 e dal DPCM 81/2021;
ii) un percorso di qualificazione dei fornitori di cloud pubblico e dei loro servizi per garantire che le caratteristiche e i livelli di servizio dichiarati siano in linea con i requisiti necessari di sicurezza, affidabilità e rispetto delle normative rilevanti e iii) lo sviluppo di una metodologia di classificazione dei dati e dei servizi gestiti dalle pubbliche amministrazioni, per permettere una migrazione di questi verso la soluzione cloud più opportuna (PSN o adeguata tipologia di cloud qualificato).
Con riferimento al punto i) creazione del PSN, a dicembre 2022, in coerenza con la relativa milestone PNRR associata, è stata realizzata e testata l’infrastruttura PSN. Si ricorda che tale infrastruttura eroga servizi professionali di migrazione verso l’infrastruttura PSN, servizi di housing, hosting e cloud nelle tipologie IaaS, PaaS.
Per maggiori informazioni sui servizi offerti da PSN si rimanda alla convenzione pubblicata sul sito della Presidenza del Consiglio dei Ministri.
Nel 2023 sono stati pubblicati e conclusi tre avvisi per la migrazione verso il PSN a valere sulla misura 1.1 del PNRR che hanno visto l’adesione di oltre 300 tra amministrazioni centrali e aziende sanitarie locali e ospedaliere. Per quanto riguarda le ASL/AO, in particolare, è stata offerta l’opportunità di decidere la destinazione dei propri servizi tra PSN, Infrastrutture della PA adeguate e soluzioni cloud qualificate coerentemente con quanto disposto dall’articolo 33-septies del Decreto-legge 179/2012. 130 Aziende sanitarie hanno scelto di portare almeno parte dei propri servizi presso il PSN.
Con riferimento ai punti ii) qualificazione e iii) classificazione a dicembre 2021 sono stati pubblicati il Regolamento cloud e infrastrutture e a gennaio 2022 i relativi atti successivi. A febbraio e a luglio sono stati pubblicati ulteriori Decreti ACN ed è prevista la pubblicazione da parte di ACN di un nuovo Regolamento.
Con riferimento alla misura 1.2 del PNRR a marzo 2023 sono stati raccolti e ammessi a finanziamento più dei 12.464 piani di migrazione richiesti dal target è stato raggiunto e superato il target italiano previsto per settembre 2023 con la migrazione di oltre 1.100 enti locali che hanno migrato i loro servizi verso soluzioni cloud qualificate.
Con riferimento al tema del cloud federato, si premette che la definizione tecnica coerentemente con la ISO/IEC 22123-1:2023 è la seguente: "modello di erogazione di servizi cloud forniti da 2 o più cloud service provider che si uniscono mediante un accordo che preveda un insieme concordato di procedure, processi e regole comuni finalizzato all'erogazione di servizi cloud". Le amministrazioni con infrastrutture classificate "A" che hanno deciso di investire sui propri data center per valorizzare i propri asset ai fini della razionalizzazione dei centri elaborazione dati, adeguandoli secondo le modalità e i termini previsti ai requisiti di cui al Regolamento adottato ai sensi del comma 4 dell'articolo 33-septies del Decreto-legge 179/2012 e agli atti successivi di ACN, hanno la facoltà di valutare la possibilità di stringere accordi in tal senso per raggiungere maggiori livelli di affidabilità, sicurezza ed elasticità, purché siano rispettati i princìpi di efficacia ed efficienza dell'azione amministrativa e della normativa applicabile. Le amministrazioni che dovessero stipulare tali accordi
98
realizzerebbero così le infrastrutture cloud federate della PA che si affiancano all’infrastruttura Polo Strategico Nazionale nel rispetto dell’articolo 33-septies del decreto-legge 18 ottobre 2012, n. 179.
Per “infrastrutture di prossimità” (o edge computing) si intendono i nodi periferici (edge nodes), misurati come numero di nodi di calcolo con latenze inferiori a 20 millisecondi; si può trattare di un singolo server o di un altro insieme di risorse di calcolo connesse, operati nell'ambito di un'infrastruttura di edge computing, generalmente situati all'interno di un edge data center che opera all'estremità dell'infrastruttura, e quindi fisicamente più vicini agli utenti destinatari rispetto a un nodo cloud in un data center centralizzato".
Le amministrazioni che intendono realizzare e/o utilizzare infrastrutture di prossimità verificano la conformità di queste ai requisiti del Regolamento di cui al comma 4 dell’articolo 33-septies del DL 179/2012.
Punti di attenzione e azioni essenziali per tutti gli enti
1) L’attuazione dell’art.33-septies Decreto-legge 179/2012, e del principio cloud-first, deve essere tra gli obiettivi prioritari dell’ente. Occorre curare da subito anche gli aspetti di sostenibilità economico-finanziaria nel tempo dei servizi attivati, avendo cura di verificare gli impatti della migrazione sui propri capitoli di bilancio relativamente sia ai costi correnti (OPEX) sia agli investimenti in conto capitale (CAPEX).
2) La gestione dei servizi in cloud deve essere presidiata dall’ente in tutto il ciclo di vita degli stessi e quindi è necessaria la disponibilità di competenze specialistiche all’interno dell’Ufficio RTD, in forma singola o associata.
Approfondimento tecnologico per gli RTD
1) La piena abilitazione al cloud richiede l’evoluzione del parco applicativo software verso la logica as a service delle applicazioni esistenti, andando oltre il mero lift-and-shift dei server, progettando opportuni interventi di rearchitect, replatform o repurchase per poter sfruttare le possibilità offerte oggi dalle moderne piattaforme computazionali e dagli algoritmi di intelligenza artificiale. In tal senso, occorre muovere verso architetture a “micro-servizi” le cui caratteristiche sono, in sintesi, le seguenti:
•
ogni servizio non ha dipendenze esterne da altri servizi e gestisce autonomamente i propri dati (self-contained)
•
ogni servizio comunica con l'esterno attraverso API/webservice e senza dipendenza da stati pregressi (lightweight/stateless)
•
ogni servizio può essere implementato con differenti linguaggi e tecnologie, in modo indipendente dagli altri servizi (implementation-indipendent)
•
ogni servizio può essere dispiegato in modo automatico e gestito indipendentemente dagli altri servizi (indipendently deployable)
•
ogni servizio implementa un insieme di funzioni legate a procedimenti e attività amministrative, non ha solo scopo tecnologico (business-oriented):
2) È compito dell’Ufficio RTD curare sia gli aspetti di pianificazione della migrazione/abilitazione al cloud che l’allineamento dello stesso con l'implementazione delle relative opportunità di riorganizzazione dell’ente offerte dall’abilitazione al cloud e dalle nuove architetture a micro-servizi.
99
3) La gestione del ciclo di vita dei servizi in cloud dell’amministrazione richiede la strutturazione di opportuni presidi organizzativi e strumenti tecnologici per il cloud-cost-management, in forma singola o associata.
Contesto normativo e strategico
In materia di infrastrutture esistono una serie di riferimenti sia normativi che strategici a cui le amministrazioni devono attenersi. Di seguito un elenco delle principali fonti.
Riferimenti normativi nazionali:

Il sistema pubblico di connettività
Scenario
Il Sistema Pubblico di Connettività (SPC) garantisce alle Amministrazioni aderenti sia l’interscambio di informazioni in maniere riservata che la realizzazione della propria infrastruttura di comunicazione.
A tale Sistema possono interconnettersi anche le reti regionali costituendo così una rete di comunicazione nazionale dedicato per l’interscambio di informazioni tra le pubbliche amministrazioni sia centrali che locali.
Per effetto della legge n. 87 del 3 luglio 2023, di conversione del Decreto-legge 10 maggio 2023, n. 51 la scadenza dell’attuale Contratto Quadro è stata prorogata al 31 dicembre 2024; entro questa data sarà reso disponibile alle Amministrazioni interessate il nuovo Contratto Quadro che prevederà oltre ai servizi di connettività anche i servizi di telefonia fissa come da informativa Consip del 13 Aprile 2023.
Il Sistema Pubblico di Connettività fornisce un insieme di servizi di rete che:
•
permette alla singola Pubblica Amministrazione, centrale o locale, di interconnettere le proprie sedi e realizzare così anche l’infrastruttura interna di comunicazione;
•
realizza un’infrastruttura condivisa di interscambio consentendo l’interoperabilità tra tutte le reti delle pubbliche amministrazioni salvaguardando la sicurezza dei dati;
•
garantisce l’interconnessione della Pubblica Amministrazione alla rete Internet;


Capitolo 7 - Sicurezza informatica
Sicurezza informatica
Scenario
L’evoluzione delle moderne tecnologie e la conseguente possibilità di ottimizzare lo svolgimento dei procedimenti amministrativi con l’obiettivo di rendere efficace, efficiente e più economica l’azione amministrativa, ha reso sempre più necessaria la “migrazione” verso il digitale che, però, al contempo, sta portando alla luce nuovi rischi, esponendo imprese e servizi pubblici a possibili attacchi cyber. In quest’ottica, la sicurezza e la resilienza delle reti e dei sistemi, su cui tali tecnologie poggiano, sono il baluardo necessario a garantire, nell’immediato, la sicurezza del Paese e, in prospettiva, lo sviluppo e il benessere dello Stato e dei cittadini.
La recente riforma dell’architettura nazionale cyber, attuata attraverso l’adozione del decreto-legge 14 giugno 2021, n. 82 che ha istituito l’Agenzia per la Cybersicurezza Nazionale (ACN), ha come obiettivo, tra gli altri, quello di sviluppare e rafforzare le capacità cyber nazionali, garantendo l’unicità istituzionale di indirizzo e azione, anche mediante la redazione e l’implementazione della Strategia nazionale di cybersicurezza, che considera cruciale, per il corretto “funzionamento” del sistema Paese, la sicurezza dell’ecosistema digitale alla base dei servizi erogati dalla Pubblica Amministrazione, con specifica attenzione ai beni ICT. Tali beni supportano le funzioni e i servizi essenziali dello Stato e, purtroppo, come dimostrano gli ultimi rapporti di settore, sono tra i bersagli preferiti degli attacchi cyber.
Per garantire lo sviluppo e il rafforzamento delle capacità cyber nazionali, con il Piano Nazionale di Ripresa e Resilienza e con i Fondi per l’attuazione e la gestione della Strategia nazionale di cybersicurezza sono state destinate significative risorse alla sicurezza cibernetica e alle misure tese a realizzare un percorso di miglioramento della postura di sicurezza del sistema Paese nel suo insieme e, in particolare, della Pubblica Amministrazione.
Gli obiettivi e i risultati attesi, definiti successivamente nel presente capitolo, sono in linea con specifici interventi realizzati dall’ACN in favore delle pubbliche amministrazioni per cui sono state individuate specifiche aree di miglioramento. In particolare, il riferimento è alla necessità di:
•
prevedere dei modelli di gestione centralizzati della cybersicurezza, coerentemente con il ruolo trasversale associato (obiettivo 7.1 di questo Piano);
•
definire processi di gestione e mitigazione del rischio cyber, sia interni sia legati alla gestione delle terze parti di processi IT (obiettivi 7.2, 7.3, 7.4);
•
promuovere attività legate al miglioramento della cultura cyber delle Amministrazioni (obiettivo 7.5).
All’interno di questo contesto, AGID metterà a disposizione della Pubblica Amministrazione una serie di piattaforme e di servizi, che verranno erogati tramite il proprio CERT, finalizzati alla conoscenza e al contrasto dei rischi cyber legati al patrimonio ICT della PA (obiettivo 7.6)
	
\section{Appunti del documento: AGENAS Protocollo n. 2022/0006276 ingresso del 24/06/2022}

Estratto dal documento: 

4. PROGETTAZIONE E SVILUPPO SERVIZI ICT Include i servizi finalizzati allo sviluppo, modifica, evoluzione e personalizzazione di soluzioni innovative rispondenti alle esigenze dei clienti. Per l’erogazione del Servizio la Sogei adotta un Processo di produzione proprietario standardizzato, certificato secondo le norme ISO 9001:2015 e conforme con la normativa ISO
27001:2013, e ISO 25012:2014 in materia di controlli sulla sicurezza e qualità dei dati nonché alle regole introdotte dal GDPR. Tale processo è basato sui modelli metodologici di sviluppo Evolutivo/Incrementale, RUP – Rational Unified Process e Agile applicati in funzione dei contesti di sviluppo per ottimizzare i fattori produttivi e gestionali. La Sogei opera nell’ambito dell’Application Lifecycle Management (ALM) che identifica un approccio strategico alla gestione delle informazioni, dei processi e delle risorse a supporto del ciclo di vita delle applicazioni software. La Manutenzione migliorativa sarà considerata parte
integrante dell’effort di sviluppo e manutenzione evolutiva in quanto attività essenziale per il raggiungimento della qualità attesa nel software prodotto. 4.1 SVILUPPO E MANUTENZIONE EVOLUTIVA DEL SOFTWARE AD HOC
Sogei ha la responsabilità di governare gli obiettivi di sviluppo e manutenzione evolutiva attraverso le seguenti attività:  analisi dei requisiti; AGENAS Protocollo n. 2022/0006276 ingresso del 24/06/2022 Pagina 47 di 293
Allegato A Descrizione dei Servizi, Livelli di servizio e Corrispettivi – pag. 18 di 95  attuazione intervento;  avviamento;  verifica di conformità;  supporto sistemistico per lo sviluppo e la manutenzione evolutiva;  estensione. La verifica di conformità viene eseguita in un ambiente, predisposto da Sogei, equivalente a quello di esercizio. Per un periodo di 365 (trecentosessantacinque) giorni solari decorrenti dalla data di inizio estensione delle applicazioni software realizzate, la Sogei è impegnata a prestare, a propria cura e spese, la manutenzione correttiva delle applicazioni software. L’effort dello sviluppo di applicazioni viene misurato mediante nuove metodologie che tengono conto delle moderne modalità di sviluppo Agile e DevOps e su architetture maggiormente complesse. Il modello dei requisiti è composto da:  Requisiti funzionali;  Requisiti non funzionali;  Requisiti di progetto/ambito. Per ciascuna di queste componenti viene stimato l’effort che essa genera utilizzando, ove possibile, unità di misura standard.  Requisiti funzionali Misura Funzionale = metrica compatibili con lo standard ISO/IEC 14143-1:2007 (di seguito FSM) in generale non
AGENAS Protocollo n. 2022/0006276 ingresso del 24/06/2022 Pagina 48 di 293
Allegato A Descrizione dei Servizi, Livelli di servizio e Corrispettivi– pag. 19 di 95 necessariamente function point IFPUG oppure Simple Function Point oppure COSMIC etc.. Ad oggi si utilizza come metrica funzionale di riferimento IFPUG FP versione 4.3.  Requisiti non funzionali Misura Impatto Non Funzionale = % di impatto sulla produttività derivante da una misura non funzionale
compatibile con la ISO/IEC 25010, tenendo conto delle
sottocategorie definite dalla metrica stessa (in questa fase si prende a riferimento SNAP IFPUG v2.4).  Logiche di Operazione sui dati  Logiche di Validazione dei dati in ingresso,  Livello di Complessità delle Operazioni Logico e Matematiche,  Formattazione Dati (o livello) di Movimentazione dei dati interni all'Applicazione,  Livello di configurabilità dell’applicazione che aggiunge valore all'utente senza necessità di interventi software,  Disegno dell’interfaccia  Complessità delle Interfacce utente,  Disponibilità di Guide e Manuali dell’applicazione,  Metodi di inserimento dati tramite modalità aggiuntive, AGENAS Protocollo n. 2022/0006276 ingresso del 24/06/2022 Pagina 49 di 293
Allegato A Descrizione dei Servizi, Livelli di servizio e Corrispettivi – pag. 20 di 95  Metodi di presentazione dati tramite modalità aggiuntive,  Ambiente Tecnologico  Grado di utilizzo Piattaforme multiple aggiuntive,  Operazione ed interventi sul database,  Processi batch,  Architetture software  Grado di utilizzo di Componenti Software nell'applicazione,  Grado di estensione delle interfacce software per motivi tecnici. In funzione delle sottocategorie interessate, ad oggi, verrà determinato il numero di giornate aggiuntive per realizzare la componente non funzionale sulla base di uno schema di riferimento che potrà andarsi a perfezionare nel tempo. A seguire, a fronte della disponibilità di una banca dati delle misure, si potrà introdurre una metrica di valutazione più puntuale.  Requisiti di progetto/ambito Misura dell’impatto dell’ambito = % di impatto sulla
produttività derivante dalla complessità/instabilità
dell’ambito in termini di:  complessità della normativa di riferimento con conseguente difficoltà interpretative o alta probabilità di cambiamento,  forte instabilità dei requisiti, AGENAS Protocollo n. 2022/0006276 ingresso del 24/06/2022 Pagina 50 di 293
Allegato A Descrizione dei Servizi, Livelli di servizio e Corrispettivi– pag. 21 di 95  alta numerosità degli stakeholder,  nuovo dominio,  dipendenze da altri progetti esterni al di fuori del
controllo diretto,  dipendenze da altri progetti interni al di fuori del
controllo diretto. Sogei, per tutti gli sviluppi che si debbano integrare con un prodotto di mercato, garantisce il pieno e corretto funzionamento della soluzione nella sua interezza. Ove, in fase di realizzazione, emergano evidenze dell’impossibilità di garantire ciò, Sogei ne darà visibilità all’Amministrazione per condividere se e come proseguire.

4.2 PERSONALIZZAZIONE DEL SOFTWARE DI MERCATO
Il Servizio è finalizzato alla realizzazione di soluzioni basate su parametrizzazione e personalizzazione di pacchetti software acquistati sul mercato. La Sogei applica tale servizio in caso di:  personalizzazione/parametrizzazione di prodotti software di mercato (con particolare riferimento ai sistemi ERP - Enterprise Resource Planning);  realizzazione di interventi di Data Warehouse (DW) e
business intelligence (B.I.). In particolare, il servizio di personalizzazione del software di mercato consiste in:  sviluppo residuale di funzioni fortemente integrate con il prodotto nell’ambito del quale la personalizzazione viene effettuata che comporta la conoscenza del prodotto e
dell’eventuale linguaggio proprietario; AGENAS Protocollo n. 2022/0006276 ingresso del 24/06/2022 Pagina 53 di 293
Allegato A Descrizione dei Servizi, Livelli di servizio e Corrispettivi – pag. 24 di 95  interventi effettuati su prodotti con tecnologie/linguaggi non dimensionabile correttamente attraverso l’uso del FP (ad esempio BI). Gli sviluppi esterni al prodotto che ne consentono l’estensione in termini di funzionalità, sono invece
considerati sviluppi ad hoc e come tali dimensionati e remunerati. Tale servizio viene erogato attraverso un processo di produzione che:  parte da un’analisi comparativa, tra il prodotto base ed i requisiti dell’utente (gap-analisys), volta ad evidenziare quali requisiti non sia possibile soddisfare mediante l’attività di parametrizzazione e per i quali di conseguenza occorrerebbero degli interventi di personalizzazione;  si sviluppa in attività progressive di affinamento di un modello iniziale standard;  condivide con l’Amministrazione ogni attività di affinamento;  utilizza estensivamente un approccio prototipale. Per quanto riguarda gli interventi di Datawarehouse e BI il servizio prevede l’utilizzo di specifiche tecnologie quali i tool di modellazione dei dati, gli strumenti di gestione dei metadati (Repository), i tool di ETL, gli strumenti di visualizzazione oltre che la realizzazione di software dedicato. Indipendentemente dalle tecnologie adottate, il processo presenta caratteristiche omogenee relativamente all’articolazione in fasi (analisi dei requisiti, attuazione, AGENAS Protocollo n. 2022/0006276 ingresso del 24/06/2022 Pagina 54 di 293
Allegato A Descrizione dei Servizi, Livelli di servizio e Corrispettivi– pag. 25 di 95 avviamento, verifica di confromità) e alla documentazione prodotta a supporto. Sogei garantisce che le personalizzazioni si integrino correttamente con il prodotto di base e segnalerà anticipatamente all’Amministrazione se, l’intervento
richiesto, in fase di realizzazione facesse emergere
problematiche in tal senso. Per un periodo di 365 (trecentosessantacinque) giorni solari, decorrenti dalla data di inizio estensione delle applicazioni software, la Sogei è impegnata a prestare, a propria cura e spese, la manutenzione correttiva delle personalizzazioni effettuate.


5. SERVIZI DI BASE DI GESTIONE, CONDUZIONE E MANUTENZIONE
In tale ambito rientrano i servizi di gestione e conduzione infrastrutturale, i servizi di manutenzione adeguativa e correttiva delle applicazioni rilasciate e il servizio di Customer Care. 5.1 GESTIONE E CONDUZIONE SERVIZI ICT 5.1.1 Manutenzione servizi ICT Il servizio comprende le attività necessarie per garantire il corretto funzionamento del Sistema Informativo tramite la manutenzione del software in esercizio. Il servizio si applica sia alle applicazioni realizzate attraverso il servizio di “Sviluppo e Manutenzione Evolutiva di software ad hoc” sia a quanto realizzato attraverso il servizio di “Personalizzazione di prodotti di mercato”. Per queste ultime, il servizio riguarda solamente la componente software realizzata da Sogei come personalizzazione. Il servizio di manutenzione del software in esercizio, nel seguito denominato MAC (Manutenzione Adeguativa e Correttiva), comprende le seguenti tipologie di interventi:  Manutenzione correttiva: modifica reattiva di un prodotto software consegnato per correggere i problemi rilevati. La modifica sarà misurata sulla base di interventi software volti a rimuovere i malfunzionamenti (incident) segnalati dagli utenti o rilevati proattivamente da Sogei stessa; la remunerazione della manutenzione correttiva decorre
dalla data di termine del periodo di garanzia del software;  Manutenzione adeguativa: comprende le modifiche necessarie ad allineare il software ai cambiamenti dell’ambiente tecnologico. In particolare la manutenzione AGENAS Protocollo n. 2022/0006276 ingresso del 24/06/2022 Pagina 58 di 293
Allegato A Descrizione dei Servizi, Livelli di servizio e Corrispettivi– pag. 29 di 95 sarà calcolata sulla base di interventi sul software che adeguano ai mutamenti intervenuti nell’ambiente tecnologico di riferimento (sistema operativo, database, etc.). La remunerazione della manutenzione adeguativa decorre dalla data di avvenuta estensione del software.

5.1.2 Servizio di customer care Il servizio di Customer Care è finalizzato ad offrire supporto agli utenti per:  risposta ad esigenze informative;  soluzione di problematiche tecniche, anche collegate alle postazioni di lavoro del personale dell’Amministrazione;  soluzione di problematiche inerenti ai Servizi ICT erogati;  anticipare le esigenze dei clienti e prevenire le richieste di servizio;  fornire una Customer Service proattiva e completa ai fini di un servizio di assistenza più efficiente ed efficace, realizzando, laddove è possibile, sistemi autonomatici di risposta o invio di informazioni verso caselle di posta o numeri telefonici. Il Servizio di Customer Care è costituito da un insieme di servizi modulari sia di tipo infrastrutturale che organizzativo utilizzabili anche singolarmente. È possibile, per soddisfare esigenze strategiche di utenti definiti VIP e/o a fronte di Servizi ICT di particolare AGENAS Protocollo n. 2022/0006276 ingresso del 24/06/2022 Pagina 60 di 293
Allegato A Descrizione dei Servizi, Livelli di servizio e Corrispettivi– pag. 31 di 95 rilevanza, prevedere modalità di assistenza specifiche. 5.1.2.1 Customer Management Il Servizio comprende l’assistenza agli utenti interni ed esterni all’Amministrazione per la soluzione di problemi che l’utente stesso può incontrare nell’interazione con il Sistema Informativo. L'assistenza è erogata tramite una struttura di Customer Support con escalation alle strutture specialistiche Sogei dei diversi domini ICT, tecnici e applicativi dando origine ad interventi di Supporto Specialistico. Il sistema è stato strutturato per offrire una soluzione completa di “Customer Relationship Management” (CRM), ovvero un sistema per organizzare le informazioni di contatto, gestire le relazioni, tracciare le interazioni con i clienti al fine di efficientare la produttività. Il servizio è stato integrato con tecnologie di tipo “cognitivo”, basate sull’analisi e sulla comprensione del linguaggio naturale, in grado di auto apprendere dalla continua analisi dei dati e dall’interazione con altri sistemi o con il personale. Le richieste possono essere di diversa tipologia: Informative, Incidents (hardware, software, ecc.), Service Request e possono essere effettuate tramite:  canale telefonico;  canale web, con l’utilizzo dello strumento email o ticket di assistenza, con soluzioni self-service a partire dalle Faq;  canali social dedicati per fornire risposte rapide e
permettere ai clienti di entrare direttamente in contatto con il servizio di assistenza; AGENAS Protocollo n. 2022/0006276 ingresso del 24/06/2022 Pagina 61 di 293
Allegato A Descrizione dei Servizi, Livelli di servizio e Corrispettivi – pag. 32 di 95  “live caring” ovvero l’assistenza via chat per fornire risposte in tempi rapidi etc.. Dal punto di vista dell’offerta alcuni dei servizi descritti possono essere attivati a richiesta. Il servizio comprende la formazione del personale in risposta al Customer Support, il controllo e il reporting del servizio. L’erogazione del Customer Support può avvenire secondo diverse classi di servizio:  Base: lun-ven 8,00-18,00 e sab 8,00-14,00;  Ampliato: per “ora” aggiuntiva in funzione delle esigenze dell’Amministrazione fino ad una copertura h24 7 giorni su 7. La remunerazione è in termini di canone per la disponibilità del servizio e di corrispettivi unitari per le richieste di assistenza risolte dal Customer Support e dal Supporto Specialistico. Per l’ampliamento dell’orario del Customer Support è
prevista una remunerazione in termini di giornate secondo quanto previsto nell’ambito dei servizi Professional
dimensionati in accordo con l’Amministrazione. Si precisa che non verrà corrisposto nessun importo per le seguenti tipologie di richieste:  solleciti;  chiamate su problemi precedentemente chiusi, ma la cui soluzione non è risultata soddisfacente;  cadute linea;  chiamate dirette ad altro Call Center; AGENAS Protocollo n. 2022/0006276 ingresso del 24/06/2022 Pagina 62 di 293
Allegato A Descrizione dei Servizi, Livelli di servizio e Corrispettivi– pag. 33 di 95  chiamate “improprie”;  richieste "chiuse d'ufficio", comprese quelle di tipo specialistico gestite dalle strutture competenti Sogei (una richiesta si intende "chiusa d'ufficio" se è non risolta, non sospesa, pervenuta almeno 60 gg solari prima della rilevazione e senza solleciti nei 30 gg precedenti la rilevazione). 5.1.2.1.1

Disaster recovery Servizi ICT e dati Le caratteristiche del servizio sono le medesime già illustrate per la componente SERVER e STORAGE per le banche dati.
Continuità Operativa – Disaster recovery Servizi ICT Ad oggi non esiste l’infrastruttura di DR per gli Appliance-A; nell’ambito della Conduzione centrale Appliance-T ed
Appliance-Q si inquadra il servizio Disaster Recovery della componente di business che ha come fine quello di permettere agli utenti la ripresa delle attività produttive in caso di disastro che colpisca il sito primario che si trova a Roma. Il servizio ha come presupposto la disponibilità della copia dei dati effettuata con il servizio di Disaster Recovery delle banche dati (storage), che se non specificamente rifiutato da parte dell’Amministrazione, è attivato di norma, a garanzia della costante disponibilità del dato. La verifica della correttezza delle azioni di predisposizione viene effettuata ogni 6 mesi con la ripartenza sul sito di recovery delle applicazioni di backup. Con cadenza annuale viene effettuata una prova di ripartenza con la partecipazione di un selezionato numero di utenti finali. Continuità Operativa – Disaster recovery Storage Il servizio mette a disposizione una copia dei dati considerati vitali presso un sito alternativo a quello primario. Questa AGENAS Protocollo n. 2022/0006276 ingresso del 24/06/2022 Pagina 85 di 293
Allegato A Descrizione dei Servizi, Livelli di servizio e Corrispettivi – pag. 56 di 95 copia in caso di disastro è la base per poter ripristinare nel sito alternativo le attività dell’Amministrazione. Per disporre di questa copia sempre aggiornata, la copia stessa viene effettuata on-line in modalità asincrona con quella esistente nel sito primario. La verifica della correttezza della copia effettuata avviene ogni 6 mesi con l’esecuzione di specifiche procedure di ripartenza delle applicazioni su basi dati scelte a campione; per l’esecuzione di queste prove senza interrompere l’attività di copia on-line, si usa una seconda copia di dati eseguita “Point in Time”. Lo Storage Disaster Recovery riguarda la parte dati e banche dati memorizzata su disco in SAN e se non specificamente rifiutato da parte dell’Amministrazione, è attivato di norma, a garanzia della costante disponibilità del dato. L’orario del servizio è H24 5.2.3.1 Livelli di Servizio In caso in cui l’Amministrazione richieda un orario H24 senza possibilità di prevedere fermi concordati, ne verrà valutata la fattibilità e attivato un apposito servizio PLATINUM
remunerato sulla base delle risorse aggiuntive necessarie.

Allegato A Descrizione dei Servizi, Livelli di servizio e Corrispettivi – pag. 58 di 95 Il servizio offre l’accesso a una piattaforma di Big Data altamente performante, di alta affidabilità e elevate prestazioni. Le Piattaforme “Big Data” sono quelle deputate alla raccolta di dati estesa in termini di volume, velocità e varietà tali da richiedere strumenti non convenzionali per estrapolare, gestire e processare informazioni entro un tempo ragionevole. Il Servizio prevede l’utilizzo di un Cluster Hadoop,
infrastruttura distribuita open source sviluppata sotto l'egida della Apache Software Foundation per l'elaborazione di Big Data, ovvero un cluster di macchine su cui è installata una distribuzione Hadoop (Cloudera, Hortonworks). In tale servizio rientrano anche i servizi di streaming distribuito per operazioni di publish & subscribe (es. prodotto Apache Kafka). Tecnicamente la piattaforma si basa su una collezione di server fisici o virtuali in grado di operare in alta affidabilità e fornendo capacità computazionale in grado di scalare al crescere dei dati e delle operazioni da eseguire su di essi consentendo elaborazioni di tipo batch, puntuali e real time. Il servizio non prevede un DR né dei dati né del servizio ICT che li utilizza. Le piattaforme Big Data hanno insito nell’infrastruttura una salvaguardia del dato, per cui non necessitano (e la mole dei dati, lo renderebbe di fatto poco praticabile) di backup su nastro. A salvaguardia dei dati in caso di disastro nel sito principale si prevedrà, comunque un servizio di backup con replica AGENAS Protocollo n. 2022/0006276 ingresso del 24/06/2022 Pagina 88 di 293
Allegato A Descrizione dei Servizi, Livelli di servizio e Corrispettivi– pag. 59 di 95 remota su sito di DR dei dati, senza ripartenza del servizio ICT. In caso di richiesta esplicita, Sogei, potrà attivare un servizio di DR del servizio ICT anche in ambito Big Data, con i limiti e le possibilità delle tecnologie adottate. L’orario del servizio è H24. 5.2.4.1 Livelli

Il DR non è fornito di default, in alternativa può essere effettuato, su richiesta, un servizio di backup su sistemi VLT, con replica su sito di Recovery. 5.2.5 Piattaforma NOSQL Le Piattaforme “NOSQL”, in analogia a quelle BIG DATA
sono quelle deputate alla raccolta di dati che utilizzano una varietà di modelli di dati per accedere e gestirli. Questi tipi di Database sono ottimizzati specificatamente per applicazioni che necessitano di grandi volumi di dati, latenza bassa e modelli di dati flessibili, ottenuti snellendo alcuni dei criteri di coerenza tipici dei database relazionali. AGENAS Protocollo n. 2022/0006276 ingresso del 24/06/2022 Pagina 89 di 293
Allegato A Descrizione dei Servizi, Livelli di servizio e Corrispettivi – pag. 60 di 95 Le piattaforme per “NOSQL” sono costituite da un sistema di calcolo distribuito realizzato da una collezione di computer indipendenti che, dal punto di vista dell’utente, risultano essere un unico sistema coerente. Un esempio di database NOSQL è MongoDB, db che si allontana dalla struttura tradizionale basata su tabelle dei database relazionali in favore di documenti in stile JSON con schema. L’offerta prevede l’utilizzo di un Cluster ossia un cluster di macchine su cui è installata un DB di tipo NOSQL (es.
MongoDB, Couchbase etc). Nel driver ‘NOSQL’ rientrano anche le piattaforme che
erogano servizi in memory distribuiti che utilizzano gli stessi principi in termini di infrastruttura: Sistemi Distribuiti con enorme quantità di ram e capacità di calcolo, che di fatto rispondono alla stessa tipologia di architettura (normalmente massiva nelle quantità di dischi, utilizzo di server fisici/virtuali, dato replicato tra più nodi). Il servizio offre l’accesso a una piattaforma altamente performante, di alta affidabilità ed elevate prestazioni. Tecnicamente la piattaforma si basa su una collezione di server fisici o virtuali in grado di operare in alta affidabilità e fornendo capacità computazionale in grado di scalare al crescere dei dati e delle operazioni da eseguire su di essi consentendo elaborazioni di tipo batch, puntuali e real time. Il servizio, non prevede, considerando la tipologia e l’onerosità del servizio, un DR né dei dati né del servizio di business che li utilizza. AGENAS Protocollo n. 2022/0006276 ingresso del 24/06/2022 Pagina 90 di 293
Allegato A Descrizione dei Servizi, Livelli di servizio e Corrispettivi– pag. 61 di 95 La piattaforma ha infatti nell’infrastruttura una salvaguardia del dato, per cui non necessitano (e la mole dei dati, lo renderebbe di fatto poco praticabile) di backup su nastro. A salvaguardia dei dati in caso di disastro nel Sito Principale si prevedrà, comunque un servizio di backup con replica remota su sito di DR dei dati, senza ripartenza del servizio di business. In caso di richiesta esplicita, Sogei, potrà attivare un servizio di DR del servizio ICT anche in ambito “NOSQL”, con i
limiti e le possibilità delle tecnologie adottate. L’orario del servizio è H24.

Analogamente all’ambito hadoop, anche i sistemi NOSQL offrono una salvaguardia del dato attraverso la replica dello stesso su copie distribuite, di conseguenza, considerando anche le moli dei dati, il DR non è fornito di default. Su richiesta può essere realizzato e per la tipologia di sistema, il costo è il medesimo di quello di produzione. AGENAS Protocollo n. 2022/0006276 ingresso del 24/06/2022 Pagina 91 di 293
Allegato A Descrizione dei Servizi, Livelli di servizio e Corrispettivi – pag. 62 di 95 In alternativa può essere effettuato, su richiesta, un servizio di backup su sistemi VLT, con replica su sito di Recovery. Il costo per tale servizio è quello determinato in ambito Storage per il backup. 5.3 SERVIZI DI COLLABORATION E COMMUNICATION
I servizi di collaboration and communication services, descritti di seguito, forniscono una serie completa di strumenti finalizzati a garantire l’efficienza e l’operatività dell’utente e a favorire la comunicazione e lo scambio delle informazioni in un contesto affidabile e sicuro. 5.3.1 Servizi navigation Internet Il servizio comprende la gestione della navigazione sulla rete Internet con protocollo standard http e protocollo sicuro https utilizzando un proxy, cioè un sistema informatico che funge da intermediario per le richieste da parte dei client alla ricerca di risorse web disaccoppiando in tal modo l'accesso alle risorse dalla postazione dell’utente. Il servizio viene offerto secondo i seguenti profili di utenza:  Utente Base: tale utenza può accedere con regole di filtraggio comuni basate su white list secondo le indicazioni dell’Amministrazione (max 300 regole di
filtraggio).  Utente Esteso: tale utenza può accedere con regole dinamiche di filtraggio dei contenuti personalizzate per garantire che l'utilizzo di Internet sia conforme ad una politica di utilizzo definita dall’Amministrazione; viene inoltre effettuata una scansione in tempo reale dei
contenuti in entrata per garantirne la sicurezza contro virus e altro malware. AGENAS Protocollo n. 2022/0006276 ingresso del 24/06/2022 Pagina 92 di 293
Allegato A Descrizione dei Servizi, Livelli di servizio e Corrispettivi– pag. 63 di 95  Utente Avanzato: tale utenza funzionalmente è simile alla categoria precedente “Utente Esteso” ma è caratterizzata dall’utilizzo “aziendale” di soluzioni Cloud per servizi di tipo Intranet. Tale servizio deve essere dichiarato
dall’Amministrazione. Il servizio è attivabile solo se tutti gli utenti di una Amministrazione aderiscono al servizio.  Sicurezza PLUS: su richiesta è possibile attivare anche un servizio avanzato di sicurezza che prevede un controllo approfondito sulla navigazione degli utenti al fine di evidenziare online eventuale codice malevolo all’interno della navigazione web dell’utenza (sandbox, minacce
zero-day). Orario del servizio H24.

5.3.2 Virtual Private Network utente Con il termine Virtual Private Network Utente si intende la realizzazione di connessioni private tra una singola
postazione e un terminatore che consente l’accesso a risorse private tipicamente servizi erogati dall’infrastruttura PSN attraverso una rete pubblica Internet. AGENAS Protocollo n. 2022/0006276 ingresso del 24/06/2022 Pagina 93 di 293
Allegato A Descrizione dei Servizi, Livelli di servizio e Corrispettivi – pag. 64 di 95 La soluzione di Virtual Private Network utente adottata in Sogei è basata sullo standard TLS. Gli utenti abilitati accedono ad un portale dal quale sarà consentito l'accesso all'infrastruttura privata previa autenticazione a 2 fattori di cui uno con token OTP.
L’accesso alle risorse interne seguirà le politiche di sicurezza definite per l’ente o per il gruppo di utenti previa una fase di verifica di fattibilità tecnica per evitare che minacce software siano introdotte all’interno del perimetro aziendale. Il servizio prevede accesso con utente e password con Token acquisite dall’Amministrazione secondo indicazioni fornite da Sogei e registrate sui sistemi (qualora il Token venga erogato via SMS l’invio SMS è a carico dell’Amministrazione). Orario del servizio: H24

Descrizione dei Servizi, Livelli di servizio e Corrispettivi– pag. 65 di 95 La soluzione di Digital WorkSpace è in grado di distribuire in modo rapido e sicuro applicazioni e desktop agli utenti. La piattaforma consente di erogare un ambiente di lavoro di ufficio privo di qualsiasi tipo di vincolo fisico e, di conseguenza, promuovere iniziative di Telelavoro e di Smart Working. La piattaforma di Digital WorkSpace offre una reale alternativa al Desktop Computing ed eroga le seguenti quattro tipologie di servizi per ognuna delle quali sono previsti due pacchetti di funzionalità: “Standard”, ovvero le funzionalità minime del singolo servizio, “Plus” ovvero funzionalità aggiuntive a quanto previsto nel
pacchetto “Standard”:  Virtual Desktop pooled: servizio di desktop virtuali Funzionalità Standard:  Distribuzione Software,  Patching,  Compliance,  Antivirus. Funzionalità Plus:  Quota disco Dati utente aggiuntiva (fornita con servizio storage),  Accesso da rete pubblica,  Reporting accesso utenti,  Configurazioni personalizzate.  Bundle di vApp: servizio di Virtual APP in bundle che prevede la configurazione di alcuni server (Session Host) per la distribuzione delle applicazioni sia ai desktop AGENAS Protocollo n. 2022/0006276 ingresso del 24/06/2022 Pagina 95 di 293
Allegato A Descrizione dei Servizi, Livelli di servizio e Corrispettivi – pag. 66 di 95 tradizionali che virtuali. L’utente può utilizzare le vAPP attraverso una sessione remota sui Session Host caratterizzata da un “look and feel” identico all’applicativo erogato. Il Bundle prevede l’erogazione di un massimo di n° 10 vAPP ed uno spazio dati per il profilo utente pari a 25 GB. Le funzionalità aggiuntive previste nel pacchetto “Plus” possono essere selezionate singolarmente.
Funzionalità Standard:  Streaming delle vAPP. Funzionalità Plus:  Reporting accesso utenti,  Accesso da rete pubblica,  Quota disco dati aggiuntiva (fornita con servizio storage).  Singola vApp: servizio per una singola Virtual APP; il servizio è identico in termini di funzionalità al servizio Bundle vAPP e prevede l’erogazione di un massimo di n° 1 vAPP ed uno spazio dati per il profilo utenti pari a 500 MB. Le funzionalità aggiuntive previste nel pacchetto
“Plus” possono essere selezionate singolarmente.
Funzionalità Standard:  Streaming della vAPP. Funzionalità Plus:  Reporting accesso utenti,  Accesso da rete pubblica,  vApp personalizzate, AGENAS Protocollo n. 2022/0006276 ingresso del 24/06/2022 Pagina 96 di 293
Allegato A Descrizione dei Servizi, Livelli di servizio e Corrispettivi– pag. 67 di 95  Quota disco dati aggiuntiva (fornita con servizio storage).  Singola vAPP “capped”: servizio del tutto analogo a quello non ‘bloccato’ con le seguenti specificazioni:
Funzionalità Standard: È stato calcolato con una concorrenza di utenze molto bassa e pari al 40\% al fine di coprire esigenze specifiche di rotazione del personale, piuttosto che di Smartworking. Essendo l’AC bloccata da questo basso livello di concorrenza, decisamente inferiore a quello osservato usualmente nell’uso del servizio, per garantire i livelli di servizio del servizio stesso e la qualità dello stesso, raggiunta la quota del 40\% degli utenti definiti (il driver è sempre computato sul numero di utenti definiti), l’utente successivo sarà respinto dall’infrastruttura; Essendo poi un driver ad hoc, considerando la pesante incidenza sull’infrastruttura data dalla concorrenza, ogni anno, in sede di Piano Operativo dovrà essere definito il massimo numero di utenti aderenti a questo driver. Eventuali necessità ulteriori per driver diversi, saranno aggiuntive rispetto al medesimo, ossia non si potranno azzerare/diminuire gli utenti qui definiti per sostituirli con utenti di tipologia diversa, se non previa verifica con Sogei. Orario di Servizio: H24

.4 SERVIZI DI TIPOLOGIA CLOUD
5.4.1 Piattaforma IaaS Sogei mette a disposizione una Piattaforma Cloud per
l’offerta di un servizio di Infrastructure as a Service (IaaS) On-Demand in cui ogni utente abilitato al servizio può crearsi in autonomia i propri Cloud Server Virtuali, di seguito CSV, scegliendoli da un catalogo. La Piattaforma interagisce con una infrastruttura virtuale sfruttandone tutte le potenzialità per la gestione dei CSV, dello Storage e del Networking. AGENAS Protocollo n. 2022/0006276 ingresso del 24/06/2022 Pagina 98 di 293
Allegato A Descrizione dei Servizi, Livelli di servizio e Corrispettivi– pag. 69 di 95 Dal punto di vista logico, la soluzione è rappresentata dal seguente schema: In linea generale:  gli utenti, compatibilmente con il ruolo loro assegnato, hanno accesso al blocco logico superiore (“Consumption and Governance”), potendo quindi inserire richieste di servizi esposti su un catalogo, operare in modalità self-
service sulle istanze CSV, accedere a dashboard che riportano informazioni sulla configurazione delle proprie istanze CSV e sulle risorse consumate;  il team IT di Sogei ha invece accesso sia al blocco logico superiore che a quelli sottostanti per predisporre, gestire e manutenere il servizio di Cloud Computing di tipo IaaS. La Piattaforma Cloud Sogei include tutti gli elementi per abilitare un Cloud Environment:  portale di Self Service e Service Catalog Cloud - fornisce agli utenti del Tenant la possibilità di richiedere il deploy di servizi e di tracciarne il ciclo di vita; AGENAS Protocollo n. 2022/0006276 ingresso del 24/06/2022 Pagina 99 di 293
Allegato A Descrizione dei Servizi, Livelli di servizio e Corrispettivi – pag. 70 di 95  service Delivery Automation - automatizza il provisioning dei servizi richiesti. La disponibilità delle istanze CSV pubblicate (Template pre-configurati da Sogei) può variare in base a nuove esigenze o secondo le caratteristiche dell’Hypervisor. Il service catalog cloud, da cui l’utente può configurare il proprio sistema, offre la possibilità di richiedere dei CSV preconfigurati con i seguenti sistemi operativi:  Microsoft Windows Server;  Red Hat Enterprise Linux;  Ubuntu Linux. La configurazione standard del CSV prevede:  una quota disco da 200GB, di cui 100GB preallocati sul
CSV esposto nel catalogo dei servizi ed altri 100GB da richiedere in fase di prima allocazione o in alternativa come action di post-deploy del CSV;  il backup dell’immagine virtuale di sistema in configurazione standard;  una replica locale dell’immagine virtuale (Alta Affidabilità Estesa);  una replica dell’immagine virtuale sul sito di Disaster. In fase di richiesta l’utente, oltre alla configurazione del CSV in termini di vCPU e vRam, può specificare il network di destinazione ed eventuali dischi aggiuntivi. Completata la creazione del CSV, l’utente potrà richiederne la modifica della sua configurazione in termini di risorse computazionali e spazio disco. AGENAS Protocollo n. 2022/0006276 ingresso del 24/06/2022 Pagina 100 di 293
Allegato A Descrizione dei Servizi, Livelli di servizio e Corrispettivi– pag. 71 di 95 È possibile richiedere uno spazio disco aggiuntivo rispetto alle quote previste contrattualmente e/o previste a catalogo, richiedibile sia in fase di prima allocazione per le varie tipologie di CSV esposte sia in fase successiva come
modifica al CSV. Tramite opzioni aggiuntive è prevista, inoltre, la possibilità di richiedere il Servizio di Backup e selezionarne la relativa tipologia. Il servizio di backup è opzionabile su più Tipologie, ognuna delle quali presenta diversi parametri di frequenza, retention e numero di restore possibili. Per default il singolo CSV è associato al livello Bronze (backup CSV) e a nessun tipo di backup per quanto riguarda il Backup Autonomo. A richiesta possono essere effettuati dei WAPT su:  Web Application  Web Service dell’Amministrazione, anche ai fini dell’esposizione del servizio su internet. Il servizio di Web Application Penetration Test consente all’Amministrazione di eseguire un’analisi di sicurezza delle applicazioni Web individuando le vulnerabilità
presenti ed esaminando l’esposizione al rischio di attacchi informatici nei confronti dei servizi e dell’infrastruttura. L’attività è eseguita da personale specializzato secondo le due principali metodologie internazionali in materia di test di sicurezza informatica: OWASP e OSSTMM. Le aree di analisi esaminate nel corso del test comprendono almeno le seguenti: AGENAS Protocollo n. 2022/0006276 ingresso del 24/06/2022 Pagina 101 di 293
Allegato A Descrizione dei Servizi, Livelli di servizio e Corrispettivi – pag. 72 di 95  Logica dell'applicazione: identificazione delle problematiche legate alla logica dell’applicazione;
 Autenticazione: analisi delle procedure di autenticazione per verificarne la robustezza;  Autorizzazione: verifica della gestione dei ruoli delle utenze applicative e identificazione di eventuali problematiche che possano portare ad una privilege escalation oppure all’accesso a dati e/o risorse di
competenza di altri utenti;  Sicurezza della comunicazione: analisi del canale di
comunicazione utilizzato;  Algoritmi di cifratura: verifica della robustezza degli algoritmi di cifratura utilizzati;  Validazione dei dati: analisi di tutti i campi di input presenti mediante diversi tipologie di payload allo scopo di verificarne la sicurezza e scongiurare l’eventuale presenza di vulnerabilità di tipo injection;  Configurazione e deployment: analisi della configurazione del server, delle eventuali risorse di default presenti e degli header http;  Gestione Errori: verifica della corretta gestione degli errori con eventuale segnalazione;  Privacy: analisi del flusso applicativo al fine di individuare eventuali esposizioni di dati identificativi, personali o sensibili dell’utente;  Gestione della sessione;  Dipendenze di terze parti: analisi delle componenti
software utilizzate dall’applicazione individuate nel corso AGENAS Protocollo n. 2022/0006276 ingresso del 24/06/2022 Pagina 102 di 293
Allegato A Descrizione dei Servizi, Livelli di servizio e Corrispettivi– pag. 73 di 95 del test e segnalazione della presenza di eventuali
vulnerabilità note. L’attività di test prevede:  Identificazione delle vulnerabilità per via manuale o con il supporto di strumenti automatici;  Raccolta delle evidenze necessarie per consentire a terzi di ripercorrere i passaggi necessari alla sua individuazione;  Analisi delle vulnerabilità e assegnazione di un livello di severità (ALTA, MEDIA, BASSA, INFORMATIVA) sulla base di metriche standard internazionali quali il CVSSv2;  Indirizzamento delle attività per il rientro delle vulnerabilità identificate indicando per ciascuna di esse le best practice da adottare;  Produzione di reportistica di dettaglio (technical report) sulle analisi eseguite con le evidenze raccolte per ciascuna delle vulnerabilità identificate. Il servizio viene erogato ai clienti su richiesta previa accettazione di un incarico ufficiale denominato Rules of Engagement (RoE) mediante il quale saranno fornite tutte le informazioni necessarie allo svolgimento dell’attività. In fase di avvio, previa raggiungibilità degli ambienti target, sarà fornita una stima dell’effort necessario sulla base di una valutazione dimensionale. Il servizio prevede, infatti, tre diversi profili di erogazione, a seconda della tipologia di applicazione sottoposta al test, come specificato nella seguente tabella AGENAS Protocollo n. 2022/0006276 ingresso del 24/06/2022 Pagina 103 di 293
Allegato A Descrizione dei Servizi, Livelli di servizio e Corrispettivi – pag. 74 di 95

Sono inoltre inclusi i seguenti servizi:  Security Operation Center Centrale: il servizio sarà assicurato a protezione dell’infrastruttura cloud nei termini e nei modi già descritti nell’ambito SERVER.  Continuità Operativa: Disaster recovery Servizi ICT: il servizio sarà assicurato a protezione dell’infrastruttura cloud nei termini e nei modi, già descritti nell’ambito SERVER ma senza assicurare la ripartenza dei servizi,
dipendente dalle configurazioni delle VM e dei software in esse installati (responsabilità dell’Amministrazione). Sogei assicurerà la ripartenza delle singole VM nella configurazione del sito principale. Non assicurerà eventuali configurazioni, comprese quelle di rete, spesso necessarie per garantire la continuità operativa del
business. Sulla base degli indicatori di seguito descritti (Disponibilità e Accessibilità), saranno messi a disposizione i seguenti strumenti di reporting e visualizzazione:  un report inviato via e-mail, contenente l’andamento dei due indicatori nelle ultime 24/48 ore, a partire dalla mezzanotte passata; AGENAS Protocollo n. 2022/0006276 ingresso del 24/06/2022 Pagina 104 di 293
Allegato A Descrizione dei Servizi, Livelli di servizio e Corrispettivi– pag. 75 di 95  un report a richiesta che fornisce l’andamento dei due indicatori dal giorno prima fino all’ora precedente.
Disponibilità dell’Infrastruttura Cloud: viene controllato lo stato di salute delle Virtual Machine in termini di vCpu, memoria, connettività e file system, ecc. I dati di
monitoraggio a cadenza regolare (ogni 5 minuti) vengono trasferiti sul Datawarehouse della piattaforma centrale Sogei. Accessibilità alla Infrastruttura Cloud: il controllo viene effettuato mediante accesso ogni 5 minuti ad una pagina https esposta su internet dalle Virtual Machine di servizio per verificare il corretto funzionamento di tutte le
componenti dell’infrastruttura Cloud: internet, sicurezza perimetrale, connettività, piattaforma cloud. Modalità di reporting ed accesso alle informazioni di controllo.
Orario di Servizio: H24

SERVIZI DI BASE DIVERSI DA QUELLI DI CONDUZIONE

Descrizione dei Servizi, Livelli di servizio e Corrispettivi– pag. 77 di 95 6. PIATTAFORME ASP Si tratta di servizi ICT offerti da Sogei ed erogati su proprie Piattaforme dotate di strumenti necessari alla loro erogazione, come servizi per la gestione documentale, soluzioni e-
learning e servizi di sicurezza, offerti in modalità ASP (Application Service Provider). Tali servizi, erogati attraverso infrastrutture remote condivise da più clienti e ubicate presso il Data Center Sogei, sono proposti attraverso un modello di pricing a canone basato sugli effettivi consumi. 6.1 SERVIZI DI SICUREZZA
6.1.1 Piattaforme di sicurezza IAM (Identity Access Management) Il servizio prevede l’identificazione, l’autenticazione e l’autorizzazione degli utenti interni ed esterni mediante sistemi di directory, di controllo delle autorizzazioni, PKI (Public Key Infrastructure) e sistemi per la gestione integrata delle componenti al fine di consentire all’Amministrazione una gestione autonoma delle autorizzazioni ai propri servizi. Il Disaster Recovery è incluso in modalità “full” per l’ambito CAST e Certification Authority e in modalità “base” (sole funzioni in lettura) per l’ambito CAU. Orario di Servizio: H24 AGENAS Protocollo n. 2022/0006276 ingresso del 24/06/2022 Pagina 107 di

Il SOC Sogei ha una struttura logica centralizzata con i seguenti obiettivi:  raccogliere e analizzare gli eventi di sicurezza rilevati dalle varie componenti infrastrutturali, le altre tecnologie (hardware e software) di sicurezza presenti anche negli asset periferici dell’Amministrazione che ha sottoscritto il servizio;  gestire e monitorare tutti i Sistemi di Sicurezza, al fine di identificare, contenere e bloccare eventuali “Minacce Informatiche” rilevate;  contribuire alla gestione e all’analisi degli incidenti informatici in collaborazione con la struttura del CERT Sogei ed i riferimenti dell’Amministrazione;  supportare le attività di Governo della Sicurezza verso l’Amministrazione. Il servizio SOC è svolto principalmente attraverso la
AGENAS Protocollo n. 2022/0006276 ingresso del 24/06/2022 Pagina 108 di 293
Allegato A Descrizione dei Servizi, Livelli di servizio e Corrispettivi– pag. 79 di 95 centralizzazione di tutti i log di evento provenienti dai dispositivi infrastrutturali di sicurezza. I suddetti log sono analizzati e correlati in tempo reale, attraverso tecnologica di tipo SIEM (Security Incident and Event Management), per dare evidenza degli eventi più significativi ai fini della sicurezza e funzionali alla gestione degli incidenti informatici. I servizi del SOC sono differenziati tra servizi endpoint e dispositivi. Per dispositivi si intendono apparati di sicurezza fisici o virtuali gestiti. Il servizio include le seguenti attività:  Monitoraggio delle sorgenti e notifica di vulnerabilità e minacce, con suggerimenti per la loro mitigazione;  Definizione, validazione e verifica delle configurazioni tecniche dei sistemi di sicurezza, aggiornamento della documentazione inerente l’infrastruttura di sicurezza IT;  Configurazione degli apparati di sicurezza in base alle politiche stabilite e attività di manutenzione sugli stessi, compreso il Patch Management;  Definizione, autorizzazione, implementazione, modifica (change) ed “enforcement”, delle regole di sicurezza da implementare; valutazione del rischio associato a tali attività, laddove acquisiti i servizi corrispondenti;  Monitoraggio in tempo reale dei sistemi di sicurezza (Real Time Security Monitoring), al fine di rilevare eventuali incidenti, attività sospette e violazioni delle politiche di sicurezza, anomalie e/o disservizi sull’infrastruttura di sicurezza; AGENAS Protocollo n. 2022/0006276 ingresso del 24/06/2022 Pagina 109 di 293
Allegato A Descrizione dei Servizi, Livelli di servizio e Corrispettivi – pag. 80 di 95  Validazione dei sistemi attraverso scansioni periodiche (o su richiesta), per rilevare ed analizzare vulnerabilità applicative e di sistema operativo con notifica alle
strutture responsabili del Patch Management, evidenziando eventuali attività di adeguamento non effettuate/posticipate (scansioni successive) laddove acquisiti i servizi corrispondenti;  Produzione di reportistica di livello tecnico ed executive;  Gestione degli eventi significativi rilevati (Incident Response, Incident Containment e Incident Recovery) coordinata anche attraverso la struttura del CERT Sogei con l’Amministrazione, di risposta e contenimento degli impatti derivanti dagli incidenti identificati e successivo ripristino delle normali attività operative.  Identificazione di possibili incidenti (Incident Identification, Classification e Notification) tramite
l’analisi delle informazioni raccolte e correlate dai sistemi del SOC, loro classificazione e notifica anche attraverso la struttura del CERT Sogei nel caso di incidenti con
caratteristiche significative e rilevanti;  Attività di supporto specialistico (Troubleshooting,
Architectural, LabTest) per l’analisi delle anomalie
applicative sui segmenti di rete dove insistono dispositivi di sicurezza perimetrale. Orario di Servizio: H24 AGENAS Protocollo n. 2022/0006276 ingresso del 24/06/2022 Pagina 110 di

SERVIZIO SITO/PORTALE WEB E COMPONENTI ACCESSORI
Il Servizio si riferisce alle attività di sviluppo e conduzione (manutenzione) di siti/portali Internet o Intranet per la condivisione di informazioni e/o servizi corredati delle necessarie personalizzazioni e degli eventuali “servizi accessori” modulari laddove richiesti. In particolare: AGENAS Protocollo n. 2022/0006276 ingresso del 24/06/2022 Pagina 111 di 293
Allegato A Descrizione dei Servizi, Livelli di servizio e Corrispettivi – pag. 82 di 95  Sviluppo sito: si tratta di uno sviluppo a corpo che prevede l’attività di analisi, progettazione, sviluppo, test e messa in esercizio di un nuovo sito/portale composto dalla home page e dai template delle pagine standard (es. pagina news, pagina di approfondimento etc.), realizzato su una delle piattaforme standard Sogei, la cui complessità è definita Extra, Alta, Media e Bassa in relazione alla
quantificazione dei parametri di seguiti definiti. È
ricompresa nel primo anno l’attività di manutenzione
applicativa.  Conduzione sito: si tratta di un servizio a canone che comprende la gestione dell’infrastruttura, le attività di manutenzione ordinaria del sito/portale, la gestione delle versioni di software di base, i piccoli interventi di manutenzione evolutiva necessari a garantire l’attualità della struttura e dei contenuti. Sono inoltre ricomprese le attività di web analytics. I Servizi accessori sono:  Redazione ordinaria: si tratta di un servizio a canone che comprende l’attività di pubblicazione, modifica e movimentazione di contenuti informativi e file, compreso l’aggiornamento delle pagine del sito, effettuata su precisa indicazione del committente sia relativamente al contenuto informativo che al posizionamento dello stesso all’interno del sito. Le attività sono eseguite secondo le modalità operative concordate. Questo servizio viene
erogato dal lunedì al venerdì, escluso festivi, in orario 8-
18.  Migrazione da altro sito: si tratta di un servizio a corpo che prevede l’attività di trasferimento di contenuti
AGENAS Protocollo n. 2022/0006276 ingresso del 24/06/2022 Pagina 112 di 293
Allegato A Descrizione dei Servizi, Livelli di servizio e Corrispettivi– pag. 83 di 95 editoriali e relativi allegati, dalla piattaforma sorgente alla piattaforma target individuata da Sogei. In dipendenza della specificità della piattaforma sorgente, è facoltà di Sogei individuare meccanismi automatici di trasferimento ovvero procedere all’inserimento manuale dei contenuti.  Ulteriori servizi opzionali come di seguito dettagliato:  Pagine con servizi applicativi: si tratta di uno sviluppo a corpo che comprende l’attività di implementazione di pagine web per l’inserimento di dati, successiva chiamata a un servizio esterno e
presentazione dei dati di risposta opportunamente formattati (per esempio: form di accesso alla rubrica, pagina di ricerca, survey, form di prenotazione appuntamenti, ecc.).  Canali Social: si tratta di un servizio a canone che
comprende l’attività relativa alla gestione (apertura, configurazione e monitoraggio) dei canali social richiesti dall’organizzazione anche attraverso piattaforme di social media management. Nell’attività di monitoraggio è inclusa la fase di assessment iniziale, l’individuazione dei keyword di ricerca, l’individuazione dei topic di analisi, pubblicazione
di notizie (post), la moderazione dei commenti, la sentiment analysis e la produzione di reportistica ad hoc. Questo servizio viene erogato dal lunedì al venerdì, escluso festivi, in orario 8-18.  Redazione speciale: si tratta di un servizio a canone
AGENAS Protocollo n. 2022/0006276 ingresso del 24/06/2022 Pagina 113 di 293
Allegato A Descrizione dei Servizi, Livelli di servizio e Corrispettivi – pag. 84 di 95 che comprende l’attività di redazione come definita
precedentemente ed erogata, su specifica richiesta del committente, dal lunedì al venerdì, in orario 18-
22 e nei giorni festivi in orario 8-14.  Sezione “area trasparenza” richiesta da ANAC: si tratta di uno sviluppo a corpo che comprende l’attività di predisposizione e aggiornamento periodico della sezione “trasparenza” in conformità
con quanto stabilito dall’Autorità Nazionale AntiCorruzione. E’ inclusa l’attività di verifica periodica di rispondenza e conformità della pagina ai requisiti dell’ANAC. La complessità di un sito è definita Extra, Alta, Media e Bassa in relazione alla quantificazione dei seguenti parametri:  Numero di pagine;  Platea di utenti visitatori;  N. visite medie;  N. linee redazionali.

(*) la classe di complessità “Extra” è prevista anche per quei siti/portali per i quali è stato necessario predisporre un’infrastruttura ad hoc per la particolare criticità del servizio collegato. (**) la singola linea redazionale rappresenta il gruppo di redattori che opera continuativamente su una determinata sezione del sito ovvero rappresenta l’attività eseguita dai gruppi redazionali distribuiti sul territorio (redazioni periferiche per Amministrazioni con sedi delocalizzate). I servizi accessori opzionali presentano invece un costo puntuale mediato. Orario di Servizio: H24 7.1 LIVELLI DI SERVIZIO
Allo sviluppo dei siti si applicano i Livelli di servizio del servizio “Personalizzazione del software di mercato”

\section{Estrazione del documento del PSN}

Contesto, oggetto e organizzazione del documento
Il Polo Strategico Nazionale S.p.A. (“PSN”) è stato promosso dal Dipartimento per la
trasformazione digitale (“DTD”) della Presidenza del Consiglio dei Ministri, con l’obiettivo
di dotare la Pubblica Amministrazione di un’infrastruttura ad alta affidabilità, resiliente e
indipendente, mettendo in campo tecnologie d’avanguardia.
Il PSN, inoltre, contribuisce alle iniziative promosse dal Piano Nazionale di Ripresa e Resilienza
(“PNRR”), accelerando la trasformazione digitale della Pubblica Amministrazione (“PA”) e
fornendo un'infrastruttura cloud sicura ed affidabile, in linea con la strategia Cloud Italia, che
considera prioritario per le PA il ricorso a tecnologie di tipo cloud.
Il PSN mette a disposizione un’infrastruttura adeguata ad “ospitare la migrazione delle
infrastrutture, delle applicazioni e dei dati frutto della razionalizzazione e del consolidamento
dei Centri di elaborazione Dati e dei relativi sistemi informatici delle pubbliche
amministrazioni1”
La prestazione dei servizi resi a favore delle PA è disciplinata dalla Convenzione sottoscritta
tra PSN e il Dipartimento della Trasformazione Digitale (“DTD”)
La Convenzione prevede, tra le direttrici di evoluzione dei servizi PSN, anche l’integrazione di
nuovi servizi non originariamente previsti dal Catalogo dei Servizi, poiché frutto dell’evoluzione
tecnologica o gestionale, o poiché resi disponibili dal mercato successivamente alla
sottoscrizione della Convenzione.
Il presente documento descrive i nuovi servizi oggetto dell’integrazione per ciascuna delle aree
di servizio del catalogo PSN (i.e., Public Cloud on PSN Managed, Secure Public Cloud, Industry
Standard).
L’ allegato “Listino-Polo-Strategico-Nazionale” riporta il listino dei servizi PSN aggiornato con
l’integrazione dei nuovi servizi.
1 Articolo 5, comma 1, Convenzione
7
1 Panoramica delle aree di integrazione servizi PSN
Guida alla lettura dell’allegato
L’allegato si compone di un foglio “Cover” contenente istruzioni per la lettura dei diversi fogli e
di un foglio per ciascuna delle aree del listino, in particolare “Industry Standard”, “Hybrid Cloud
on PSN site”, “SPC – GCP”, “SPC – Azure”, “SPC - Confidential Azure”, “PSN Managed”.
Ciascuno dei fogli riporta sia i servizi già presenti nell’offerta PSN prima di questa integrazione
che i nuovi servizi oggetto di questa integrazione. Come dettagliato nel seguito di questo
paragrafo, una specifica colonna in ciascun foglio indica se il servizio era già presente o se è
un nuovo servizio. Solo i fogli ““Hybrid Cloud on PSN site” e “Figure professionali” non
contengono nuovi servizi, ma solo servizi già presenti nell’offerta precedente a questa
integrazione.
Foglio “Industry Standard”
Il foglio mantiene la struttura e le colonne del listino precedente a questa integrazione e vede
l’aggiunta dei seguenti campi:
• “FAMIGLIA” e “SOTTO-FAMIGLIA”: i campi contengono una tassonomia “commerciale”
aggiuntiva. che viene affiancata alla tassonomia ufficiale del listino PSN originario, che
è organizzata in “tipologie” e che viene mantenuta nel campo “MAPPATURA SU
TIPOLOGIA OFFERTA ORIGINARIA”- Questa tassonomia “commerciale” viene
aggiunta solo ai fini di una maggiore fruibilità da parte della PA ed è pienamente
mappata sulla tassonomia dell’allegato C di Convenzione.
• “MAPPATURA SU TIPOLOGIA OFFERTA ORIGINARIA”: per ogni servizio, il campo
riporta la “tipologia” di appartenenza, secondo la tassonomia del listino precedente a
questa integrazione - per i servizi pre-esistenti a questa integrazione, questo campo
mantiene il valore già presente a listino;
• “MAPPATURA ALLEGATO L”: per ogni servizio, il campo riporta la “tipologia” di
appartenenza, secondo la tassonomia contenuta nell’Allegato L di Convenzione, elenco
dei servizi Core, No Core e CSP - per i servizi pre-esistenti a questa integrazione, questo
campo non viene popolato;
• “MAPPATURA ALLEGATO H”: per ogni servizio, il campo riporta li indicatori di
riferimento per ogni servizio, coerentemente con l’allegato H di Convenzione;
• “TIER”: come dettagliato ed argomentato nel successivo paragrafo 4.4, in
corrispondenza di questa integrazione, alcuni servizi vengono offerti ingegnerizzate con
8
tecnologie alternative, ovvero tecnologie economicamente più vantaggiose per la PA
(sempre licenziate e con supporto enterprise) e tecnologie più avanzate o leader di
mercato – Questo campo consente di identificare le diverse ingegnerizzazioni di uno
stesso servizio, indicando in generale come tier “Basic” le versioni con tecnologie
economicamente più vantaggiose e come tier “Premium” le eventuali versioni con
tecnologie più avanzate;
• “LISTINO DI PARTENZA O PRIMA INTEGRAZIONE”: il campo specifica se il servizio
contenuto in ciascuna riga del file fosse già presente nell’offerta prima di questa
integrazione o se sia oggetto di questa integrazione;
Foglio “Hybrid Cloud on PSN site”
Il foglio mantiene la struttura e le colonne del listino precedente a questa integrazione e vede
l’aggiunta del solo seguente campo:
• “LISTINO DI PARTENZA O PRIMA INTEGRAZIONE”: il campo specifica se il servizio
contenuto in ciascuna riga del file fosse già presente nell’offerta prima di questa
integrazione o se sia oggetto di questa integrazione.
Fogli “SPC - GCP”, “SPC – Azure” ed “SPC – Confidential Azure”
I fogli mantengono la struttura e le colonne del listino precedente a questa integrazione e
vedono l’aggiunta dei seguenti campi:
• “LISTINO DI PARTENZA O PRIMA INTEGRAZIONE”: il campo specifica se il servizio
contenuto in ciascuna riga del file fosse già presente nell’offerta prima di questa
integrazione o se sia oggetto di questa integrazione.
Foglio “PSN Managed”
Il foglio mantiene la struttura e le colonne del listino precedente a questa integrazione e vede
l’aggiunta dei seguenti campi:
• “LISTINO DI PARTENZA O PRIMA INTEGRAZIONE”: il campo specifica se il servizio
contenuto in ciascuna riga del file fosse già presente nell’offerta prima di questa
integrazione o se sia oggetto di questa integrazione.
9
2 Public Cloud PSN Managed
L'attuale offerta Public Cloud PSN Managed prevista in Convenzione include servizi
ingegnerizzati con tecnologie Oracle e GCP.
La presente integrazione arricchisce l'offerta Public Cloud PSN Managed per quanto attiene ai
servizi basati sulla tecnologia Oracle Alloy.
Invece, per quando riguarda i servizi PSN Managed basati su tecnologia GCP, vengono
confermati i servizi già presenti nell'offerta di Convenzione. PSN è tuttavia impegnato, con
l'Hyperscaler e le autorithies, nel percorso di progettazione, ingegnerizzazione e certificazione
di nuovi servizi PSN Managed basati su tecnologia GCP, che però, alla data di questa Relazione
tecnica, sono in fase di predisposizione, per cause fuori dal controllo di PSN.
I nuovi servizi Public Cloud PSN Managed basati sulla tecnologia Oracle Alloy vengono ora
integrati in virtù di due fattispecie previsto dell’art. 5, comma 4 della Convenzione:
• perché “frutto dell’evoluzione tecnologica”, in quanto, alla data della Convenzione, la
tecnologia Alloy non era ancora stata resa disponibile da Oracle;
• perché “frutto dell’evoluzione gestionale” in quanto la nuova tecnologia abilita anche
nuove funzionalità che l’esperienza maturata da PSN ha rilevato come utili a supportare
le esigenze della PA e la razionalizzazione e migrazione al Cloud dei loro CED.
I servizi PSN Managed Oracle, implementati dal PSN, offrono una piattaforma cloud sicura, ad
alte prestazioni e altamente disponibile, appositamente progettata per rispondere alle
esigenze della PA italiana, per la gestione e la sovranità dei dati critici e strategici delle
amministrazioni centrali, delle ASL e delle principali amministrazioni locali.
I servizi PSN Managed Oracle consentono alle PA di migrare, creare e gestire tutte le operazioni
IT, dai workload esistenti alle nuove applicazioni e piattaforme dati cloud native. La soluzione
garantisce la residenza dei dati in Italia, assicurando la trasparenza e la sovranità degli stessi,
attraverso la separazione fisica e/o logica dei workload e dei rispettivi dati. Questo approccio
risponde alle esigenze di sicurezza e conformità normativa della PA, offrendo un ambiente
tecnologico avanzato e affidabile per il trattamento dei dati pubblici.
I nuovi servizi Public Cloud PSN Managed utilizzano un’infrastruttura sovrana dedicata al PSN
basata sul prodotto Oracle Alloy: l’hardware e il software sviluppati da Oracle vengono ospitati
all’interno dei datacenter del PSN, responsabile di erogare e controllare la connettività con gli
apparati.
I nuovi servizi basati sulla tecnologia Oracle Alloy rappresentano un significativo avanzamento
gestionale rispetto a quelli attuali, basati sulla soluzione Oracle Cloud Infrastructure (OCI)
Dedicated Region (Oracle DRCC).
10
Rispetto all’attuale listino di Public Cloud PSN Managed, che mette già a disposizione delle PA
diverse istanze SQL basate su tecnologia Oracle, ad esempio Oracle Database Exadata Cloud
at Customer o Oracle Autonomous Data Warehouse, con l’introduzione di Oracle Alloy sarà
possibile per le PA accedere ad una gamma più estesa di servizi, precedentemente non abilitati
su Oracle Cloud Infrastructure (OCI) Dedicated Region (Oracle DRCC), comprensivi sia di
elementi infrastrutturali (es. VM, runtime container, etc etc) che di sicurezza.
Tali servizi permettono alle PA di rendere più agevole la migrazione di applicazioni e
infrastruttura esistenti, con particolare attenzione a tutti quelle esigenze di migrazione che
richiedono una vicinanza tra layer applicativo e layer dati basato appunto su tecnologia Oracle.
Con Alloy è infatti possibile erogare un ambiente completo e a latenza molto inferiore rispetto
ad architetture multicloud che prevedono invece l’installazione dell’application in un altro cloud
(es. SPC Azure o SPC Google) e l’esecuzione in PSN Managed Oracle della sola componete di
database.
I nuovi servizi Public Cloud PSN Managed basati su Oracle Alloy consentono alle PA di
accedere ad un tenant dedicato da cui possono attivare una vasta gamma di prodotti Oracle,
con modalità operative analoghe a quelle offerte dal cloud pubblico OCI, ma con tutte le
garanzie tecniche e operative offerte dalla securitizzazione PSN.
Da ultimo, Oracle Alloy permette inoltre al PSN di beneficiare di una gestione multi-tenant più
avanzata e flessibile, facilitando in tal modo il soddisfacimento delle esigenze di
implementazione delle PA.
Il Public Cloud PSN Managed rientra all’interno della classificazione del DTD come servizio
cloud Privato/Ibrido “su licenza”, permettendo di accogliere al suo interno dati e servizi di
rilevanza strategica. Questo approccio prevede la localizzazione dei dati in Italia ed un
maggiore isolamento dalle regioni pubbliche dei principali CSP, garantendo autonomia
operativa attraverso la gestione da parte di un fornitore soggetto a vigilanza e monitoraggio
pubblico, in questo caso il PSN.
La soluzione erogata dal PSN Managed basata su Oracal Alloy è isolata (sia a livello di network
che di infrastruttura) dalle region Public Cloud di OCI, permettendo alle PA di accedere a servizi
dei CSP erogati da Region dedicata al PSN, con separazione logica e gestione operata da
personale PSN.
11
Figura 2 – Infrastruttura PSN Oracle Alloy
Per rispondere alle esigenze delle principali amministrazioni centrali, PSN ha ingegnerizzato
Oracle Alloy all’interno della propria region Sud, con la possibilità di implementarlo anche
all’interno della region Nord. A livello di ridondanza, Oracle Alloy distruibuisce i carichi
utilizzando 3 diversi fault domain permettendo di avere una protezione da fault e HA all’interno
di una singola region. In particolare, presenta le seguenti caratteristiche:
• Tutte le componenti di networking sono ridondate intra-region
• Le componenti di compute sono disponibili in HA con meccanismi di Anty Affinity
• Lo storage è replicato sui 3 fault domain
• I database (Exadata o VM) sono replicati tramite la funzionalità RAC
• L’alimentazione e la connettività viene garantita tramite i livelli di Tier 4 del DC di PSN
• Le operazioni di maintenance vengono eseguite senza tempo di downtime
programmato
Si precisa inoltre che il PSN, oltre ad offrire le funzionalità di housing e facility del DC, applica
anche controlli di sicurezza avanzati, come la crittografia e la gestione delle chiavi tramite
Oracle Vault, in conformità con il paradigma di External Key Manager (EKM).
12
ll PSN gestisce la sottoscrizione del tenant cliente, mentre Oracle è responsabile della gestione
dell'infrastruttura di base. I tenant verranno integrati con i servizi forniti dal PSN, tra cui la
gestione delle identità, il logging, il backup e, come già anticipato, la sicurezza e la gestione
delle chiavi crittografiche.
A livello di Operations, l'offerta PSN Managed Oracle è volta ad assicurare che i livelli di
supporto 1 e 2 siano forniti ai clienti finali direttamente dal PSN. Oracle, invece, fornisce il livello
di supporto 3 attraverso i propri team operativi, sfruttando le proprie entità legali europee, sui
quali il PSN può esercitare il diritto di audit.
La sovranità dei dati è garantita grazie all’impiego di un Realm separato dal cloud Pubblico di
OCI ed il controllo sull’accesso dei dati è gestito centralmente dal PSN, che supporta le
operazioni dei clienti e garantisce i controlli di sicurezza e privacy. I dati archiviati su Oracle
Cloud sono crittografati "at-rest" e le chiavi sono gestite esternamente sulla piattaforma
Thales CCKM e integrate con il Vault di Oracle Cloud, garantendo la sicurezza e la conformità
alle normative sulla protezione dei dati.
Rispetto alle capability native fruibili direttamente attraverso Alloy, la soluzione PSN Managed
Oracle offre alle PA una vasta gamma di vantaggi aggiuntivi, tra cui:
• Servizi integrati: integrazione dei servizi Oracle Cloud con i servizi e piattaforme
gestite da PSN all'interno dei propri Datacenter (i.e., EKM Thales per utilizzo e gestione
delle chiavi di cifratura e SIEM Splunk per la raccolta di log di sicurezza e di audit).
• Backup e Disaster Recovery: possibilità di usufruire di soluzioni di backup erogate dal
PSN.
• Sicurezza e Compliance: possibilità di usufruire di servizi di sicurezza e compliance,
offerte dal PSN, per supportare le PA nell’erogazione dei propri servizi digitali.
Viene riportata di seguito una tabella riepilogativa della presente integrazione dell’offerta PSN
Managed, con le numeriche relative ai servizi attualmente a listino, alle integrazioni PSN
Managed Oracle Alloy effettuate ed ai servizi che sono stati dismessi e sostituiti per ragioni
tecniche o formali/di tassonomia
13
• Servizi già presenti nell’offerta PSN di Luglio 2024: 174
• Servizi integrati, secondo il razionale di Convenzione “evoluzione gestionale”: 232
• Servizi dismessi (motivi tecnici o di tassonomia): 8
• Servizi sostituiti (motivi tecnici o di tassonomia): 13
Si riporta di seguito il dettaglio dei servizi dismessi e sostituiti con il relativo razionale.
SERVIZIO DISMESSO RAZIONALE
Oracle Database Exadata Cloud
at Customer - Database OCPU
Il servizio Cloud at Customer (ExaCC) non è più afferente
all'offerta Cloud di Oracle.
Oracle Database Exadata Cloud
at Customer - Database OCPU –
BYOL
Il servizio Cloud at Customer (ExaCC) non è più afferente
all'offerta Cloud di Oracle.
NoSQL Document DB ops Il servizio NOSQL Document DB non è più afferente all’offerta
Cloud di Oracle.
NoSQL Document DB storage Il servizio NOSQL Document DB non è più afferente all’offerta
Cloud di Oracle.
MySQL Database - Standard - E2 Il servizio MySQL Database è stato reingegnerizzato
astraendosi dalle caratteristiche/tipologie delle CPU/RAM
MySQL Database - Bare Metal
Standard - E2
Il servizio MySQL Database è stato reingegnerizzato
astraendosi dalle caratteristiche/tipologie delle CPU/RAM
MySQL Database - Standard - E3 Il servizio MySQL Database è stato reingegnerizzato
astraendosi dalle caratteristiche/tipologie delle CPU/RAM
MySQL Database - Standard - E3
– Memory
Il servizio MySQL Database è stato reingegnerizzato
astraendosi dalle caratteristiche/tipologie delle CPU/RAM
SERVIZIO SOSTITUITO SERVIZIO INTEGRATO RAZIONALE
Exadata Cloud at Customer -
Autonomous Transaction
Processing - Database OCPU
OCI - Oracle Autonomous
Transaction Processing -
Dedicated
Il servizio Cloud at Customer
(ExaCC) non è più afferente
all'offerta Cloud di Oracle.
14
Il servizio Autonomous può
essere tradotto nella sua
declinazione ExaCS, per cui
viene introdotto un nuovo
servizio con cui sostituire il
precedente.
Exadata Cloud at Customer -
Autonomous Data Warehouse
- Database OCPU
OCI - Oracle Autonomous Data
Warehouse - Dedicated
Il servizio Cloud at Customer
(ExaCC) non è più afferente
all'offerta Cloud di Oracle.
Il servizio Autonomous può
essere tradotto nella sua
declinazione ExaCS, per cui
viene introdotto un nuovo
servizio con cui sostituire il
precedente.
Exadata Cloud at Customer -
Autonomous Transaction
Processing - Database OCPU -
BYOL
OCI - Oracle Autonomous
Transaction Processing -
Dedicated - BYOL
Il servizio Cloud at Customer
(ExaCC) non è più afferente
all'offerta Cloud di Oracle.
Il servizio Autonomous può
essere tradotto nella sua
declinazione ExaCS, per cui
viene introdotto un nuovo
servizio con cui sostituire il
precedente.
Exadata Cloud at Customer -
Autonomous Data Warehouse
- Database OCPU – BYOL
OCI - Oracle Autonomous Data
Warehouse - Dedicated - BYOL
Il servizio Cloud at Customer
(ExaCC) non è più afferente
all'offerta Cloud di Oracle.
Il servizio Autonomous può
essere tradotto nella sua
declinazione ExaCS, per cui
viene introdotto un nuovo
servizio con cui sostituire il
precedente.
Oracle Autonomous Data
Warehouse - Exadata Storage
OCI - Oracle Autonomous Data
Warehouse - Exadata Storage
Il servizio erogato non viene
modificato tecnicamente, si
preferisce modificare la
nomenclatura a seguito della
variazione del listino da parte
di Oracle.
15
Oracle Autonomous
Transaction Processing -
Exadata Storage
OCI - Oracle Autonomous
Transaction Processing -
Exadata Storage
Il servizio erogato non viene
modificato tecnicamente, si
preferisce modificare la
nomenclatura a seguito della
variazione del listino da parte
di Oracle.
Oracle Database Cloud Service
- Standard Edition
OCI - Oracle Base Database
Service - Standard
Il servizio erogato non viene
modificato tecnicamente, si
preferisce modificare la
nomenclatura a seguito della
variazione del listino da parte
di Oracle.
Oracle Database Cloud Service
- Enterprise Edition
OCI - Oracle Base Database
Service - Enterprise
Il servizio erogato non viene
modificato tecnicamente, si
preferisce modificare la
nomenclatura a seguito della
variazione del listino da parte
di Oracle.
Oracle Database Cloud Service
- Enterprise Edition High
Performance
OCI - Oracle Base Database
Service - High Performance
Il servizio erogato non viene
modificato tecnicamente, si
preferisce modificare la
nomenclatura a seguito della
variazione del listino da parte
di Oracle.
Oracle Database Cloud Service
- Enterprise Edition Extreme
Performance
OCI - Oracle Base Database
Service - Extreme Performance
Il servizio erogato non viene
modificato tecnicamente, si
preferisce modificare la
nomenclatura a seguito della
variazione del listino da parte
di Oracle.
Oracle Database Cloud Service
- All Editions – BYOL
OCI - Oracle Base Database
Service - BYOL
Il servizio erogato non viene
modificato tecnicamente, si
preferisce modificare la
nomenclatura a seguito della
variazione del listino da parte
di Oracle.
16
MySQL Database – Storage MySQL Database - Storage Il servizio erogato non viene
modificato tecnicamente, si
preferisce modificare la
nomenclatura a seguito della
variazione del listino da parte
di Oracle.
MySQL Database - Backup
Storage
MySQL Database - Backup
Storage
Il servizio erogato non viene
modificato tecnicamente, si
preferisce modificare la
nomenclatura a seguito della
variazione del listino da parte
di Oracle.
Nell’ambito dell’integrazione dei servizi di Convenzione, oggetto di questa relazione tecnica, i
servizi offerti sono raggruppati nelle seguenti famiglie:
• Core Infrastructure
• Data Management
• AI and Machine Learning
• Security
• Developer Services
• Application
Disaster Recovery
Si precisa che la componente di servizio X9M viene introdotta come ampliamento ed è erogata
esclusivamente all'interno di Alloy, non essendo il Public Cloud di Oracle (OCI) all'interno del
listino PSN.
Nei paragrafi successivi sono dettagliati i servizi PSN Managed Oracle Alloy oggetto di
integrazione.
2.1 Core Infrastructure
All’interno della macrotipologia di “Core Infrastructure” sono previste le seguenti soluzioni:
• Compute (VM, Bare Metal, GPU, Confidential)
• VMware Solution
• Storage (Block, Object, File)
• Networking
17
• Fast Interconnect
• Container and Serverless
L’aggiornamento di questi prodotti infrastrutturali core, (es. OCPU, Ram, Dischi), è anche
dovuto alla disponibilità di un nuovo hardware OCI (i.e., i nuovi server X9) e alla possibilità di
sfruttare la 4° e 5° generazione dei processori “AMD EPYC Processors™”, al fine di rendere
disponibili alle PA le OCPU E4 e E5, che permettono di migliorare le performance fino a 2x
rispetto alla precedente versione.
Le nuove famiglie di VM presenti consentono alle PA di migrare ed eseguire una vasta gamma
di applicazioni, comprese le applicazioni che richiedono elevate prestazioni di calcolo parallelo
(i.e., per l’addestramento di modelli di Machine Learning). Ciò è reso possibile grazie alla
disponibilità di risorse computazionali basate su GPU (Graphic Processor Unit), in particolare i
modelli A100, A10 e L40S.
All’interno della famiglia di “Core Infrastructure” sono inoltre incluse le voci di listino afferenti
al servizio “Oracle Cloud VMware Solution (OCVS)”, che permette alle PA di eseguire carichi di
lavoro VMware direttamente sull'infrastruttura Oracle PSN Managed. Questa soluzione è
particolarmente importante in quanto permette una migrazione fluida e veloce degli ambienti
VMware on-premise al cloud, senza la necessità di modificare le applicazioni esistenti, i
processi operativi e il proprio know-how tecnico, permettendo al contempo alla PA un accesso
completo allo stack VMware.
Le Pubbliche Amministrazioni possono avvalersi delle stesse funzionalità e strumenti di
VMware utilizzati nei propri data center, beneficiando inoltre della scalabilità, flessibilità e delle
avanzate capacità di sicurezza offerte dal cloud PSN Managed.
Tra i componenti fondamentali per la realizzazione di infrastrutture moderne, robuste e
scalabili, i servizi di storage rivestono un ruolo cruciale. PSN Managed Oracle offre alle
Pubbliche Amministrazioni diverse opzioni di memorizzazione, in funzione della tipologia
richiesta (block, file, object) e del livello di affidabilità, includendo soluzioni basate su file system
ZFS.
Completano l’elenco dei servizi “Core Infrastructure” i prodotti dedicati all’esecuzione di
workload containerizzati (sfruttando lo standard di orchestrazione di Kubernetes) e gli oggetti
virtuali necessari a garantire la corretta connettività dei componenti in cloud (es. load balancer,
VPN, IP statici) e esterni all’ambiente Cloud (i.e., connettività FastConnect che permette alle PA
di realizzare un canale di comunicazione dedicato e a latenza garantita tra l’ambiente PSN
Managed Oracle e altre reti, on-premise o cloud).
Ulteriori informazioni inerenti ai servizi della macrotipologia “Core Infrastructure” sono
disponibili nella tabella sottostante:
18
Tipologia Approfondimenti
Compute - Bare Metal I servizi di tale macro-tipologia non conservano i dati del cliente
(rimangono sullo storage cifrato at-rest).
https://docs.oracle.com/en-us/iaas/Content/Compute/home.htm
I servizi in questione sono inscindibili per i clienti che richiedono
la possibilità di sviluppare applicazioni all'interno dello IaaS
“Compute” e “Storage”.
Compute - GPU I servizi di tale macro-tipologia non conservano i dati del cliente
(che rimangono sullo storage cifrato at-rest).
https://docs.oracle.com/en-us/iaas/Content/Compute/home.htm
I servizi in questione sono inscindibili per i clienti che richiedono
la possibilità di sviluppare applicazioni all'interno dello IaaS
“Compute” e “Storage”.
Compute - Virtual Machine I servizi di tale macro-tipologia non conservano i dati del cliente
(che rimangono sullo storage cifrato at-rest).
https://docs.oracle.com/en-us/iaas/Content/Compute/home.htm
I servizi in questione sono inscindibili per i clienti che richiedono
la possibilità di sviluppare applicazioni all'interno dello IaaS
“Compute” e “Storage”.
Compute - VMware I servizi di tale macro-tipologia non conservano i dati del cliente
(che rimangono sullo storage cifrato at-rest).
https://docs.oracle.com/en-us/iaas/Content/VMware/home.htm
I servizi in questione sono inscindibili per i clienti che richiedono
la possibilità di sviluppare applicazioni all'interno dello IaaS
“Compute” e “Storage”.
Compute - VMware with GPU Tali servizi usano dischi locali iNVME privi di integrazione con
EKM esterni.
Container Engine for
Kubernetes
I servizi di tale macro-tipologia non conservano i dati del cliente
(che rimangono sullo storage cifrato at-rest).
https://docs.oracle.com/en-us/iaas/Content/ContEng/home.htm
19
I servizi in questione sono inscindibili per i clienti che richiedono
la possibilità di sviluppare applicazioni all'interno dello IaaS
“Compute” e “Storage”.
Networking I servizi di Tipologia "Networking" sul tipo "Core Infrastructure"
sono servizi di interconnessione di rete. Hanno lo scopo di
collegare servizi e dati che sono stati già securizzati e quindi
veicolano informazioni che non devono e non possono essere
ulteriormente securizzati.
La securitizzazione avviene "at-rest", ovvero "a riposo", in
quanto i dati vengono cifrati a livello di archivi di
memorizzazione.
I servizi afferenti alla famiglia "Networking" prevedono la
cifratura in transit e, nello specifico, tutto il traffico di rete tra le
istanze di calcolo, i database e altri servizi all'interno di Alloy
sono cifrati.
I servizi di “Networking” sono da considerarsi inscindibili per
poter erogare i servizi “Compute” e “Storage” per i clienti.
Networking - FastConnect I servizi di Tipologia "Partner Interconnect" sul tipo "Core
Infrastructure" sono servizi di interconnessione di rete. Hanno lo
scopo di collegare servizi e dati che sono stati già securizzati e
quindi veicolano informazioni che non devono e non possono
essere ulteriormente securizzati.
La securitizzazione avviene "at-rest", ovvero "a riposo", in
quanto i dati vengono cifrati a livello di archivi di
memorizzazione.
I servizi di “Partner Interconnect” sono inscindibili per poter
erogare i servizi “Compute” e “Storage” verso i clienti che
vogliano accedere in maniera sicura.
Storage - Block Volumes https://docs.oracle.com/en-us/iaas/Content/Block/home.htm
Storage - File Storage https://docs.oracle.com/en-us/iaas/Content/Object/home.htm
Storage - Object Storage https://docs.oracle.com/en-us/iaas/Content/Object/home.htm
Storage - ZFS https://docs.oracle.com/en-us/iaas/Content/Object/home.htm
20
2.2 Data Management
All’interno di questa categoria di servizi sono presenti diverse opzioni e prodotti per la gestione
a scala dei dati, consentendo alle PA di utilizzare non solo la tecnologia più adatta per i propri
scopi ma anche di individuare i motori di database più funzionali con lo scenario applicativo e
prestazionale richiesto.
La macro-tipologia "Data Management" inclusa in PSN Managed Oracle permette di scegliere
tra diverse versioni e configurazioni di servizi DBMS basati su tecnologia ORACLE (i.e.,
ExaData, Base Database Service, Autonomous Database, …). Inoltre, consente l'adozione di
altre tecnologie, come Oracle HeatWave, un software specializzato nell'elaborazione inmemory
dei dati, e soluzioni NoSQL, particolarmente efficaci per la gestione di grandi volumi
di dati non strutturati o semi-strutturati oppure soluzioni basate su Hadoop, framework opensource,
progettato per supportare l'elaborazione distribuita e scalabile di grandi volumi di dati
su cluster di server, utilizzando il modello di programmazione MapReduce e un file system
distribuito (HDFS) per garantire affidabilità e velocità di accesso ai dati.
In questa famiglia di prodotti sono inoltre presenti tutti i servizi ed addon che consentono alle
PA di adottare in modo flessibile l’infrastruttura del Polo, ad esempio sono previsti dei prodotti
di storage specifici che possono consentire alle PA di acquistare solo il quantitativo di risorse
adeguato alle proprie esigenze, evitando gli sprechi. Sono infine previsti gli aggiornamenti di
voci di listino imputabili al refresh tecnologico intercorso, come il passaggio a versioni più
recenti delle piattaforme computazionali sottostanti (AMD EPYC Processors™).
Ulteriori informazioni inerenti ai servizi della macrotipologia “Data Management” sono
disponibili nella tabella sottostante:
Tipologia Approfondimenti
Autonomous Database I servizi “Oracle Autonomous”, “Base Database Service”,
“Exadata Cloud Infrastructure” supportano nativamente
l'utilizzo di un EKM.
https://docs.oracle.com/en-us/iaas/autonomousdatabase/
index.html
Autonomous JSON Database I servizi “Oracle Autonomous”, “Base Database Service”,
“Exadata Cloud Infrastructure” supportano nativamente
l'utilizzo di un EKM.
https://docs.oracle.com/en-us/iaas/autonomousdatabase/
index.html
21
Backup I seguenti servizi non conservano dati del cliente (che rimangono
sullo storage cifrato at-rest):
https://docs.oracle.com/en/learn/backup-and-restore-standbydb/
index.html#introduction
Il servizio "OCI - Oracle Database Backup Cloud - Object
Storage" è abilitante al backup di database gestiti dal cliente
sullo storage target "OCI - Object Storage - Storage", per cui non
presenta dati.
Base Database Service I servizi “Oracle Autonomous”, “Base Database Service”,
“Exadata Cloud Infrastructure” supportano nativamente
l'utilizzo di un EKM.
https://docs.oracle.com/en-us/iaas/base-database/index.html
Database Management I seguenti servizi non conservano dati del cliente (che rimangono
sullo storage cifrato at-rest):
https://docs.oracle.com/en-us/iaas/databasemanagement/
home.htm
Database Migration I seguenti servizi non conservano dati del cliente (che rimangono
sullo storage cifrato at-rest):
https://docs.oracle.com/en-us/iaas/databasemigration/
index.html
Exadata Cloud Infrastructure I servizi “Oracle Autonomous”, “Base Database Service”,
“Exadata Cloud Infrastructure” supportano nativamente
l'utilizzo di un EKM.
https://docs.oracle.com/en-us/iaas/exadatacloud/index.html
Exadata Database Service I servizi “Oracle Autonomous”, “Base Database Service”,
“Exadata Cloud Infrastructure” supportano nativamente
l'utilizzo di un EKM.
https://docs.oracle.com/en-us/iaas/exadatacloud/index.html
Exadata Exascale Infrastructure I servizi “Oracle Autonomous”, “Base Database Service”,
“Exadata Cloud Infrastructure” supportano nativamente
l'utilizzo di un EKM.
https://docs.oracle.com/en-us/iaas/exadb-xs/index.html
22
Golden Gate I seguenti servizi non conservano dati del cliente (che rimangono
sullo storage cifrato at-rest):
https://docs.oracle.com/en-us/iaas/goldengate/index.html
Hadoop Big Data Service I seguenti servizi non conservano dati del cliente (che rimangono
sullo storage cifrato at-rest):
https://docs.oracle.com/en-us/iaas/Content/bigdata/home.htm
I servizi in questione sono inscindibili per i clienti che richiedono
la possibilità di sviluppare applicazioni all'interno dello IaaS
“Compute” e “Storage”
In-Memory Caching I seguenti servizi non conservano dati del cliente (che rimangono
sullo storage cifrato at-rest) in quanto servizio “in-memory”:
https://docs.oracle.com/en-us/iaas/Content/ocicache/home.htm
MySQL I seguenti servizi, ad oggi, non supportano le funzionalità di EKM:
“MySQL in previsione per CY25H1”
NoSQL I seguenti servizi, ad oggi, non supportano le funzionalità di EKM:
“NoSQL in previsione per CY25H1”
PostgreSQL I seguenti servizi, ad oggi, non supportano le funzionalità di EKM:
“PostgreSQL in previsione per CY25H1”
Visualization I seguenti servizi non conservano dati del cliente (che rimangono
sullo storage cifrato at-rest):
https://docs.oracle.com/en/cloud/paas/analyticscloud/
acsgs/index.html
https://docs.oracle.com/en/cloud/paas/analyticscloud/
acubi/introduction-visualization-and-reporting.html
I servizi in questione sono inscindibili per i clienti che richiedono
la possibilità di sviluppare applicazioni all'interno dello IaaS
“Compute” e “Storage”.
23
2.3 AI and Machine Learning
I prodotti all’interno della categoria di “AI and Machine Learning” rappresentano una suite
completa di servizi avanzati basati sull'intelligenza artificiale e su machine learning, progettati
per potenziare l'innovazione e migliorare l'efficienza operativa delle PA.
Questi servizi abbracciano una vasta gamma di funzionalità, tra cui chatbot intelligenti, analisi
del linguaggio naturale, rilevamento di anomalie, riconoscimento vocale e visione artificiale.
Rientrano in questa categoria i servizi di:
• Data Labeling: Servizio che consente di etichettare i dati per l'addestramento di modelli
di machine learning, facilitando la preparazione di dataset annotati.
• Speech: Servizio di riconoscimento vocale che converte l'audio parlato in testo, utile per
applicazioni di trascrizione e assistenti virtuali.
• Vision - Image Analysis: Analisi delle immagini utilizzando algoritmi di visione artificiale
per riconoscere oggetti, scene e attività all'interno delle immagini.
• Vision - OCR: Riconoscimento ottico dei caratteri (OCR) che estrae testo da immagini e
documenti digitalizzati.
• Vision - Custom Training: Addestramento personalizzato di modelli di visione artificiale
utilizzando dataset specifici per applicazioni personalizzate.
• Oracle Cloud AI Services - Language - Pre-trained Inferencing: Inferenza con modelli di
linguaggio pre-addestrati per compiti come l'analisi del sentiment, la classificazione del
testo e il riconoscimento delle entità.
• Language - Custom Training: Addestramento personalizzato di modelli di linguaggio su
dataset specifici per migliorare le performance su compiti particolari.
• Language - Text Translation: Traduzione automatica del testo tra diverse lingue
utilizzando modelli di machine translation.
• Document Understanding - OCR: Estrarre testo dai documenti digitalizzati tramite
riconoscimento ottico dei caratteri.
• Document Understanding - Document Properties: Rilevare e analizzare proprietà e
metadati dei documenti, come autore, data e tipo di documento.
• Document Understanding - Document Extraction: Estrazione di informazioni strutturate
dai documenti, come nomi, indirizzi e numeri di telefono.
• Document Understanding - Custom Training: Addestramento personalizzato per
migliorare l'accuratezza dell'estrazione dei dati e la comprensione dei documenti.
24
• Document Understanding - Custom Document Properties: Definire e rilevare proprietà
personalizzate nei documenti per applicazioni specifiche.
• Document Understanding - Custom Document Extraction: Estrarre dati personalizzati
da documenti in base a esigenze specifiche del business.
• Generative AI - Cohere: Integrazione di modelli di intelligenza artificiale generativa di
Cohere per creare contenuti nuovi e creativi.
• Generative AI - Llama2-70: Utilizzo del modello generativo Llama2-70 per la creazione
di contenuti avanzati.
• WebCenter Content: Gestione dei contenuti aziendali, offrendo strumenti per creare,
gestire e distribuire contenuti digitali.
• WebCenter Enterprise Capture: Soluzione per la cattura e digitalizzazione dei
documenti aziendali, migliorando l'efficienza dei processi di acquisizione dei dati.
• WebCenter Forms Recognition: Riconoscimento e elaborazione automatica dei moduli
cartacei e digitali per estrarre dati strutturati.
• WebCenter Imaging: Gestione delle immagini aziendali, con strumenti per
l'archiviazione, l'indicizzazione e il recupero delle immagini digitali.
• WebCenter Portal: Piattaforma per creare portali aziendali personalizzati, facilitando la
collaborazione e l'accesso alle informazioni.
• WebCenter Sites: Gestione dei siti web aziendali, offrendo strumenti per la creazione,
gestione e pubblicazione di contenuti web.
• WebCenter Sites Satellite Server: Server per migliorare la distribuzione e la gestione dei
contenuti web, garantendo alta disponibilità e performance.
• WebCenter Universal Content Management: Soluzione completa per la gestione dei
contenuti aziendali, integrando strumenti di gestione, collaborazione e distribuzione.
Con strumenti per la gestione e l'estrazione dei dati, l’offerta PSN Managed di Oracle consente
alle organizzazioni di automatizzare processi complessi, migliorare la precisione delle
operazioni e ottenere preziose informazioni dai loro dati (Document Understanding).
In questa categoria di servizi rientrano strumenti che permettono di creare chatbot e assistenti
virtuali intelligenti oppure che consentono l’analisi del sentiment e la classificazione del testo.
Il servizio di “Anomaly detection” è in grado di analizzare grossi volumi di dati tabellari al fine
di identificare pattern anomali, che possano rappresentare frodi, errori o eventi inusuali.
25
Sono inoltre previsti servizi di “Data labelling”, per ottenere un’etichettatura dei dati al fine di
alimentare ulteriori algoritmi di ML e AI, e strumenti in grado di elaborare il linguaggio naturale
(testo, voce) o immagini al fine di estrarre informazioni, riconoscere oggetti o classificare i
materiali esaminati.
Tutti questi servizi sono esposti tramite API e in modalità As-A-Service e consentono le PA di
integrare tali funzionalità all’interno delle proprie soluzioni e/o processi ma soprattutto di non
dover spostare i propri dati al di fuori del perimetro PSN per poter usufruire di queste feature.
Tutti i servizi indicati all'interno dell'offerta Cloud PSN Managed (Oracle Alloy) vengono
erogati esclusivamente all'interno delle Region e dei Data Center PSN, garantendo quindi la
sovranità del dato. Tutti i servizi non conservano dati del cliente (che rimangono sullo storage
cifrato at-rest).
Ulteriori informazioni inerenti ai servizi della macrotipologia “AI and Machine Learning” sono
disponibili nella tabella sottostante:
Tipologia Approfondimenti
Data Preparation
I seguenti servizi non conservano dati del cliente (che rimangono
sullo storage cifrato at-rest):
https://docs.oracle.com/en-us/iaas/Content/datalabeling/
using/home.htm
Generative AI I seguenti servizi non conservano dati del cliente (che rimangono
sullo storage cifrato at-rest):
https://docs.oracle.com/en-us/iaas/Content/generativeai/
home.htm
Image Recognition
I seguenti servizi non conservano dati del cliente (che rimangono
sullo storage cifrato at-rest):
https://docs.oracle.com/enus/
iaas/Content/vision/using/home.htm
Speech Recognition
I seguenti servizi non conservano dati del cliente (che rimangono
sullo storage cifrato at-rest):
https://docs.oracle.com/en-us/iaas/Content/speech/home.htm
26
Text Analysis
I seguenti servizi non conservano dati del cliente (che rimangono
sullo storage cifrato at-rest):
https://docs.oracle.com/en-us/iaas/language/using/home.htm
https://docs.oracle.com/en-us/iaas/Content/documentunderstanding/
using/home.htm
WebCenter Content I seguenti servizi non conservano dati del cliente (che rimangono
sullo storage cifrato at-rest):
https://docs.oracle.com/en-us/iaas/contentmanagement/
index.html
2.4 Security
Rientrano all’interno di questa macrocategoria di prodotti strumenti avanzati di sicurezza ad
ampio respiro quali ad esempio:
• Application Firewall ed Enterprise Firewall
• Identity Management
• Security Monitoring, Assessment, and Advice
• Observability
• Key Management
• Sensitive Data Identification
I prodotti di “Application Firewall” e di “Enterprise firewall” sono pensati per il monitoraggio del
traffico web, HTTP e HTTPS al fine di bloccare attività malevole o non autorizzate mentre i
prodotti di “Network Firewall” sono pensati per proteggere e controllare qualsiasi tipologia di
traffico, non solamente di tipo web.
I prodotti di “Identity management” sono invece volti a garantire la conformità della sicurezza
e la gestione dell’identità degli utenti e degli accessi. Tramite questi prodotti le PA possono
gestire le identità di utenti e “client” (cioè anche applicazioni oltre che utenti) esterni,
permettendo loro di accedere in modo sicuro alle risorse cloud utilizzate.
La suite di” Identity management “è particolarmente completa in quanto consente di gestire
non solo utenti interni ma anche credenziali per utenti esterni. Sono inoltre supportati metodi
avanzati di interazione con gli utenti finali, come ad esempio l’utilizzo di SMS, token e sono a
disposizione funzionalità di replica dei dati attraverso più region / ambienti.
27
Questa funzionalità garantisce che le informazioni critiche riguardanti gli utenti, i gruppi, i
permessi e le politiche di accesso siano sincronizzate e disponibili in più location per migliorare
la resilienza, la disponibilità e il disaster recovery.
Inoltre, facilita la gestione centralizzata delle identità e degli accessi in ambienti distribuiti,
riducendo i rischi di incongruenze e assicurando che le politiche di sicurezza siano
uniformemente applicate.
All’interno di questa categoria di prodotti sono inoltre inclusi diversi prodotti di “Security
Monitoring, Assessment, and Advice”. Questi servizi monitorano continuamente le risorse cloud
per rilevare, prevenire e rispondere a minacce alla sicurezza, utilizzano l'intelligenza artificiale
e l'analisi avanzata per identificare attività sospette, anomalie e configurazioni non conformi
alle best practice di sicurezza, forniscono report dettagliati e strumenti per la gestione delle
vulnerabilità, consentendo alle organizzazioni di mantenere un elevato livello di sicurezza e
conformità per le loro risorse cloud in un ambiente enterprise.
All’interno di questa categoria di prodotti sono riportate diverse opzioni di Observability.
PSN Managed Oracle offre una suite completa di strumenti per il monitoraggio e la gestione
delle applicazioni e delle infrastrutture come ad esempio:
• Health Checks include versioni Basic e Premium per monitorare lo stato di salute delle
risorse cloud, assicurando il funzionamento ottimale e l'identificazione proattiva dei
problemi
• Monitoring che copre sia l’ingestion che il recupero dei dati, fornendo una visione
dettagliata delle metriche di performance
• Notifications che permettono la consegna di allarmi via HTTPS, email e SMS,
garantendo che gli amministratori siano informati in tempo reale su eventi critici
• Logging e Logging Analytics che offrono funzionalità di archiviazione e analisi dei log,
con opzioni di storage attivo e archivio per una gestione efficiente dei dati di log
• Operations Insights che fornisce approfondimenti operativi per database cloud,
database Oracle esterni e data warehouse, migliorando la gestione e fornendo spunti
per eventuali ottimizzazioni ed interventi
• Application Performance Monitoring Service che consente di identificare eventuali colli
di bottiglia e comportamenti applicativi non ottimali oppure consente di simulare
condizioni di stress sugli ambienti per verificare che queste ultime possano supportare
carichi reali una volta in produzione
28
• Operations Insights for Oracle Autonomous Databases che fornisce un monitoraggio
completo, assicurando una gestione efficace delle risorse DB e in particolare degli
Autonomous Database.
Nella categoria di “Key Management” invece rientrano quelle soluzioni particolarmente
importanti per garantire non solo i più elevati livelli di sicurezza e crittografia (le singole PA
possono cifrare i propri dati mediante le chiavi crittografiche gestite in questi prodotti) ma
anche il grado di sovranità necessaria e richiesta sull’infrastruttura del PSN Managed Oracle.
L’ultima categoria di prodotti inclusi è afferente alla possibilità di gestire informazioni sensibili
e applicare contromisure come, ad esempio, la rimozione o il mascheramento di certi tipi di
informazioni, ad esempio quelle identificate con pattern specifici come carte di credito, date,
codici fiscali etc etc.
Ulteriori informazioni inerenti ai servizi della macrotipologia “Security” sono disponibili nella
tabella sottostante:
Tipologia Approfondimenti
Application Firewall I servizi “Cloud Security” hanno lo scopo di securizzare gli
ambienti “Compute” e “Storage”dei clienti; non conservano i dati
dei clienti che rimangono ospitati sullo storage che è cifrato atrest.
https://docs.oracle.com/en-us/iaas/Content/WAF/home.htm
I servizi “Cloud Security” sono inscindibili dai servizi “Compute”
e “Storage” perché sono abilitanti ai meccanismi di
securitizzazione.
Enterprise Firewall I servizi “Cloud Security” hanno lo scopo di securizzare gli
ambienti “Compute” e “Storage” dei clienti; non conservano i dati
dei clienti che rimangono ospitati sullo storage che è cifrato atrest.
https://docs.oracle.com/en-us/iaas/Content/networkfirewall/
home.htm
I servizi “Cloud Security” sono inscindibili dai servizi “Compute”
e “Storage” perché sono abilitanti ai meccanismi di
securitizzazione.
Identity Management I servizi “Cloud Security” hanno lo scopo di securizzare gli
ambienti “Compute” e “Storage” dei clienti; non conservano i dati
dei clienti che rimangono ospitati sullo storage che è cifrato atrest.
29
https://docs.oracle.com/en-us/iaas/Content/Identity/home.htm
I servizi “Cloud Security” sono inscindibili dai servizi “Compute”
e “Storage” perché sono abilitanti ai meccanismi di
securitizzazione.
Key Management I servizi di tale macro tipologia consentono la gestione delle
chiavi di crittografia e sono inscindibili dal servizio EKM esterno
di THALES.
https://docs.oracle.com/enus/
iaas/Content/KeyManagement/home.htm
Observability I servizi di tale macro tipologia non conserva i dati del cliente
(che rimangono sullo storage cifrato at-rest).
https://docs.oracle.com/enus/
iaas/Content/HealthChecks/home.htm
https://docs.oracle.com/enus/
iaas/Content/Monitoring/home.htm
https://docs.oracle.com/enus/
iaas/Content/Notification/home.htm
https://docs.oracle.com/en-us/iaas/operationsinsights/
home.htm
https://docs.oracle.com/en-us/iaas/application-performancemonitoring/
home.htm
I servizi in questione sono inscindibili per i clienti che vogliono
utilizzare strumenti di operation per la gestione del loro tenant.
I seguenti servizi, che presentando dati cifrati at-rest, sono
inscindibili per l'attivazione della corretta fruizione dei
meccanismi di securitizzazione (Landing Zone) come da
progetto collaudato con il DTD:
“OCI - Logging – Storage”
“OCI - Logging Analytics - Archival Storage”
“OCI - Logging Analytics - Active Storage”
https://docs.oracle.com/en-us/iaas/Content/Logging/home.htm
https://docs.oracle.com/en-us/iaas/logging-analytics/home.htm
30
Security Monitoring,
Assessment, and Advice
I servizi di tale macro tipologia non conserva i dati del cliente
(che rimangono sullo storage cifrato at-rest).
https://docs.oracle.com/en-us/iaas/cloud-guard/home.htm
https://docs.oracle.com/en-us/iaas/accessgovernance/
index.html
I servizi in questione sono inscindibili per i clienti che vogliono
utilizzare strumenti di sicurezza per la gestione del loro tenant.
Sensitive Data Identification I servizi di tale macro tipologia non conserva i dati del cliente
(che rimangono sullo storage cifrato at-rest).
https://docs.oracle.com/en-us/iaas/data-safe/index.html
I servizi in questione sono inscindibili per i clienti che vogliono
utilizzare strumenti di operation per la gestione del loro tenant.
2.5 Developer Services
I servizi offerti all’interno del cloud PSN Managed non sono solo di carattere infrastrutturale
ma sono anche orientati alla semplificazione e all’efficientamento delle capacità di sviluppo ed
ingegnerizzazione di nuovi software da parte delle PA.
A questo proposito sono previsti una serie di strumenti avanzati progettati per migliorare lo
sviluppo delle applicazioni, la gestione dei server e l'integrazione delle API, garantendo al
contempo un ambiente di sviluppo flessibile e scalabile. Questi strumenti sono fondamentali
per le PA che desiderano accelerare il processo di sviluppo, migliorare le prestazioni delle
applicazioni e semplificare la gestione dell'infrastruttura.
Esempi concreti di quanto integrato nell’offering del Polo sono rappresentati da Oracle Visual
Builder, una piattaforma lowcode pensata per i team di sviluppo che desiderano prototipare
rapidamente nuove idee, creare interfacce utente intuitive e integrare facilmente i dati aziendali
senza dover scrivere codice complesso. Oracle Visual Builder offre un ambiente di sviluppo
integrato che supporta l’intero ciclo di vita dell'applicazione, dalla progettazione alla
distribuzione, garantendo tempi di sviluppo ridotti e una maggiore qualità delle applicazioni.
Contribuiscono ad arricchire l’offerta anche i prodotti della famiglia “WebLogic”, application
server evoluti e maturi in grado di eseguire applicazioni Java di classe EE sia in ambito
tradizionale, e quindi utilizzando macchine virtuali come infrastruttura, sia che all’interno
ambienti containerizzati (in particolare che sfruttino kubernetes come cluster management).
Infine, sempre per supportare l’adozione e il passaggio ad un’infrastruttura evoluta come
quella offerta dal Polo, sono presenti gli ambienti di sviluppo ed acceleratori compatibili con
31
APEX, la piattaforma di sviluppo rapida che permette di creare applicazioni web interattive e
dinamiche utilizzando strumenti di sviluppo low-code.
Ulteriori informazioni inerenti ai servizi della macrotipologia “Developer Services” sono
disponibili nella tabella sottostante:
Tipologia Approfondimenti
API Management I seguenti servizi non conservano dati del cliente (che rimangono
sullo storage cifrato at-rest):
https://docs.oracle.com/enus/
iaas/Content/APIGateway/home.htm
I “Developer Services” sono inscindibili dai servizi “Compute” e
“Storage” per quei clienti che hanno necessità di sviluppare in
ottica cloud-native sulla piattaforma del PSN.
Digital Assistant I servizi di tale macro-tipologia non conservano i dati del cliente
(che rimangono sullo storage cifrato at-rest).
https://docs.oracle.com/en-us/iaas/digital-assistant/index.html
I “Developer Services” sono inscindibili dai servizi “Compute” e
“Storage” per quei clienti che hanno necessità di sviluppare in
ottica cloud-native sulla piattaforma del PSN.
Email Distribution I servizi di tale macro-tipologia non conservano i dati del cliente
(che rimangono sullo storage cifrato at-rest).
https://docs.oracle.com/en-us/iaas/Content/Email/home.htm
I “Developer Services” sono inscindibili dai servizi “Compute” e
“Storage” per quei clienti che hanno necessità di sviluppare in
ottica cloud-native sulla piattaforma del PSN.
Function I servizi di tale macro-tipologia non conservano i dati del cliente
(che rimangono sullo storage cifrato at-rest).
https://docs.oracle.com/enus/
iaas/Content/Functions/home.htm
I “Developer Services” sono inscindibili dai servizi “Compute” e
“Storage” per quei clienti che hanno necessità di sviluppare in
ottica cloud-native sulla piattaforma del PSN.
32
Low Code Application I seguenti servizi non conservano dati del cliente (che rimangono
sullo storage cifrato at-rest):
https://docs.oracle.com/en-us/iaas/visual-builder/index.html
https://docs.oracle.com/en-us/iaas/apex/index.html
https://www.oracle.com/java/weblogic/weblogic-for-oraclecloud-
infrastructure/
I “Developer Services” sono inscindibili dai servizi “Compute” e
“Storage” per quei clienti che hanno necessità di sviluppare in
ottica cloud-native sulla piattaforma del PSN.
Workflow Orchestration I seguenti servizi non conservano dati del cliente (che rimangono
sullo storage cifrato at-rest):
https://docs.oracle.com/en-us/iaas/processautomation/
index.htm
I “Developer Services” sono inscindibili dai servizi “Compute” e
“Storage” per quei clienti che hanno necessità di sviluppare in
ottica cloud-native sulla piattaforma del PSN.
2.6 Application
All’interno di questa categoria di prodotti compaiono dei software evoluti che permettono alle
PA di eseguire applicazioni in modo serverless, cioè senza doversi occupare di creare,
manutenere, gestire o configurare l’infrastruttura necessaria e che abilitano scenari di calcolo
“event-driven”, e servizi volti all’integrazione di dati e/o di diverse applicazioni.
I servizi serverless abilitano un nuovo modello di funzionamento per gli applicativi che possono
quindi beneficiare di una scalabilità senza precedenti e possono eseguire funzioni solo in
risposta a determinati eventi esterni o interni, liberando risorse nei momenti di mancato utilizzo
del sistema.
Questo nuovo paradigma di cloud computing non offre solo vantaggi tecnologici e di
scalabilità, ma consente ai programmatori di focalizzarsi sulla scrittura del codice, riducendo il
tempo necessario per la gestione dell'infrastruttura e dell'ambiente. Inoltre, permette di
segmentare la logica di business in servizi o microservizi, rendendo il codice più manutenibile
e comprensibile a lungo termine.
I prodotti invece afferenti alla categoria delle API Integration consentono invece alle PA di far
interagire tra loro processi e applicazioni diverse in logica real-time oppure abilitare
l’elaborazione sincrona (streaming) o batch (in blocchi predefiniti) di grosse moli di informazioni
da e per diverse fonti informative diverse.
33
Questi strumenti permettono di estrarre dati dai database di produzione per alimentare un
data lake o un data warehouse, abilitando analisi avanzate. Inoltre, supportano i processi
decisionali basati sulle evidenze raccolte in tempo reale dai sistemi sorgente o da altre
applicazioni, sia all'interno della singola PA che in altri enti. Tra le caratteristiche fondamentali
di questi prodotti, come ad esempio GoldenGate, Oracle Integration Cloud Service o Process
Automation, vi sono il supporto a vari protocolli e formati di dati, per garantire l'interoperabilità
tra piattaforme eterogenee; strumenti di gestione e monitoraggio che consentono di
visualizzare e controllare i flussi di dati e le integrazioni in tempo reale; e la capacità di
orchestrazione dei processi, che permette di definire e automatizzare sequenze complesse di
attività tra diverse applicazioni.
Tutte queste soluzioni sono progettate per essere scalabili e flessibili, adattandosi alle esigenze
di crescita e cambiamento delle PA, e per fornire strumenti di sicurezza avanzati per
proteggere i dati durante il trasferimento e l'integrazione.
Ulteriori informazioni inerenti ai servizi della macrotipologia “Applications” sono disponibili
nella tabella sottostante:
Tipologia Approfondimenti
Data Exploration I seguenti servizi non conservano dati del cliente (che rimangono
sullo storage cifrato at-rest):
https://docs.oracle.com/en-us/iaas/Content/searchopensearch/
home.htm
ETL I seguenti servizi non conservano dati del cliente (che rimangono
sullo storage cifrato at-rest):
https://docs.oracle.com/en-us/iaas/data-integration/home.htm
Integration I servizi di tale macro-tipologia non conservano i dati del cliente
(che rimangono sullo storage cifrato at-rest).
https://docs.oracle.com/en-us/iaas/integration/index.html
https://docs.oracle.com/enus/
iaas/Content/Streaming/home.htm
https://docs.oracle.com/en/cloud/paas/integration-cloud/soasuite-
adapter/index.html
https://docs.oracle.com/en-us/iaas/data-integration/home.htm
34
I servizi in questione sono inscindibili per i clienti che richiedono
la possibilità di sviluppare applicazioni all'interno dello IaaS
“Compute” e “Storage”
Queue Messaging I seguenti servizi non conservano dati del cliente (che rimangono
sullo storage cifrato at-rest):
https://docs.oracle.com/en-us/iaas/Content/queue/home.htm
Secure Desktops I servizi di tale macro-tipologia non conservano i dati del cliente
(che rimangono sullo storage cifrato at-rest).
https://docs.oracle.com/en-us/iaas/secure-desktops/home.htm
2.7 Disaster Recovery
In generale l'architettura fisica del cloud pubblico di OCI garantisce la massima disponibilità
delle risorse attraverso la sua struttura fisica a tre livelli. Oracle Cloud Infrastructure è
organizzata in Regioni, Availability Domain (AD) e Fault Domain.
Una regione rappresenta un'area geografica specifica, mentre un Availability Domain
comprende uno o più data center all'interno di una regione. Gli Availability Domain sono
progettati per essere isolati l'uno dall'altro, resilienti ai guasti ed è altamente improbabile che
falliscano contemporaneamente. Grazie all'assenza di un’infrastruttura condivisa, ovvero
evitando componenti comuni come l'alimentazione, il raffreddamento e la rete interna, un
guasto in un Availability Domain all'interno di una regione ha un impatto minimo sulla
disponibilità degli altri domini nella stessa regione.
Ogni Availability Domain è composto da tre Fault Domain, che sono cluster di hardware e
infrastruttura.
I fault domain garantiscono l'anti-affinità, consentendo di distribuire le istanze in modo che
non vengano collocate sull'hardware fisico dello stesso fault domain all'interno di un singolo
Availability Domain. Questa configurazione garantisce la resilienza, poiché un guasto
hardware o un evento di manutenzione specifico di un Fault Domain non influisce sulle istanze
negli altri Fault Domain.
I tenant hanno la possibilità di sottoscrivere più regioni e distribuire i propri dati e servizi tra di
esse. Combinando altri servizi OCI come DNS, i tenant possono preparare i propri sistemi per
resistere ai disastri.
35
L'architettura Alloy con cui è indirizzata la soluzione PSN Managed Oracle garantisce la
massima disponibilità delle risorse attraverso una struttura fisica a due livelli: Fault Domain e
Region. Entro Q3 2025 il PSN sarà dotato di due Region Alloy (Region Sud e Region Nord).
Ogni Region è composta da tre Fault Domain, separati tra loro sia logicamente che
fisicamente.
Ogni Fault Domain garantisce:
• anti-affinità: garantisce la distribuzione delle istanze in modo che non vengano
collocate sullo stesso hardware fisico
• resilienza: un eventuale guasto hardware e\o un eventuale evento di manutenzione
all'interno di un Fault Domain non influisce sulle istanze negli altri Fault Domain.
Inoltre, attraverso la disponibilità della doppia Region, è possibile abilitare il Disaster Recovery
dei servizi dei singoli clienti a protezione degli eventuali scenari di disastro di una intera Region
con RPO definiti in Allegato H (IQ12 e IQ14) e relativi campionamenti previsti come da
convenzione (104/anno).
Il Polo mette a disposizione su ciascun sito ospitante la Region Alloy di Oracle, l'infrastruttura
di rete, i tool e le tecnologie abilitanti il servizio di Disaster Recovery. L'Amministrazione potrà
in maniera autonoma o qualora lo ritenga necessario, richiedendo il supporto di PSN,
configurare le repliche inter-region secondo il proprio perimetro applicativo e necessità.
L'infrastruttura Alloy garantisce la sovranità dei dati attraverso:
• installazione all'interno dei Data Center del PSN
• interconnessione geografica tra le due Region su rete esclusiva del PSN
• configurazione dei sistemi all'interno della sicurezza perimetrale del PSN
• cifratura dei dati secondo le specifiche ACN per dati e servizi critici e strategici
Tutti i servizi indicati all'interno della soluzione PSN Managed in tecnologia Oracle sono
attivati e residenti unicamente all'interno della piattaforma Alloy del PSN, quindi all'interno
del perimetro di sicurezza implementato e gestito dal PSN.
Ulteriori informazioni inerenti i servizi della macrotipologia “Disaster Recovery” sono disponibili
nella tabella sottostante:
Tipologia Approfondimenti
36
Disaster Recovery as-a-Service I servizi di “Disaster Recovery as-a-Service” sono funzionalità
abilitanti all'implementazione del Disaster Recovery della
produzione sulla seconda Region del PSN. Non sono servizi che
conservano dati, ma che abilitano i meccanismi di replica dello
storage che è di fatto cifrato at-rest.
https://docs.oracle.com/en-us/iaas/disaster-recovery/index.html
I servizi “Disaster Recovery as-a-Service) sono inscindibili per
poter abilitare il Disaster Recovery per i clienti che hanno servizi
strategici.
37
3 Secure Public Cloud
3.1 L’attuale offerta PSN di Convenzione
L’attuale offerta PSN include i servizi di Secure Public Cloud, che consentono alle Pubbliche
Amministrazioni aderenti di accedere ai prodotti e all’infrastruttura di calcolo delle region
pubbliche degli Hyperscaler selezionati, con l’aggiunta da parte del PSN di tutti gli elementi di
sicurezza necessari per una gestione “sovrana“ delle informazioni.
L’architettura di questi servizi prevede di due componenti principali:
• Public Cloud: l’infrastruttura erogata da una o più region da parte dell’Hyperscaler e
fornitore di Cloud pubblico che deve necessariamente risiedere su territorio nazionale;
agli ambienti Cloud il PSN applica una serie di configurazione, policy e controlli di
sicurezza al fine di garantire ai clienti ambienti segregati e conformi agli scopi del PSN;
• Security & governance: l’insieme di tutti gli elementi necessari per garantire l’adeguato
livello di sicurezza dei servizi erogati sul Public Cloud, come ad esempio, a titolo
illustrativo e non esaustivo, componenti di gestione chiavi crittografiche e backup; tali
componenti sono erogati dai data center del PSN distribuiti sul territorio nazionale.
Scendendo maggiormente nel dettaglio, l’implementazione già collaudata si basa 5 elementi
chiave per l’implementazione e la securitizzazione dei servizi SPC:
a) Gestione delle chiavi
La gestione delle chiavi di crittografia è esterna al perimetro di controllo del CSP
b) Governance Model
Viene garantita la security by policy/design creando per ogni cliente un ambiente
standard segregato e auto-consistente
c) Confidential Computing
Il dato, anche quando in uso, è inaccessibile agli operatori del cloud provider
d) Soluzione Hub&Spoke
Tutto il traffico di rete è controllato e monitorato
e) Back-up
I back-up sono memorizzati all’interno del cloud privato del PSN
L’attuale offerta PSN include due piattaforme di Secure Public Cloud: Microsoft Azure e Google
Cloud Platform, erogate rispettivamente da Microsoft e Google.
38
I servizi ed i prezzi esplicitati nel listino attuale sono afferenti al sottoinsieme di servizi utili a
realizzare delle “progettualità che simulano […] una PAC di media-grande dimensione e una
PAC di medio-piccola dimensione” (v. Allegato 3, Sub 5).
3.2 La recente esplicitazione dei Servizi GCP ed Azure e l’introduzione di AWS
Al fine di consentire la massima flessibilità e la più efficiente migrazione possibile, sempre nel
rispetto dei parametri di gara su sicurezza e sovranità, PSN ha avviato con il DTD un cantiere
definito di “esplicitazione” con la finalità di integrare nel listino:
• Nuove SKU di prodotti esistenti che non erano stati esplicitati per volumi di ordini ridotti
all’epoca della gara
• Modifiche apportate alle SKU da parte dell’Hyperscaler a prodotti già presenti
nell’attuale listino
• Servizi di AWS (Amazon Web Services) come quarto Hyperscaler
Al netto dell’introduzione di AWS, gli esiti del cantiere di esplicitazione hanno portato ad una
maggiore granularità delle SKU relative ad Azure e Google Cloud Platform.
Si riportano di seguito le tabelle riassuntive con le numeriche delle principali modifiche ad oggi;
per l’elenco aggiornato e completo delle SKU si rimanda alla relazione tecnica del cantiere di
Esplicitazione (“Relazione_Tecnica_Esplicitazione_Listini_Secure_Public_Cloud_PSN_con
Allegato.pdf”).
Servizi Google Cloud Platform
39
Servizi Microsoft Azure
3.3 I nuovi servizi SPC GCP ed Azure oggetto di questa integrazione
L’offerta Secure Public Cloud, prevista in Convenzione ed evoluta dal cantiere esplicitazione,
viene ora integrata con i servizi rilasciati da GCP ed Azure dopo la finestra temporale di agosto
2022, “perché frutto dell’evoluzione tecnologica”, come previsto dll’art. 5, comma 4 della
Convenzione. Questi nuovi servizi sono l’oggetto della presenta relazione tecnica.
I nuovi servizi SPC GCP ed Azure vengono integrati nell’offerta PSN al fine di rendere il servizio
flessibile e incrementale nel tempo sulla base delle naturali evoluzioni apportate dagli
Hyperscaler garantendone al contempo sempre la massima sicurezza, così che le
Amministrazioni possano usufruire di un’offerta completa e continuamente aggiornata.
In coerenza con l’Allegato 3, Sub 5 (“Proposta PSN-Listino servizi 2.0”, sezione “Guida Secure
Public Cloud”), è stato condotto un esercizio di integrazione di nuovi prodotti sia per Google
Cloud Platform che per Azure.
Nello specifico, si evidenzia che l’integrazione dei servizi SPC sia di Google Cloud che di
Microsoft Azure, prevede l’introduzione di servizi di tipo PaaS afferenti alle tecnologie di
Intelligenza artificiale. Per il massimo controllo il PSN configurerà per la "distribuzione
regionale", assicurandosi che l'elaborazione e l'archiviazione dei dati avvengano
esclusivamente all'interno delle region “Italy North” dei CSP selezionati in conformità con le
politiche sulla sovranità dei dati.
Questi nuovi servizi di AI sono servizi PaaS, erogati anche in forma di API richiamabili da
processi ed applicazioni. Sono infatti elementi “pre-configurati” che, a differenza di un SaaS,
40
non erogano “applicazioni di business” specifiche per lo svolgimento di uno specifico e ben
definito processo, ma sono abilitatori di funzionalità evolute all’interno di soluzioni applicative.
Resta infatti alle PA il compito di adattare questi PaaS “pre-configurati” al proprio contesto e
di utilizzarli per generare modelli di intelligenza artificiale adatti a svolgere un ruolo nei singoli
processi di interesse. I SPC forniscono alle PA servizi evoluti di piattaforma (es. Speech to text,
Text to speech, Image recognition, modelli di linguaggio LLM, …) assorbendo e “nascondendo”
la complessità di ingegnerizzazione di questi servizi, che possono quindi essere richiamati in
modo programmatico da applicazioni e processi della PA.
Tali servizi vengono integrati come frutto della recente evoluzione tecnologica, offrendo alle
PA soluzioni native in cloud per la creazione di nuovi processi digitali e per abilitare nuovi livelli
di efficienza nella PA. Inoltre, tali servizi sono funzionali a supportare le ri-progettazione / refactoring
di applicazioni tradizionali che elaborano informazioni e dati non strutturati con
tecnologie tradizionali.
In fase di definizione progettuale il PSN supporterà la PA interessata nella identificazione e
selezione dei servizi che risulteranno essere tecnicamente interdipendenti dal punto di vista
tecnico e/o gestionale (come, ad esempio, per il servizio Vertex AI di Google).
A titolo esemplificativo, su SPC Azure si sottolinea come alcune risorse siano fondamentali per
abilitare o supportare altre funzionalità nell'ecosistema cloud. Ad esempio, Azure Active
Directory è essenziale per la gestione delle identità e degli accessi, fornendo una base per
servizi come Azure Virtual Machines, App Services. Allo stesso modo, la configurazione di una
rete virtuale (Azure Virtual Network) è necessaria per garantire la comunicazione sicura tra
risorse distribuite. Molti servizi, come Azure Monitor, richiedono dipendenze configurate, come
Log Analytics, per raccogliere e analizzare dati. Comprendere queste interdipendenze aiuta a
pianificare correttamente l'infrastruttura, evitando errori e garantendo una gestione scalabile,
sicura e ottimizzata. Una progettazione che tenga conto delle propedeuticità consente di
risparmiare tempo e risorse, migliorando l'efficienza operativa.
Tutti i servizi delle famiglie Application Service e Application Platform Services sono erogati in
modalità PaaS in quanto il modello di servizio è basato sulla disponibilità di una piattaforma
che abilita le PA (e ai loro sviluppatori) all'utilizzo degli ambienti di sviluppo completi senza
doversi preoccupare dell'infrastruttura sottostante. In particolare, Apigee è una piattaforma
completa pensata per la gestione delle API (Application Programming Interface) che offre
strumenti e servizi per sviluppare, proteggere, analizzare e scalare le API.
I servizi Secure Public Cloud sono basati su un modello cloud unmanaged dove all'interno del
Tenant la PA ha la responsabilità della gestione dei suoi servizi. Tramite l'acquisto del servizio
di Supporto, sarà consentito alle PA l'accesso alla console tecnica del Cloud Service Provider
per l'apertura della richiesta di supporto sui servizi CSP afferenti ai tenant in gestione al Cliente.
Il servizio consente di fornire un livello di assistenza avanzato, garantendo un supporto tecnico
24 ore su 24, 7 giorni su 7.
41
3.3.1 Nuovi servizi SPC GCP
Relativamente ai servizi SPC di Google Cloud Platform, l’integrazione dell’offerta PSN riflette 4
principali evoluzioni tecnologiche occorse:
• Evoluzione tecnologica dei prodotti presenti nel catalogo dell’hyperscaler nella region
di Milano (europe-west8);
• Rilascio e nuova disponibilità di una seconda Region ubicata a Torino (europe-west12);
• Introduzione di nuovi servizi di supporto erogati direttamente dall’Hyperscaler;
• Esplicitazione della voce di listino “Sovereign Controls ”: tramite la configurazione
“Sovereign Controls “ è possibile impostare delle policy all’interno di GCP per definire /
restringere la localizzazione e memorizzazione dei dati, la crittografia e avere accesso
ai log di sistema che tracciano il consumo delle chiavi crittografiche (Key Access
Justifications – KAJ), inoltre è possibile definire la localizzazione del personale di
supporto e assistenza che deve ad esempio rispondere da una certa area geografica e
monitorare in modo continuativo la posture di conformità applicata all’interno
dell’ambiente GCP.
Per l’elenco esaustivo delle singole SKU si rimanda al relativo allegato
Oltre alle SKU GCP recentemente rese disponibili dall’Hyperscaler, la presente integrazione
comprende, secondo il razionale di Convenzione di “evoluzione tecnologica”, anche le offerte –
descritte nel seguito nel paragrafo 4.3.3.3 - relative a:
• Database MongoDB per SPC
• Database Oracle per SPC
Le offerte MongoDB ed Oracle per SPC saranno esplicitate in termini di specifiche SKU non
appena le SKU verranno rese note e disponibili dagli Hyperscaler.
Viene riportata di seguito una tabella riepilogativa dell’integrazione del listino Secure Public
Cloud – GCP con le numeriche relative ai servizi attualmente a listino ed alle integrazioni
effettuate.
42
• Servizi già presenti nel listino PSN di Luglio 2024: 2009
• Servizi integrati secondo il razionale di Convenzione di “evoluzione tecnologica”: 2477
Non sono da segnalare dismissioni o sostituzioni di servizi precedenti
La presente integrazione delle SKU SPC GCP introduce – in virtù della recente evoluzione
tecnologica - le seguenti nuove macrotipologie:
o AI
o API
o SecOps
o Support
Le macrotipologie “API”, “SecOps” e “Support” sono da ritenersi “inscindibili” da un punto di
vista gestionale dai servizi della altre macrotipologie già presenti nell’offerta di Convenzione,
in quanto sono abilitanti per consentire alle PA aderenti di accedere a strumenti necessari alla
migrazione infrastrutturale dei workload. Qualora le PA, non potendo accedere a tali servizi,
dovessero ricorrere a soluzioni alternative o workaround, non potrebbe essere garantito l’esito
delle migrazioni stesse.
È il caso, ad esempio, dei servizi API, necessari per la migrazione di applicazioni modernizzate,
o SecOps necessari per la gestione sicura dei workload una volta migrati.
Tali servizi consentono inoltre al PSN di ingegnerizzare una piattaforma coerente ed
omogenea, e quindi di esercitare efficacemente il proprio ruolo di “esclusività”, avendo la
visibilità end-to-end di tutto l’ambiente erogato alle PA, dalle catene SecOps necessarie per
l’ingegnerizzazione dei servizi da parte delle PA, ai sistemi di API gateway necessari per
esporre le applicazioni in esecuzione sui servizi infrastrutturali IAAS “puri”, fino ai servizi di
Support necessari per il controllo ed il mantenimento operativo dell’infrastruttura dele PA che
ne fanno richiesta.
43
La macrotipologia di PaaS “AI” è frutto dell’evoluzione tecnologica, abilita lo sviluppo e
l’implementazione da parte delle PA di applicazioni AI / GenAI e può, in alcuni casi, supportare
la riprogettazione delle applicazioni per la migrazione.
Si conferma che tutti i servizi PaaS AI sono erogati su region Italiana: europe-west8 (Milano) e
europe-west12 (Torino. In particolare, il servizio GCP Vertex Colab è disponibile nella sola
region europe-west12 (Torino).
44
3.3.2 Nuovi servizi SPC Azure
Relativamente ai servizi SPC di Microsoft Azure, l’integrazione dell’offerta PSN riflette 3
principali evoluzioni tecnologiche occorse:
• Evoluzione tecnologica dei prodotti presenti nel catalogo dell’hyperscaler;
• Localizzazione dei servizi nella nuova Region di Milano (Italy North) avvenuta
nell’ultimo trimestre del 2023;
• Introduzione del nuovo catalogo di servizi Cognitive Services, particolarmente
interessanti sono i servizi di Azure OpenAI;
Per l’elenco esaustivo delle singole SKU si rimanda al relativo allegato. I servizi presenti nel
listino deputati all’erogazione di applicazioni di Business, come le Application Platforms o i
DevOps Services, saranno erogati in un contesto PaaS consentendo alle Amministrazioni di far
parte di un ecosistema più ampio di strumenti e piattaforme basate su Azure, In questo modo
le risorse Azure incluse nella sottoscrizione potranno essere utilizzate per costruire, distribuire
e gestire applicazioni personalizzate.
Oltre alle SKU Azure recentemente rese disponibili dall’Hyperscaler, la presente integrazione
comprende, secondo il razionale di Convenzione di “evoluzione tecnologica”, anche le offerte –
descritte nel seguito nel paragrafo 4.3.3.3 - relative a:
• Database MongoDB per SPC
• Database Oracle per SPC
Le offerte MongoDB ed Oracle per SPC saranno esplicitate, in termini di specifiche SKU, non
appena le SKU verranno rese note e disponibili dagli Hyperscaler
Viene riportata di seguito una tabella riepilogativa dell’integrazione del listino SPC – Azure con
le numeriche relative ai servizi attualmente a listino e alle integrazioni effettuate.
45
• Servizi già presenti nel listino di Luglio 2024: 6219
• Servizi integrati secondo il razionale di evoluzione tecnologica: 2210
Non sono da segnalare dismissioni o sostituzioni di servizi precedenti. A titolo di esempio, non
esaustivo, citiamo come integrazione rispetto al precedente listino i servizi di Application
platform. Questi servizi offrono un framework preconfigurato per la formazione dei discenti.
Sono tutti elementi “pre-configurati” che, a differenza di un SaaS, non erogano “applicazioni di
business” necessarie per lo svolgimento di uno specifico e ben definito processo, ma sono
abilitatori di funzionalità evolute all’interno di soluzioni applicative fornendo un framework
preconfigurato che può essere personalizzato per ospitare e distribuire contenuti educativi
specifici per la PA con controllo completo sui contenuti e sull'infrastruttura, senza dover gestire
direttamente l'hardware o la rete. Inoltre, per i servizi SPC - Azure: Advanced Messaging e per
il servizio Channel Fee - Whatapp il PSN fornirà accesso alla Console tecnica di Azure dedicata
dove la PA potrà in autonomia configurare i servizi. Molti di questi nuovi servizi Azure sono
interdipendenti e costituiscono le basi necessarie per implementarne altri. La propedeuticità
dei servizi Azure sottolinea come alcune risorse siano fondamentali per abilitare o supportare
altre funzionalità nell'ecosistema cloud. Ad esempio, Azure Active Directory è essenziale per
la gestione delle identità e degli accessi, fornendo una base per servizi come Azure Virtual
Machines, App Services. Allo stesso modo, la configurazione di una rete virtuale (Azure Virtual
Network) è necessaria per garantire la comunicazione sicura tra risorse distribuite. Molti
servizi, come Azure Monitor, richiedono dipendenze configurate, come Log Analytics, per
raccogliere e analizzare dati. Comprendere queste interdipendenze aiuta a pianificare
correttamente l'infrastruttura, evitando errori e garantendo una gestione scalabile, sicura e
ottimizzata. Una progettazione che tenga conto delle propedeuticità consente di risparmiare
tempo e risorse, migliorando l'efficienza operativa.
Rispetto all’offerta di Convenzione, l’unica nuova macro-tipologia introdotta, in virtù della
recente evoluzione tecnologica, è:
o Support
La macro-tipologia “Support” è da ritenersi “inscindibile”, da un punto di vista gestionale, dalle
altre macro-tipologie esistenti, in quanto costituisce un requisito tecnico necessario per la
migrazione e il successivo controllo e mantenimento operativo dell’infrastruttura delle PA che
ne fanno richiesta.
3.3.3 Database Oracle e MongoDB per SPC
Ad integrazione delle nuove SKU rese disponibili da Azure, GCP e AWS relativamente alla
tipologia “Database”, si riportano a titolo esemplificativo alcuni servizi:
• MongoDB Atlas, soluzione database, per gestire e scalare applicazioni basate su
MongoDB direttamente nel secure public cloud. Atlas offre un’implementazione
46
completamente gestita di MongoDB, riducendo i costi e il tempo associati alla gestione
dell’infrastruttura e al monitoraggio delle performance. Le principali caratteristiche sono
le seguenti:
1. Integrazione con secure public cloud: MongoDB Atlas viene distribuito nelle
region italiane di Azure, GCP, AWS migliorando la latenza e garantendo la
conformità ai requisiti di residenza dei dati.
2. Automazione del Cluster: MongoDB Atlas automatizza la creazione e la
gestione dei cluster MongoDB, tra cui:
▪ Provisioning dei server: Atlas crea i server virtuali e installa il software
MongoDB.
▪ Configurazione: Il servizio gestisce la configurazione ottimale dei cluster.
▪ Aggiornamenti automatici: Le patch di sicurezza e gli aggiornamenti
vengono gestiti automaticamente.
▪ Scalabilità automatica: Supporta la scalabilità automatica sia in termini
di storage che di risorse di calcolo (verticale e orizzontale).
3. Distribuzione multi-zona: Atlas sfrutta le zone di disponibilità nelle regioni
italiane di Azure, GCP e AWS per implementare replica set distribuiti in più
“Availability Zones”, migliorando la tolleranza ai guasti e l'alta disponibilità.
4. Gestione dei dati:
▪ Backups automatici: MongoDB Atlas offre backup automatici e continui,
con punti di ripristino, consentendo di recuperare i dati da diversi
momenti temporali.
▪ Replica e Sharding: Supporta replica set per la disponibilità e sharding
per la distribuzione scalabile di dati su più server.
5. Monitoraggio e Ottimizzazione: Atlas offre strumenti di monitoraggio integrato
per tracciare le metriche di performance del database, come utilizzo della CPU,
memoria, I/O disco, e tempi di risposta delle query. Inoltre, suggerisce
ottimizzazioni automatiche basate su questi dati.
6. L'integrazione di HSM (Hardware Security Module) con MongoDB Atlas
permette di crittografare i dati dei database MongoDB all’interno del secure
public cloud. I dati risiederanno nelle regioni italiane del Secure Public Cloud ma
le chiavi per decriptare tali dati saranno custodite all’interno delle region del
PSN. Questo aggiunge un ulteriore livello di sicurezza e conformità.
• Oracle Database@Azure/GCP/AWS permette di eseguire i database Oracle
utilizzando l'infrastruttura del secure public cloud, combinando le funzionalità avanzate
di Oracle con i propri servizi. Le principali caratteristiche sono le seguenti:
1. Integrazione con secure public cloud: Oracle Database@Azure/GCP/AWS è un
servizio di database Oracle ospitato all'interno del Secure public cloud, situato
nella Region North Italy. Questa soluzione assicura che Oracle
47
Database@Azure/GCP/AWS possa accedere rapidamente alle risorse e alle
applicazioni del Secure public cloud, minimizzando i tempi di latenza tra i
workload applicativi e quelli Oracle.
Tale infrastruttura permette il deploy di servizi Oracle nativi come come Oracle
Autonomous Database, Oracle Exadata, Oracle RAC e servizi PAAS Oracle,
offrendo notevoli vantaggi in termini di scalabilità, efficienza e automazione.
Oracle Autonomous Database, automatizza operazioni di provisioning e la
gestione dei database. Oracle Exadata fornisce prestazioni elevate per i carichi
di lavoro intensivi, ottimizzando le operazioni di database su larga scala. Oracle
RAC (Real Application Clusters) consente alta disponibilità e affidabilità,
distribuendo i carichi su più nodi, mentre i servizi PAAS Oracle facilitano lo
sviluppo e la gestione delle applicazioni con strumenti integrati, alleggerendo la
gestione operativa.
2. Scalabilità e Prestazioni: La soluzione consente di scalare le risorse di calcolo e
storage per supportare carichi di lavoro variabili. L’infrastruttura viene di default
ottimizzata per garantire le best practice Oracle, garantendo un uso efficiente
delle risorse.
3. Backup: È possibile configurare backup automatici di Oracle Database tramite
“Oracle Recovery Manager (RMAN), che consente di pianificare backup regolari
e offre opzioni di replica per una maggiore sicurezza. Queste funzionalità
assicurano la continuità operativa e garantiscono un rapido recupero dei dati in
caso di necessità.
4. Migrazione: È possibile migrare facilmente i workload dall'on-premise a
Oracle@Azure/GCP/AWS utilizzando strumenti nativi Oracle come "Active Data
Guard". Inoltre, per garantire alta affidabilità e replica, è possibile avvalersi di
soluzioni come Oracle GoldenGate, che assicurano continuità operativa e
sincronizzazione dei dati in tempo reale.
5. L'integrazione di un HSM (Hardware Security Module) con Oracle
Database@Azure/GCP/AWS permette di gestire in modo sicuro le chiavi di
crittografia utilizzate per proteggere i dati. I dati risiederanno nelle regioni
italiane del Secure Public Cloud ma le chiavi per decriptare tali dati saranno
custodite all’interno delle region del PSN. Questo aggiunge un ulteriore livello di
sicurezza e conformità.
48
3.3.4 Nuovi servizi Secure Public Cloud Confidential Azure
Il servizio Confidential Public Cloud IaaS è un servizio cloud basato sulla region Italiana
pubblica dell’Hyperscaler Microsoft Azure e i datacenter PSN.
La soluzione Confidential Public Cloud IaaS è progettata per fornire servizi cloud pubblici con
un alto livello di sicurezza e sovranità digitale. Questa soluzione include strumenti di gestione
delle chiavi di crittografia, governance e sicurezza per garantire che i dati siano protetti in ogni
fase del loro utilizzo. La piattaforma offre servizi IaaS all’interno dei datacenter dell’hyperscaler
garantendo la protezione delle informazioni basate su confidential computing, OS immutabili
e chiavi di crittografia.
Tali servizi riducono ulteriormente potenziali rischi legati alla sicurezza dei servizi offerti da
PSN pur mantenendo l’attuale classificazione di tale servizio per dati critici (classificazione
ACN), diversamente da altri servizi utili alla gestione di dati strategici (es., hybrid cloud).
Come anticipato, la soluzione Confidential Public Cloud IaaS presenta le seguenti
caratteristiche:
• Confidential Computing: nella soluzione proposta sono disponibili all’utente solo le
soluzioni basate su confidential computing che al momento sono VM Confidential,
Ledger Confidential, Databricks Confidential, K8S Confidential. Oltre a queste
soluzioni il cliente può inserire solo soluzioni di networking e soluzioni che non
memorizzano dati;
Tracciatura e monitoraggio dei sorgenti: PSN tramite i soci ha stretto un accordo con
Microsoft che consente di accedere alla piattaforma Microsoft Code Center Premium (CCP) in
qualità di Partner. CCP è un portale web sicuro che consente ai Partner l'accesso controllato al
codice sorgente di specifici prodotti Microsoft. Fa parte della Microsoft Shared Source Initiative
e offre visibilità su tecnologie strategiche come Microsoft Azure Attestation e soluzioni di
Confidential Computing. Il portale supporta i Partner selezionati nell'analisi e nella verifica del
codice, migliorando trasparenza e sicurezza. L'accesso just-in-time è regolato da criteri rigorosi
per garantire la protezione della proprietà intellettuale, un elevato livello di trasparenza e
accessibilità e la possibilità di visualizzare, sfogliare, cercare e fare riferimento al codice
sorgente del prodotto. Questo accesso privilegiato consente di analizzare il codice sorgente dei
prodotti Microsoft utilizzati e ciò garantisce la massima trasparenza;
• HYOK: oltre alla soluzione BYOK presente nello scenario Secure Public Cloud, in questa
soluzione viene aggiunta anche una implementazione HYOK basata su macchina
virtuale confidential in cui gira il sistema Hascicorp Vault. Le chiavi in solo possesso di
PSN (Leonardo) vengono utilizzate per cifrare le soluzioni confidential computing;
Mentre nel modello BYOK, il CSP utilizza le chiavi crittografiche generate dalla PA o
dal PSN per cifrare e decifrare i dati all’interno del proprio ambiente Cloud, nel
paradigma HYOK, il CSP non ha mai accesso alle chiave crittografiche del cliente, che
49
vengono conservate all’interno di una VM “Confidential” nella soluzione Hashicorp
Vault indipendente dal CSP e vengono utilizzate per le operazioni di cifratura e
decifratura dal PSN, mai dal CSP.
Autorizzazione accesso al sistema: tramite lo strumento di Azure LockBox è possibile
richiedere l’autorizzazione cliente/PSN prima che il cloud provider agisca sui sistemi interessati.
Inoltre, nello strumento di trasparency log vengono inserite tutte le azioni fatte sui sistemi che
ospitano i dati cliente; il servizio viene messo a disposizione nel novero dei servizi di Cloud for
Sovereignty e permette di ricevere notifiche per gli interventi che gli operatori di Azure
effettuano direttamente sulle risorse dei clienti (interventi molto rari che vengono
preventivamente autorizzati dal cliente stesso tramite il meccanismo di Just in Time Access).
Questo servizio fornisce un registro verificabile dalla PA, che documenta le operazioni eseguite
sui servizi acquistati dall'Amministrazione garantendo la trasparenza e la possibilità di audit
indipendenti. Il servizio Transparency Logs traccia esclusivamente azioni del Cloud service
provider a livello di piattaforma Azure, tramite l’invio di un report via mail (primo mercoledì di
ogni mese) permettendo una maggiore visibilità sull'operato del Cloud Provider. Il servizio
complementa tutti gli altri sistemi di raccolta log già in essere sul Secure Public Cloud, ovvero
Security Logs, Activity Logs, Resource Logs ed Entra ID Logs.
Per il servizio Transparency Logs la conservazione non è direttamente configurabile dai clienti
ma è progettata per fornire un registro verificabile fino a 90 giorni dalla generazione del report.
Architettura Logica Hub&Spoke: Per quanto riguarda l’ambiente Public cloud è previsto l’uso
di un modello Hub & Spoke per consentire al PSN il controllo del traffico e la gestione delle DMZ
per l’ambiente cloud. Le Amministrazioni potranno creare reti virtuali spoke nei segmenti, dove
saranno attive Policy che forzeranno la connessione con Virtual Network Hub e impediranno
la creazione di tipologie di risorse controllate centralmente, come, ad esempio, gli indirizzi IP
pubblici.
50
La piattaforma SPC Confidential Azure è quindi volta a fornire alle PA aderenti un livello ancora
più elevato di protezione per i dati delle PA.
Nel caso dell’offerta SPC Confidential Azure, sono ricomprese solo SKU stateless oppure
“confidential” dell’offerta Azure, cioè servizi di calcolo che non necessitano di decriptare le
informazioni per poterle elaborare.
Le SKU dell’offerta SPC Confidential Azure sono una porzione del listino SPC Azure (es. le
stesse VM di tipo “Confidential” nelle due offerte SPC di PSN), ma il servizio complessivo
erogato alla PA a parità di SKU è differente per via della configurazione integrale dell’ambiente
messo a disposizione delle PA. Per facilitare la comprensione univoca del servizio
corrispondente alla SKU Azure, la è stata mantenute la tassonomia delle singole SKU identica
nelle due offerte PSN SPC Azure e PSN SPC Confidential Azure.
\end{document}