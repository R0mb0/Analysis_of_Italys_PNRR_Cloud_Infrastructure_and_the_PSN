\documentclass{article}
\usepackage[utf8]{inputenc}
\usepackage[italian]{babel}
\usepackage{amssymb}
\usepackage{amsmath}
\usepackage{color}
\usepackage[pdftex]{graphicx}
\usepackage[svgnames]{xcolor}
\usepackage{array}
\usepackage{parskip}
\usepackage[margin=1in]{geometry}
\usepackage[T1]{fontenc}
\usepackage[many]{tcolorbox}
\usepackage{enumitem}
\usepackage{hyperref}
\usepackage{appendix}
\usepackage{fancyhdr}
\usepackage{titling}
\usepackage{authblk}

\usepackage{minted} % Pacchetto per evidenziare la sintassi
\usepackage{tcolorbox} % Per riquadri personalizzabili

\usepackage{pgfplots}
\pgfplotsset{compat=1.18} % o una versione compatibile con la tua TeX

\usepackage{caption}

% Configurazione globale per minted: riquadro, numeri di riga, font più piccolo e spezzatura aggressiva
\setminted{
	frame=single,
	linenos,
	fontsize=\footnotesize, % o \scriptsize se vuoi ancora più margine
	breaklines,
	breakanywhere,
	numbersep=4pt,      % distanza tra numeri di riga e codice
	xleftmargin=6pt,    % margine sinistro interno al frame
	framesep=2pt        % distanza testo-bordo
}

\usepackage{csquotes} % consigliato con biblatex
\usepackage{biblatex}
\addbibresource{riferimenti.bib}

\title{\color{FireBrick}\bf{Realizzazione di una applicazione OCR per manoscritti italiani mediante Kraken}}
\author[1]{\color{FireBrick}\bf{Francesco Rombaldoni}}
\affil[1]{f.rombaldoni@campus.uniurb.it}

\date{}

\begin{document}
	\fancypagestyle{firstpage}
	{
		\fancyhead[L]{\footnotesize{\bf{Universit\`a degli Studi di Urbino Carlo Bo}}}
		\fancyhead[R]{\footnotesize{\bf{CdL Magistrale Informatica e Innovazione Digitale}}}
	}
	\thispagestyle{firstpage}
	
	\pagestyle{fancy}
	
	\fancyhead{} % clear all header fields
	\fancyhead[L]{\color{Black}{\footnotesize{\thetitle}}}
	\fancyfoot{} % clear all footer fields
	\fancyfoot[R]{\footnotesize{\bf{\thepage}}}
	\fancyfoot[L]{\footnotesize{\bf{Progetto corso Machine Learning}}}
	
	
	
	\twocolumn
	%------------------------------------------                                       
	%                      Title
	%------------------------------------------
	[{
		\maketitle
		\thispagestyle{firstpage}
		\title{\color{Black}\bf{Realizzazione di una applicazione OCR per manoscritti italiani mediante Kraken}}
		%------------------------------------------                                                           
		%                   Abstract
		%------------------------------------------
		\normalsize
		\begin{tcolorbox}[  colback = WhiteSmoke,
			,
			width=\linewidth,
			arc=1mm, auto outer arc,
			]
			\section*{Riassunto}
			
		
		\end{tcolorbox}
		\vspace{1.5ex}
	}]
	
	
	%------------------------------------------
	%                   Main Matter
	%------------------------------------------
	
	\section{Introduzione}
	
	\section{Architettura del modello}
	
	\section{Modello di partenza e difficoltà di documentazione}
	
	\section{Metodi}
	
	\section{Risultati sperimentali}
	
	\section{Conclusioni}
	
	
\end{document}